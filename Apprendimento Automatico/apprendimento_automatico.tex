%----------------------------------------------------------------------------------------
% PACKAGES AND OTHER DOCUMENT CONFIGURATIONS
%----------------------------------------------------------------------------------------

% !TEX encoding = UTF-8
% !TEX TS-program = pdflatex
% !TEX root = apprendimento_automatico.tex
% !TEX spellcheck = it-IT

\documentclass[a4paper, 11pt]{article} % Font size (can be 10pt, 11pt or 12pt) and paper size (remove a4paper for US letter paper)
\usepackage[italian]{babel}      							% Lingua italiana
\usepackage[margin=1.2in]{geometry}             % Imposta i margini del documento

\usepackage[T1]{fontenc} % Required for accented characters
\usepackage[mathletters]{ucs}    % Caratteri matematici come UTF8
\usepackage[utf8]{inputenc}      % Ancora utf8

\usepackage{eurosym}                %simbolo dell'euro
\usepackage{listings}
\usepackage[usenames,dvipsnames,svgnames,table]{xcolor}
% Imposta lo spazio nella list of listing in modo simile alla list of figures/tables
%\makeatletter
%\let\my@chapter\@chapter
%\renewcommand*{\@chapter}{%
%  \addtocontents{lol}{\protect\addvspace{10pt}}%
%  \my@chapter}
%\makeatother


\definecolor{codegreen}{rgb}{0,0.6,0}
\definecolor{codegray}{rgb}{0.5,0.5,0.5}
\definecolor{backcolor}{rgb}{0.98,0.98,0.98}

\renewcommand{\lstlistingname}{Codice}% Listing -> codice
\renewcommand{\lstlistlistingname}{Elenco dei frammenti di codice}% List of Listings -> Frammenti di codice

\lstdefinestyle{mystyle}{
    backgroundcolor=\color{backcolor},   
    commentstyle=\color{Peach}\ttfamily,
    keywordstyle=\color{RoyalBlue},
    numberstyle=\tiny\color{codegray},
    stringstyle=\color{SeaGreen}\ttfamily,
    basicstyle=\footnotesize\ttfamily,
    breakatwhitespace=false,         
    breaklines=true,                 
    captionpos=b,                    
    keepspaces=true,                 
    numbers=left,                    
    numbersep=5pt,                  
    showspaces=false,                
    showstringspaces=false,
    showtabs=false,                  
    tabsize=2,
    frame=trbl, % draw a frame at the top, right, left and bottom of the listing
	frameround=ftff, % angolo in basso a destro curvo
	framesep=4pt, % quarter circle size of the round corners,
	inputencoding=utf8,
    extendedchars=true,
    literate={á}{{\'a}}1 {à}{{\`a}}1 {é}{{\'e}}1 {è}{{\`e}}1 {ù}{{\`u}}1 {ò}{{\`o}}1,
    belowskip=1em,
    aboveskip=1em,
}

 
\lstset{style=mystyle}

\lstdefinelanguage{JavaScript}
{
  % list of keywords
  morekeywords={ true, false, catch, function, break,	new, class, extends, var, require, switch, return, import, if, while, for, this, View, Text, StyleSheet},
  sensitive=false, % keywords are not case-sensitive
  morecomment=[l]{//}, % l is for line comment
  morecomment=[s]{/*}{*/}, % s is for start and end delimiter
  morestring=[b]' % defines that strings are enclosed in double quotes
}

\lstdefinelanguage{JSON}
{
  % list of keywords
  morekeywords={string, boolean, int, Array, Node, Asset, AssetDetail, Filter, FilterItem},
  sensitive=false, % keywords are not case-sensitive
  morecomment=[l]{//}, % l is for line comment
  morecomment=[s]{/*}{*/}, % s is for start and end delimiter
  morestring=[b]" % defines that strings are enclosed in double quotes
}

%\tightlist per compatibilità con pandoc
\providecommand{\tightlist}{%
  \setlength{\itemsep}{0pt}\setlength{\parskip}{0pt}}


\usepackage[labelfont=bf]{caption}

\usepackage[protrusion=true,expansion=true]{microtype} % Better typography
\usepackage{graphicx} % Required for including pictures
\usepackage{wrapfig} % Allows in-line images


\usepackage{subfig}

\usepackage{mathpazo} % Use the Palatino font

\linespread{1.05} % Change line spacing here, Palatino benefits from a slight increase by default
\usepackage{placeins}
\usepackage{sourcecodepro}
\usepackage{hyperref}                   % collegamenti ipertestuali

\usepackage[colorinlistoftodos,prependcaption]{todonotes} %todo

\usepackage{amsmath}

\makeatletter
\renewcommand\@biblabel[1]{\textbf{#1.}} % Change the square brackets for each bibliography item from '[1]' to '1.'
\renewcommand{\@listI}{\itemsep=0pt} % Reduce the space between items in the itemize and enumerate environments and the bibliography

\renewcommand{\maketitle}{ % Customize the title - do not edit title and author name here, see the TITLE block below
\begin{flushright} % Right align
{\LARGE\@title} % Increase the font size of the title

\vspace{50pt} % Some vertical space between the title and author name

{\large\@author} % Author name
\\\@date % Date

\vspace{100pt} % Some vertical space between the author block and abstract
\end{flushright}
}

%----------------------------------------------------------------------------------------
% TITLE
%----------------------------------------------------------------------------------------

\title{\textbf{Apprendimento Automatico}\\ % Title
A.A. 2015-2016} % Subtitle

\author{\textsc{Giacomo Manzoli}
\\ 1130822 % Author
\\{\textit{Università degli Studi di Padova}}} % Institution

\date{\today} % Date

%----------------------------------------------------------------------------------------

\begin{document}

\maketitle % Print the title section

%----------------------------------------------------------------------------------------
% ABSTRACT AND KEYWORDS
%----------------------------------------------------------------------------------------

%\renewcommand{\abstractname}{Summary} % Uncomment to change the name of the abstract to something else

\begin{abstract}
Appunti riorganizzati del corso di Apprendimento Automatico

\end{abstract}

\clearpage
\tableofcontents

%\hspace*{3,6mm}\textit{Keywords:} lorem , ipsum , dolor , sit amet , lectus % Keywords

\vspace{30pt} % Some vertical space between the abstract and first section

%----------------------------------------------------------------------------------------
% ESSAY BODY
%----------------------------------------------------------------------------------------
\clearpage

% !TEX encoding = UTF-8
% !TEX program = pdflatex
% !TEX root = InformationRetrieval.tex
% !TEX spellcheck = it-IT

% 29 Settembre 2016

\chapter{Introduzione}

Lo scopo del reperimento dell'informazione è quello di essere in grado di soddisfare il bisogno informativo di una determinata categoria di utenti, mediante tecniche di ricerca e di memorizzazione dei dati.
Tali dati possono essere in qualsiasi formato: testo, foto, audio, ecc. e possono anche esserci problemi riguardo l'internazionalizzazione dei dati, legati all'utilizzo di lingue diverse.

Si ha quindi che c'è un utente che ha un'esigenza informativa che deve essere colmata reperendo delle informazioni da una collezione di documenti.

\section{Terminologia}

\begin{itemize}
	\item \textbf{Esigenza informativa}: mancanza di informazione che un utente vuole colmare per risolvere un suo problema e prendere delle decisioni.
	\item \textbf{Documento}: inteso come ``materializzazione'' dell'informazione, è quasi un oggetto che contiene informazioni utili a soddisfare l'esigenza informativa.
	\item \textbf{Rilevanza}: proprietà di un ``documento'' di contenere l'informazione utile e necessaria a soddisfare una specifica esigenza informativa dell'utente.
	\item \textbf{Sistema di Reperimento dell'informazione}: sistema che a partire da una query e da un insieme di documenti, utilizza delle specifiche rappresentazione per valutare la similarità tra i documenti e la query, in modo da fornire in risposta all'utente un insieme ordinato di documenti secondo un determinato ranking. Può esserci anche un sistema di feedback basato sulle preferenze dell'utente (assessment).
\end{itemize}


\section{Informazioni organizzative}

Orario delle lezioni

\begin{itemize}
	\item Giovedì dalle 10:30 - Aula Oe
	\item Venerdì dalle 12:30 - Aula Oe
\end{itemize}

Ci saranno degli homework che riguardano:
\begin{itemize}
	\item Un progetto individuale di valutazione di un sistema di reperimento dell'informazione. 
	\item Readings:  si studia uno o due articoli e si fa una presentazione stile seminariale.
\end{itemize}

Il ricevimento è per appuntamento.

Comunicazioni via mail a \url{maristella.agosti@unipd.it}, specificando come oggetto \texttt{[IR2016-2017]} e il nome/cognome/matricola.


\subsection{Modalità d'esame}

\begin{itemize}
	\item Esame scritto su tutto il programma svolto, da 0 a 20 punti, con sufficienza a 12.
	\item Progetto individuale di valutazione, da 0 a 5 punti.
	\item Readings, da 0 a 5 punti.
	\item Eventuale integrazione orale.
\end{itemize}

La pagina di riferimento del corso è \url{http://www.dei.unipd.it/~agosti/ir-2016-2017/}. 
C'è anche il corso sul Moodle del DEI: la password del corso è \texttt{corso-IR201617}.

Testi di riferimento:
\begin{itemize}
	\item Information Retrieval (2ed). Cornelis Joos. 1979. \url{http://www.dcs.gla.ac.uk/Keith/Preface.html}.
	\item Information Retrieval in Practice. W. Bruce Croft. 2010. \url{http://ciir.cs.umass.edu/irbook/}
\end{itemize}


\section{Programma e quadro d'insieme}

L'obiettivo del corso è quello di fornire le competenze necessario per la ideazione, progettazione e implementazione di un sistema di reperimento dell'informazione.

Una parte importante del corso sarà dedicata all'indicizzazione: la parte del sistema che si occupa di gestire la rappresentazione dei documenti e della query dell'utente, specialmente per quanto riguarda l'ambito testuale.

Una volta effettuata l'indicizzazione viene definito un modello di valutazione che è la componente del sistema che si occupa di valutare la similarità tra la query e i documenti.

Il modello non è solo un algoritmo, ma c'è anche un insieme di costrutti che sono formalizzati allo scopo di rendere possibile la rappresentazione del contenuto dei documenti e delle interrogazioni. 

La valutazione del sistema si basa sull'efficacia e riguarda la coerenza che c'è tra i documenti reperiti e la richiesta dell'utente, così come ci sono metodi di valutazione per la parte legata all'indicizzazione e al modello.

\subsection{Programma}

\begin{itemize}
	\item Elementi introduttivi per la rappresentazione, gestione e reperimento automatico dell'informazione testuale.
	\item Indicizzazione: strutture dati idonee al reperimento dell'informazione.
	\item Modelli e sistemi per il reperimento dell'informazione.
	\item Valutazione: collezioni sperimentali, misure di efficacia e efficienza.
\end{itemize}

% !TEX encoding = UTF-8
% !TEX TS-program = pdflatex
% !TEX root = computabilità e algoritmi.tex
% !TEX spellcheck = it-IT
\chapter{Lezione 2}
\section{Introduzione e Algoritmi sui grafi}\label{lezione-2---introduzione-e-algoritmi-sui-grafi}

C'è la possibilità di fare un pre-orale nella settimana dei compitini.

Libro: Cormen, Introduzione agli algoritmi e strutture dati
\href{http://catalogo.unipd.it/F/FCKK1DACESL2TDH5CF15FLDL2BUM936U1XG9U15MFDCKI764BV-10675?func=full-set-set\&set_number=011139\&set_entry=000001\&format=999}{BIB}:

\begin{itemize}
\tightlist
\item
  Introduzione agli algoritmi sui grafi

  \begin{itemize}
  \tightlist
  \item
    Strutture dati per i grafi
  \item
    Operazioni elementari sui grafi
  \end{itemize}
\item
  Algoritmi su stringhe (capitolo 22)
\item
  Algoritmi paralleli (capitolo 27)
\item
  Algoritmi di geometria computazionale
\end{itemize}

\subsection{Terminologia dei grafi}\label{terminologia-dei-grafi}

Un grafo \emph{G} è costituito da un insieme di vertici \emph{V} e di
archi \emph{E}. Ad ogni arco vengono associati due vertici in \emph{V}.

Se c'è un ordine tra i due estremi degli archi, il grafo prende il nome
di \textbf{orientato} o \textbf{diretto}. In questo caso, il primo
vertice prende il nome di \textbf{coda} e l'ultimo \textbf{testa}.

Un \textbf{cappio} è un grafo i cui due estremi coincidono.

Un grafo non orientato si dice \textbf{semplice} se non ha cappi e non
ci sono due archi con gli stessi estremi. Mentre se il grafo è
orientato, perché sia semplice non devono esserci archi con gli stessi
estremi, iniziali e finali. Un grafo non semplice prende il nome di
\textbf{multi-grafo}.

\begin{figure}[htbp]
\centering
\includegraphics[width=0.75\textwidth]{./notes/immagini/l2-grafi.png}
\caption{Varie tipologie di grafi}
\end{figure}

Se un grafo è semplice, un arco può essere espresso con:

$$
e = uv \in E \text{, con} u,v \in V
$$

e si dice che l'arco \emph{e} è incidente in \emph{u} e \emph{v}. Da
notare che se il grafo è orientato
$e = uv \neq vu$ e la terminologia diventa
``l'arco \emph{e} esce da \emph{u} entra in \emph{v}''.

Il \textbf{grado} di un vertice \emph{v} viene indicato con $\delta(v)$ e rappresenta il numero di archi incidenti in quel
vertice. Se il grafo è ordinato, il suo \textbf{grado uscente} $\delta^+(v)$ è il numero di archi uscenti e il suo \textbf{grado entrante} è $\delta^-(v)$.

Se due vertici sono collegati da un arco, questi vengono detti
\textbf{adiacenti}.

Un \textbf{cammino} di lunghezza \emph{k} da un vertice \emph{u} ad un
vertice \emph{v} in un grafo \emph{G=(V,E)}, è una sequenza di
\emph{k+1} vertici $x_0 \ldots x_k$, tali che $x_0 = u$, $x_k = v$ e $x_{i-1}x_i \in E \forall i = 1\ldots k$.

Se il cammino ha lunghezza 0, questo viene detto \textbf{nullo}, mentre
se il vertice di partenza coincide con quello di arrivo, il cammino
prende il nome di \textbf{chiuso}.

Un cammino viene detto \textbf{semplice} quanto tutti i vertici che lo
compongono sono distinti, ad eccezione del primo, che può coincidere
con l'ultimo. Un cammino semplice e il primo vertice coincide con
l'ultimo, questo prende il nome di \textbf{ciclo}. L'esempio più
semplice di ciclo è dato da un cappio.

Un grafo \textbf{aciclico} è un grafo che non contiene cicli.

Quando esiste almeno un cammino dal vertice \emph{u} al vertice
\emph{v}, si dice che \emph{v} è \textbf{accessibile} (o
\textbf{raggiungibile}) da \emph{u}. Questa definizione è simmetrica
solamente nel caso di un grafo non orientato.

Un grafo non orientato si dice \textbf{connesso} se esiste almeno un
cammino tra ogni coppia di vertici.

Le \textbf{componenti connesse} di un grafo sono le classi di
equivalenza dei suoi vertici rispetto alla relazione di accessibilità,
ovvero un sottoinsieme di vertici che sono tutti tra loro accessibili.

Nel caso di un grafo orientato, si dice che è \textbf{fortemente
connesso} se esiste almeno un cammino tra ogni vertice del grafo. In
modo analogo è possibile definire le \textbf{componenti fortemente
connesse}

Sia la \textbf{connessione} che la \textbf{connessione forte} hanno le
proprietà:

\begin{itemize}
\item
  \textbf{riflessiva}: se c'è una connessione tra \emph{u} e \emph{v},
  c'è anche tra \emph{v} e \emph{u}
\item
  \textbf{transitiva}: se c'è una connessione tra \emph{u} e \emph{v} e
  tra \emph{v} e \emph{z}, allora c'è anche tra \emph{u} e \emph{z}.
\end{itemize}

Un sotto-grafo del grafo \emph{G=(V,E)} è un grafo \emph{G' = (V', E')}
tale che:

$$
V' \subseteq V \: \text{e} \: E' \subseteq \{ uv : uv \in E \text{ e } u,v \in V' \}
$$

ovvero un grafo che ha alcuni vertici e alcuni archi del grafo iniziale.
Da notare che se tolgo un vertice, devo togliere anche tutti gli archi
incidenti in quel vertice.

Se il sotto-grafo viene ottenuto rimuovendo solo dei vertici, questo
prende il nome di \textbf{indotto}, perché la rimozione degli archi
viene forzata dalla rimozione dei vertici.

\subsection{Rappresentazione dei
grafi}\label{rappresentazione-dei-grafi}

Per rappresentare i grafi in un calcolatore è possibile utilizzare la
matrice delle adiacenze o la lista delle adiacenze.

\begin{figure}[htbp]
\centering
\includegraphics[width=0.75\textwidth]{./notes/immagini/l2-rappr.png}
\caption{Rappresentazione dei grafi}
\end{figure}

\subsubsection{~Lista delle adiacenze}\label{lista-delle-adiacenze}

Per ogni vertice del grafo viene tenuta in memoria una lista \textit{Adj} dei vertici
adiacenti al vertice:

$$
Adj[u] = \{v | uv \in E\} \: \forall u \in V 
$$

Questa rappresentazione richiede memoria per:

\begin{itemize}
\tightlist
\item
  \textit{n\ =\ \textbar{}V\textbar{}} puntatori alla cima delle liste
\item
  \textit{m\ =\ \textbar{}E\textbar{}} elementi per le liste (in totale)
  se il grafo orientato, se è non orientato è \textit{2m}.
\end{itemize}

\subsubsection{Matrice delle adiacenze}\label{matrice-delle-adiacenze}

Viene utilizzata una matrice booleana quadrata che tante righe e tante
colonne, quanti sono i vertici del grafo.

Ogni elemento della matrice vale 1 se i due vertici sono adiacenti, 0
altrimenti:

$$
a_{u,v} = 1 \text{ se } uv \in E
$$

Se il grafo è non orientato, la matrice delle adiacenze è simmetrica.

Il consumo di memoria è $n^2$.

Se il grafo è \textbf{sparso}, ovvero il grado dei vertici è minore del
logaritmo del numero dei vertici, la matrice delle adiacenze risulta
peggiore della rappresentazione con liste in termini di memoria
occupata.

Più formalmente, assumendo che il grafo abbia \textit{n} vertici e \textit{m} archi e che, sia i puntatori, sia gli interi, occupino 32 bit.

Si ha che la lista delle adiacenze occupa $32(n+2m)$, mentre la matrice richiede $n^2$.

La matrice risulta quindi vantaggiosa quando:

\begin{align*}
	32(n+2m) &< n^2 \\
	m &< \frac{n(n-32)}{64}
\end{align*}


\subsection{Calcolo del grafo trasposto}\label{calcolo-del-grafo-trasposto}

Dato un grafo orientato \emph{G=(V,E)} si vuole ottenere $ G^T = (V, E^T)$ in modo che gli archi siano rovesciati, ovvero $E^T = \{uv | vu \in E\}$.

Utilizzando la rappresentazione con la matrice delle adiacenze, è
necessario attraversare metà della matrice e mettere a 1 la cella
\emph{i,j} se \emph{j,i} è a 1. La complessità risulta quindi essere
$O(n^2)$.

Con la lista delle adiacenze l'algoritmo risulta essere


\begin{algorithm}
	\begin{algorithmic}
		\Function{Trasponi}{$Adj,\: Adj^T,\: n$}
			\For{$v = 1 \: to \: n$}
				\State $Adj^T[v] \gets nil$
			\EndFor
			\For{$u = 1 \: to \: n$}
				\State {$x \gets Adj[u]$}
				\While{$x \neq nil$}
					\State{$v \gets x.v$}
					\State{$y \gets nodo(u, Adj^T[v])$}
					\State{$Adj^T[v] \gets y$}
				\EndWhile
			\EndFor
		\EndFunction
	\end{algorithmic}
	\caption{Calcolo del grafo trasposto utilizzando la rappresentazione con la lista delle adiacenze}
\end{algorithm}

Ovvero viene attraversata la lista delle adiacenze del grafo originale,
e per ogni elemento delle liste, lo aggiunge ``\emph{al contrario}''
nella nuova lista delle adiacenze.

La complessità risulta quindi essere \emph{O(m+n)}, questo perché il
secondo \texttt{for} esamina tutti i possibili archi, quindi anziché
avere complessità \emph{n} (numero di vertici) ha complessità \emph{m}
(numero di archi).

\subsection{(Esercizio) Ricerca del pozzo
universale}\label{esericizio-ricerca-del-pozzo-universale}

Un vertice è un \textbf{pozzo universale} se può essere raggiunto da
tutti gli altri vertici del grafo, dal quale però non è possibile
raggiungere altri vertici.

Trovare un algoritmo che riesce a risolvere il problema in \emph{O(n)}.

% !TEX encoding = UTF-8
% !TEX TS-program = pdflatex
% !TEX root = data mining.tex
% !TEX spellcheck = it-IT
\chapter{Modello lineare semplice}\label{lezione-3---modello-lineare-semplice}

\begin{itemize}
\item
  \textbf{OLTP}: strumenti di interrogazione su specifiche informazioni
  da rihicedere ai vari database, detti operativi
\item
  \textbf{OLAP}: \ldots{}
\item
  \textbf{KDD}: Knowledge discovery database, si parte da uno o più
  database operativi per costruirne uno strategico, il data whare house;
  questa costruzione comporta anche un'operazione di omogeneizzazione di
  definizione di variabili e operazioni di pulizia dei dati (\textbf{data
  mining analitico}).
\end{itemize}



\textbf{Modello}: (o algoritmo) rappresentazione semplificata del
fenomeno di interesse, funzionale ad un obiettivo specifico.

Non esiste un modello vero in quanto si tratta di approssimazioni molto
dettagliate, ci sono dei modelli che in determinati contesti risultano
migliori di altri. Specialmente in ambiti non scientifici, il criterio
per la bontà di un modello è il \textbf{basta che funzioni}, questo
perché tipicamente i dati che vengono utilizzati non sono stati raccolti
con un criterio sperimentale.

Tipicamente il modello viene visto come una scatola nera, funziona ma
non si sa quale sia il vero meccanismo che regola il fenomeno. Tuttavia,
questa black box deve essere comunque manutenuta, non basta avere
solamente l'hardware e il software.

\section{Il modello lineare semplice}\label{il-modello-lineare-semplice}

Si parte da due variabili e si costruisce un modello che li mette in
relazione tra loro.

Il data set di riferimento riguarda i dati di 200 mercati su cui opera
un'azienda, per i quali si conosce la quantità di merce venduta in
migliaia e il budget speso per la pubblicità radiofonica in quella
zona.

Si vuole ottenere un'equazione che permetta di prevedere le vendite in
funzione del budget.

Il primo passo è quello del costruire un grafico di dispersione, dove
nelle \emph{x} ci sono le spese e nelle \emph{y} ci sono le unità
vendute.

\begin{figure}[htbp]
\centering
\includegraphics{./notes/immagini/l3-figura1.png}
\caption{Grafico di dispersione per il data set delle vendite}
\end{figure}

Dal grafico si può osservare un andamento lineare che può essere
approssimata con:

$$
vendite = \beta_0 + \beta_1 (radio) + (errore)
$$

dove la componente \emph{errore} esprime la parte delle vendite non
legate alle pubblicità via radio.

Un modello di questo tipo prende il nome di \textbf{modello di
regressione lineare semplice}.

La variabile \emph{y} (\emph{vendite}) prende il nome di
\textbf{variabile risposta/dipendente/output} mentre \emph{x}
(\emph{radio}) prende il nome di \textbf{variabile
esplicativa/indipendente/input} e i vari coefficiente $\beta$
prendono il nome di \textbf{parametri}.

Il gioco adesso diventa quello di andare a trovare dei valori
$\hat{\beta}_0$ e $\hat{\beta}_1$ che approssimano la retta nel miglior modo
possibile.

 La ricerca avviene utilizzando i dati presenti nel data set:


\begin{align*}
	y_1 &\approx \hat{\beta}_0 + \hat{\beta}_1 x_1 \\
	y_2 &\approx \hat{\beta}_0 + \hat{\beta}_1 x_2 \\
	&\ldots \\
	y_n &\approx \hat{\beta}_0 + \hat{\beta}_1 x_n
\end{align*}

Raffinando l'idea si ottiene il metodo dei \textbf{minimi quadrati}

$$
s^2(\beta_0, \beta_1) = \sum\limits_{i=1}^n(y_i - \beta_0 - \beta_1 x_i)^2
$$

e si vanno a cercare i parametri che minimizzano l'errore di stima
ai minimi quadrati.

$$ s^2(\hat{\beta}_0,\hat{\beta}_1) \leq s^2(\beta_0, \beta_1) $$

Si utilizza il quadrato della distanza, sia per rendere l'errore
indipendente dal segno, sia per dare maggior peso ad errori maggiori.

Ci sono un po' di barbatrucchi matematici per trovare il minimo quello
che interessa è:

\begin{align*}
	\beta_1 &= \frac{\sum\limits_{i=1}^{n} (x_i - \bar{x})(y_i - \bar{y})}{ \sum\limits_{i=1}^{n} (x_i - \bar{x})^2} \\
	 &= \frac{cov(X,Y)}{var(X)} \\
	 \: \\
	\beta_0 &= \bar{y} - \beta_1 \bar{x}
\end{align*}

Da notare che nel lato pratico non ci sarà mai un dataset con varianza
nulla, perché in quel caso il problema di regressione non ha senso. Per
il dataset delle vendite si ottiene come retta ai minimi quadrati

\begin{figure}[htbp]
\centering
\includegraphics{./notes/immagini/l3-figura5.png}
\caption{Retta ai minimi quadrati per il dataset delle vendite}
\end{figure}

\subsection{Residui}\label{residui}

Non è detto che la retta ai minimi quadrati approssimi bene i dati.

Un indicatore dell'andamento è dato dai \textbf{residui}, che
rappresentano la differenza tra i valori osservati e quelli ottenuti
utilizzando la retta.

$$ r_i = y_i - \hat{\beta}_0 - \hat{\beta}_1x_i \: \: \forall i = 1, \ldots, n$$

Per costruzione della retta, la somma di tutti i residui risulta essere
0:

\begin{align*}
	\sum\limits_{i=1}^{n} r_i &= \sum\limits_{i=1}^{n} (y_i - \hat{\beta}_0 - \hat{\beta}_1x_i)\\
												&= \sum\limits_{i=1}^{n} y_i -n\hat{\beta}_0 - \hat{\beta}_1\sum\limits_{i=1}^{n}x_i\\
												&= n\bar{y} -n(\bar{y}-\hat{\beta}_1\bar{x}) - n \hat{\beta}_1 \bar{x}\\
												&= 0
\end{align*}

Quindi per valutare la retta è necessario utilizzare la
\textbf{varianza} dei residui. Minore è la varianza, migliore è la
retta.

\begin{align*}
	var(r_1, \ldots, r_n) &= \frac{1}{n} \sum\limits_{i=1}^{n} (r_{i} - 0)^2 \\
									  &= var(Y) - \frac{cov^2(X,Y)}{var(X)}
\end{align*}

Nel caso peggiore, la varianza dei residui ha come bound superiore la
varianza della risposta, ovvero \emph{var(Y)}.

Più piccola è la varianza dei residui, più la retta di regressione ``spiega'' le variazioni della risposta.

È quindi possibile definire il \textbf{coefficiente di determinazione}:

$$
R^2 = 1 - \frac{var(r_i, \ldots, r_n)}{var(Y)}
$$

$R^2$ varia tra 0 e 1, dove 1 è il valore ottimo e 0 è il valore
peggiore.

Da notare che ciò vale solo per la retta.

\chapter{Laboratorio}

\section{Un po' di cose su R}\label{un-po-di-cose-su-r}

\begin{itemize}
\item
  Tutti gli oggetti sono vettori
\item
  \texttt{ls()} per vedere le variabili disponibili
\item
  \texttt{x\ \textless{}-\ c(2,3,4,5)} crea un vettore con 1,2,3,4.
\item
  notazione \texttt{{[}1:20{]}} per un vettore con la successione da 1 a
  20
\item
  \texttt{xx\ \textless{}-\ seq(from=100,\ to=1)} crea sempre una
  sequenza di numeri, con parametro opzionale \texttt{by} per
  specificare lo step
\item
  \texttt{rep(2,5)} crea un vettore con 5 elementi uguali a 2
\item
  \texttt{a\ \textless{}-\ c(rep(2,3),4,5,rep(1,5))},
  \texttt{a\ =\ 2\ 2\ 2\ 4\ 5\ 1\ 1\ 1\ 1\ 1}
\item
  \texttt{2*x} esegue il prodotto scalare
\item
  \texttt{length(x)} per la lunghezza del vettore
\item
  \texttt{max(x)} e \texttt{min(x)}
\item
  \texttt{sum(x)} che ritorna un vettore di un solo elemento con la
  somma
\item
  \texttt{mean(x)}, \texttt{var(x)}, \texttt{range(x)}
\item
  \texttt{x{[}7{]}} per estrarre il settimo elemento di \texttt{x},
  l'indice credo parta da 1
\item
  \texttt{x{[}-4{]}} ritorna un vettore senza il quarto elemento
\item
  \texttt{x\ \textless{}-\ matrix(c(2,3,5,7,11,13),nrow\ =\ 3)} crea una
  matrice con gli elementi specificati e 3 righe. Alternativamente è
  possibile specificare anche il numero di colonne.
\item
  \texttt{x2\ \textless{}-\ scan("nome\ file",\ sep="")} con
  \texttt{sep} opzionale, per caricare il contenuto di un file in un
  vettore, per caricare una matrice
  \texttt{x2\ \textless{}-\ matrix(scan(...),\ ncol\ =\ 3,\ byrow=TRUE}.
\item
  \texttt{str(x)} specifica la struttura dell'oggetto
\item
  \texttt{dim(x)} ritorna la dimensione di una matrice, se invocato con
  un vettore ritorna \texttt{NULL}.
\item
  \texttt{x{[}18,{]}} per ottenere la 18-esima riga di una matrice
\item
  \textbf{Dataframe}: matrice le cui colonne possono avere formati
  diversi
\item
  \texttt{ciliegi\ \textless{}-\ read.table("nome\ file")}.
\item
  \texttt{names(ciliegi)} è il vettore con i nomi delle colonne del
  dataframe
\item
  \texttt{names(ciliegi)\ \textless{}-\ c("diametro",\ "altezza",\ "volume")}
  permette di impostare il nome delle colonne, può anche essere
  specificato come parametro opzionale \texttt{col.names} della
  funzione \texttt{read.table}.
\item
  \texttt{summary(ciliegi)} fornisce degli indicatori per ciascuna
  colonna
\item
  \textbf{Mediana}: elemento centrale di una distribuzione ordinata in
  senso crescente, \textbf{primo e terzo quartile}: generalizzazione
  della mediana, rispettivamente l'elemento che sta al 25 e 75 per cento
  della distribuzione. La differenza tra i due quartili da l'idea di
  quanto è variabile la distribuzione.
\item
  I dataframe possono essere acceduti anche con il nome della colonna
  \texttt{ciliegi\$volume}.
\item
  \textbf{attach di un file}: aggiungere al workspace un oggetto, ovvero
  \texttt{attach(ciliegi)} permette di accedere al nome della colonna
  direttamente utilizzando \texttt{volume}. Come complementare c'è il
  comando \texttt{detach}.
\item
  \texttt{hist(diametro)} crea l'istogramma per il diametro
\item
  \texttt{help(hist)} per avere l'help di una funzione
\item
  l'istrogramma che viene generato di default può contenere dei buchi,
  conviene quindi adattare il numero di colonne utilizzando il parametro
  \texttt{breaks}
\item
  \texttt{boxplot(diametro)} fornisce il box plot di un valore, è un
  grafico che rappresenta la mediana, i quartili e il 5 e 95\%. Risulta
  più espressivo dell'istogramma. L'ampiezza della scatola rappresenta
  la variabilità dei dati.
\item
  \texttt{ciliegi{[}altezza\textgreater{}80,{]}} prende tutti i ciliegi
  con altezza maggiore di 80.
\item
  \texttt{library(MASS)} permette di caricare la libreria MASS
\item
  \texttt{search()} permette di visualizzare la lista degli ottetti in
  cui R va a cercare quando deve eseguire un comando
\item
  Gli attributi qualitativi vengono trattati come tipo Factor
\item
  \texttt{table(painters\$School)} crea la tabella con le frequenze
  delle varie qualità
\item
  \texttt{barplot(..)} fa il plot delle barre per una variabile discreta
\item
  \texttt{pie(...)} fa il grafico a torta, anche se è sconsigliabile
  utilizzare un grafico a torta perché per l'occhio umano fa fatica a
  vedere la differenza tra gli angoli.
\item
  come scale colori si possono utilizzare \texttt{heat.colors(k)},
  \texttt{rainbow(k)}, \ldots{}
\item
  \texttt{plot(x,y)} disegna un diagramma di dispersione, il parametro
  \texttt{pch} specifica il tipo di carattere, \texttt{pch=16}
  rappresenta i pallini pieni, \texttt{col} specifica il colore da
  utilizzare, possono indicare \texttt{col=painter\$School} per far
  variare il colore in base al valore dell'attributo quantitativo
\end{itemize}

% !TEX encoding = UTF-8
% !TEX program = pdflatex
% !TEX root = InformationRetrieval.tex
% !TEX spellcheck = it-IT

% 13 Ottobre 2016

%section indicizzazione
%subsection fasi del processo di indicizzazione
%subsubsection Stemming

\paragraph{Dictionary-based stemmer}

Un modo per risolvere questi problemi è quello di basarsi su degli elenchi di parole preparati da degli esperti di linguistica.

Questi dizionari possono essere utilizzati ad esempio per riconoscere che \textit{is, be} e \textit{was} sono tutte forme dello stesso verbo.

C'è però un problema con l'aggiornamento del dizionario, perché nella lingua naturale vengono aggiunte in continuazione nuove parole.
Si può quindi creare il dizionario in modo automatico effettuando un'analisi statistica dei testi del documento, oppure si può scegliere di utilizzare in modo combinato uno stemmer algoritmico e uno basato su dizionario.

\subsubsection{Fase 4: Composizione dei termini}

In alcuni contesti può essere importante ricostruire alcune frasi o termini, come nella ricerca dei documenti scritti da una determinata persona.
Inoltre la maggior parte delle query effettuate dagli utenti sono costituite da varie parole e in alcuni casi questo insieme deve essere considerato come una frase.

Volendo questa fase può essere fatta subito dopo l'analisi lessicale, ma le informazioni che vengono estratte dalle altre fasi può aiutare ad effettuare una composizione migliore.
Infatti si può scegliere di effettuare la composizione delle frasi sia utilizzando le parole originali che con gli stem.

L'analisi e la ricostruzione delle frasi è computazionalmente onerosa, pertanto il problema è stato studiato molto e i risultati che si sono ottenuti sono utilizzabili per raggiungere obiettivi specifici di realizzazione di un IRS.

Il come e quando ricostruire le frasi e le relative conseguenze dipendono molto dal modello che viene utilizzato, pertanto l'argomento verrà trattato più avanti.

\subsubsection{Fase 5: Creazione dell'indice}

Il modo più semplice di costruire l'indice è quello di utilizzare una matrice binaria che ha come righe le parole o gli stem, come colonne i vari documenti e in una determinata cella viene messo un 1 solo se la parola compare nel documento.
Così facendo però non viene tenuto conto della frequenza e del contenuto informativo della parola, inoltre non è possibile trovare un modo per ordinare i risultati della ricerca.

\begin{figure}[ht]
\centering
\begin{minipage}[b]{0.45\linewidth}
		\centering
  \includegraphics[width=0.7\linewidth]{images/l5-index-1}
  \caption{Indice delle parole}

\end{minipage}
\quad
\begin{minipage}[b]{0.45\linewidth}
	\centering
  \includegraphics[width=0.7\linewidth]{images/l5-index-2}
  \caption{Indice degli stem. Ci sono dei casi in cui uno stem compare in tutti i documenti}
\end{minipage}
\end{figure}

Per risolvere questi problemi si posso utilizzare dei pesi per le varie parole, che vengono assegnati con una \textbf{funzione di pesatura} la quale può essere:

\begin{itemize}
	\item \textbf{Binaria}: se tiene conto solamente della presenza o meno della parola.
	\item \textbf{In base alla frequenza}: se considera il numero di occorrenze delle parole nei documenti e nella collezione.
\end{itemize}

\begin{figure}[htbp]
	\centering
	\includegraphics[width=0.4\linewidth]{images/l5-index-3}
	\caption{Indice degli stem che prende in considerazione anche le frequenze}
	\label{fig:l5-index-3}
\end{figure}


La pesatura può assegnare a ciascun termine la sua frequenza di occorrenza in un documento (\textbf{Term Frequency}) oppure la sua frequenza di occorrenza all'interno della collezione (\textbf{Inverse Document Frequency}). Entrambi gli schemi verranno approfonditi più avanti.

Tipicamente nell'indice vengono memorizzati i dati grezzi, ovvero la frequenza assoluta, per calcolare i pesi in fase di reperimento, questo perché se l'indice contenesse direttamente i pesi, all'aggiunta di un nuovo documento sarebbe necessario andare a ricalcolare tutti i pesi di tutti i documenti. Un altro motivo è dato dal fatto che il peso serve solo per le parole che compaiono all'interno della query e quindi il calcolo non è oneroso e può essere fatto online.

L'indice finale dei descrittori è tipicamente una matrice sparsa molto grande e quindi è necessario utilizzare delle apposite strutture dati per memorizzare i dati.
Tipicamente viene utilizzato un \textbf{posting file} (o \textbf{inverted index}) che contiene la lista delle parole e per ogni parola è presente una lista di coppie, ognuna delle quali contiene l'identificativo del documento e la frequenza assoluta della parola all'interno del documento.

\begin{figure}[htbp]
	\centering
	\includegraphics[width=0.3\linewidth]{images/l5-index-4}
	\caption{Esempio di posting file}
\end{figure}

\textbf{{\color{Red} Possibile esercizio:}} Dati dei documenti testuali si effettui la loro indicizzazione automatica creando l'indice dei descrittori. Si descriva una possibile strategia per la costruzione di una stop-list corrispondente ai documenti forniti.

\section{Ranking}

L'approccio base è quello di fornire una lista di documenti ordinata a partire dai dati presenti nell'indice.
Ma non sempre questo va bene:
\begin{itemize}
\item Due utenti possono effettuare la stessa query anche se hanno esigenze informative diverse.
\item Alcuni utenti filtrano la lista, non sempre si concentrano sul primo risultato, ma guardano anche altre cose, come lo snippet fornito nella SERP di Google.
\end{itemize}

Si può quindi pensare di creare un sistema di ranking che si basa su altre risorse, tipicamente sul machine learning, per creare un ordinamento migliore e che tiene conto delle caratteristiche degli utenti.

Quindi l'idea è quella di fare un primo ranking tenendo conto degli indici del sistema e poi ristrutturarlo utilizzando altre tecniche.

% !TEX encoding = UTF-8
% !TEX program = pdflatex
% !TEX root = MEMOC.tex
% !TEX spellcheck = it-IT

\chapter{Meta-euristiche}

Ci sono due modi per risolvere i problemi di ottimizzazione combinatoria:

\begin{itemize}
	\item \textbf{Metodi esatti}: viene creato un modello lineare e, mediante un risolutore come CPLEX, viene prodotta una soluzione ottima in modo esatto.
	\item \textbf{Metodi euristici}: sono metodi che forniscono delle \textit{buone} soluzioni che possono anche essere ottime, ma non riescono a provarne l'ottimalità.  I metodi euristici sono utili perché non sempre è possibile applicare l'approccio esatto.
\end{itemize}

\noindent Tipicamente prima si prova a creare un modello esatto e di risolverlo con un algoritmo efficiente, ma questo non sempre è possibile oppure per risolvere il modello è necessario troppo tempo.
Si può quindi a risolvere il problema con un risolutore generico come CPLEX, ma spesso questi problemi sono NP-hard e quindi al crescere della dimensione, il tempo per trovare una soluzione aumenta in modo esponenziale.
Quando la complessità aumenta troppo entrano in gioco gli approcci euristici.

Ci sono altri casi in cui è preferibile adottare un approccio euristico, ad esempio quando è necessario integrare la risoluzione del modello in un sistema real-time, perché magari l'approccio esatto richiede qualche ora per la risoluzione, ma è necessario avere la risposta in meno di un secondo.

\section{Classificazione dei metodi euristici}

Ci sono due macro categorie di metodi euristici:

\begin{itemize}
	\item \textbf{Euristiche specifiche}: sono euristiche create ad-hoc per un problema, che sfruttano alcune caratteristiche del problema per renderne la risoluzione più efficiente.
	\item \textbf{Euristiche generiche}: sono euristiche che forniscono uno schema di risoluzione generico che si è visto funzionare bene su più problemi.
\end{itemize}

\noindent Le euristiche generiche possono poi essere classificate in:

\begin{itemize}
	\item \textbf{Constructive heuristics}: sono euristiche che possono essere applicate a problemi nei quali è necessario selezionare un sottoinsieme ottimo a partire da un insieme di elementi. In questo caso si parte dall'insieme vuoto e si cerca di aggiungere iterativamente un elemento alla volta, secondo qualche specifico criterio.
	\item \textbf{Meta-euristiche}: sono delle euristiche che possono essere applicate a più categorie di problemi, per le quali è necessario definire alcune regole: come trovare una soluzione, come modificare una soluzione per migliorarla, ecc. Alcune di queste meta-euristiche sono: Local Search, Simulated Annealing e Variable Neighborhood Search.
	\item \textbf{Approximation algorithms}: sono algoritmi che garantiscono di trovare una soluzione con una \textit{percentuale di ottimalità}, ovvero garantiscono che la soluzione che trovano è al massimo peggiore del X\% rispetto la soluzione ottima. Hanno più un interesse teorico.
	\item \textbf{Iper-heuristics}: sono algoritmi in grado di costruire altri algoritmi che riescono a risolvere in modo efficiente uno specifico problema. Ovvero anziché effettuare una ricerca sullo spazio delle soluzione, cercano sullo spazio degli algoritmi in grado di produrre una soluzione.
\end{itemize}

\section{Constructive heuristics}

Queste euristiche cercano una soluzione a partire da quella vuota, andando ad aggiungerci iterativamente degli elementi, cercando di limitare il back-tracking. Il criterio con il quale viene scelto l'elemento da aggiungere prende il nome di \textit{expansion criterion}.

L'euristica più semplice è quella greedy, che ad ogni passo sceglie l'elemento che è migliore in quel momento.

\subsection{Algoritmi greedy}

Adottano un criterio di espansione locale, perché la scelta viene effettuata tenendo conto della scelta migliore per lo stato attuale della soluzione.

Lo schema generico è:

\begin{enumerate}
	\item Inizializza la soluzione \textit{S}.
	\item Per ogni scelta che deve essere fatta:
	\begin{enumerate}
		\item Effettua la scelta migliore per il contesto attuale, tenendo conto dei vincoli del problema.
	\end{enumerate}
\end{enumerate}

\noindent Anche se lo schema è molto semplice, questa tipologia di algoritmi viene molto utilizzata, perché è semplice da implementare, efficiente e segue un approccio intuitivo.
Inoltre, possono essere combinati con altri algoritmi: ovvero viene prima utilizzata una ricerca greedy per trovare una soluzione sub-ottima, per poi utilizzare la ricerca locale per produrre una soluzione migliore.

Una prima modifica a questi algoritmi sta nel introdurre la randomizzazione, ad esempio è possibile scegliere a caso tra le prime mosse migliori, introducendo così la possibilità di effettuare una mossa sub-ottima in modo da ottenere una soluzione diversa.
Così facendo è possibile eseguire più volte l'algoritmo, che tanto è molto efficiente, per poi scegliere la soluzione migliore tra quelle trovate.

L'approccio che cerca una soluzione a partire da una \textit{soluzione vuota} pernde il nome di \textbf{primal heuristics}, mentre le \textbf{dual heuristics} partono da una soluzione unfeasible e cercano di ridurre il livello di \textit{unfeasibility}.


\subsection{Algoritmi che utilizzano metodi esatti}

Questi algoritmi vedono la scelta del miglior elemento da aggiungere come un sotto-problema più facile da risolvere rispetto a quello di partenza.
Ad esempio è possibile rilassare alcuni vincoli per ottenere una soluzione rilassata.

Questa soluzione viene poi utilizzata per aggiungere nuovi vincoli alla versione rilassata del problema, finché non viene trovata una soluzione sufficientemente buona.

Così facendo è necessario più tempo rispetto alla scelta greedy, ma la soluzione che si ottiene è tipicamente migliore.


\subsection{Semplificazione dell'approccio esatto}

Tipicamente gli algoritmi esatti effettuano un'enumerazione delle soluzioni, partendo anch'essi da una soluzione vuota, scegliendo di volta in volta se aggiungere o non aggiungere un elemento (vincolare una variabile) alla soluzione. Viene quindi creato un albero binario di ricerca che deve essere esplorato tutto per avere la garanzia che la soluzione sia ottima.

Un'alternativa per semplificare questo approccio è quella di iniziare l'enumerazione e dopo un tot di iterazioni, iniziare ad effettuare delle scelte greedy per esplorare l'albero in profondità.

Una di queste euristiche prende il nome di \textbf{beam search}: viene prima effettuata una ricerca breath-first parziale, ovvero per ogni nodo aperto vengono creati tutti \textit{b} possibili nodi figli, vengono valutati e sono i \textit{k} migliori vengono espansi.
Così facendo ogni livello dell'albero ha \textit{k} nodi e quindi non c'è l'esplosione combinatoria.

Se l'albero è alto $n$ e ogni nodo ha $b$ figli, si ha che l'albero ha $O(n \cdot k)$ nodi e in tutto vengono valutati $O(n\cdot k \cdot b)$ nodi.
Si ha quindi che il tempo d'esecuzione può essere stimato e regolato in base al parametro $k$.

Nella versione base di beam search non è previsto il backtrack, ma nel \textbf{recovery beam search} viene aggiunta questa possibilità, solo che questo può portare alla perdita della complessità polinomiale.

\section{Local Search}

L'idea alla base di questa meta-euristica è quella di partire da una soluzione del problema e di provare a migliorarla, cercando nel suo vicinato.
Se c'è una soluzione vicina migliore, viene scelta e viene effettuata un'altra iterazione.

Lo schema alla base dell'algoritmo è:

\begin{enumerate}
	\item Determina una soluzione iniziale $x$
	\item Finché $\exists x' \in N(x) : f(x') < f(x) $
	\begin{enumerate}
		\item $x = x'$
	\end{enumerate}
	\item $x$ è la soluzione ottima \textbf{locale}
\end{enumerate}

Per poter applicare questo algoritmo è necessario definire un po' di componenti:

\begin{itemize}
	\item un metodo per trovare la soluzione ottima iniziale;
	\item uno schema di rappresentazione della soluzione;
	\item una funzione che a partire da una soluzione riesca a generare le soluzioni vicine;
	\item una funzione di valutazione per le soluzioni;
	\item una strategia di esplorazione del vicinato.
\end{itemize}







% !TEX encoding = UTF-8
% !TEX program = pdflatex
% !TEX root = AALP.tex
% !TEX spellcheck = it-IT

% 25 Ottobre 2016
%\section{Estensioni del nostro linguaggio}
%\subsection{Unit}

\subsection{Sequenzialità delle operazioni}

Nei linguaggi imperativi c'è la possibilità di definire una sequenza di operazioni da svolgere, tipicamente separando i termini con un punto e virgola.

$$
M_1 ; M_2
$$

\noindent Le regole di riduzione per il nostro linguaggio funzionale sarebbero

\begin{prooftree}
	\AxiomC{$M_1 \rightarrow M_1'$}
	\UnaryInfC{$M_1 ; M_2 \rightarrow M_1' ; M_2$}
\end{prooftree}

\begin{prooftree}
	\AxiomC{$ \: $}
	\UnaryInfC{$v ; M_2 \rightarrow M_2$}
\end{prooftree}

\noindent da notare che il valore calcolato dal primo termine viene scartato, cosa che avviene anche nei linguaggi imperativi, con la differenza che non ci sono side-effect.
Altra cosa importante è che nell'aggiunta di nuovi costrutti servono sempre degli assiomi che specificano come viene calcolato il costrutto e delle regole che ne propagano l'effetto.

Serve inoltre una regola di tipo:

\begin{prooftree}
	\AxiomC{$\Gamma \vdash M_1 : S$}
	\AxiomC{$\Gamma \vdash M_2: T$}
	\BinaryInfC{$\Gamma \vdash M_1 ; M_2 : T$}
\end{prooftree}

\noindent Anche se il valore di $M_1$ viene scartato, è importante che sia comunque ben tipato, perché altrimenti il termine potrebbe diventare un termine stuck anche se $M_2$ è ben tipato.

Lo stesso sistema di sequenzialità può essere replicato utilizzando l'applicazione di una funzione:

$$
(\fn x : T_1 . M_2) \: (M_1) \equiv M_1 ; M_2
$$

\noindent con la condizione che $x \notin fv(M_2)$, perché così facendo, quando viene fatta l'applicazione si ha che $M_2\{x := v\} = M_2$.

Per evitare però di definire tante regole per i possibili valori di $T_1$ conviene imporre $T_1 = \text{ Unit}$ perché tanto il valore calcolato dal termine $M_1$ viene scartato.

\subsection{Coppie di valori}

Vengono aggiunti al nostro linguaggio funzionale:

$$
M ::= (M_1, M_2) \vbar M.\_1 \vbar M.\_2 \vbar \ldots
$$

$$
v ::= (v_1, v_2) \vbar \ldots
$$

$$
T ::= \underbrace{T_1 * T_2}_{\text{tipo coppia}} \vbar \ldots
$$

\noindent Bisogna poi definire una semantica per la valutazione del tipo coppia.

Come prima cosa ci sono gli assiomi per la proiezione.

\begin{prooftree}
	\AxiomC{$ $}
	\UnaryInfC{$(v_1, v_2)\proj 1 \rightarrow v_1$}
\end{prooftree}

\begin{prooftree}
	\AxiomC{$ $}
	\UnaryInfC{$(v_1, v_2)\proj 2 \rightarrow v_2$}
\end{prooftree}

\noindent Ci sono poi le regole per il calcolo dei due valori della coppia:

\begin{prooftree}
	\AxiomC{$M_1 \rightarrow M_1' $}
	\UnaryInfC{$(M_1, M_2)\proj 1 \rightarrow (M_1', M_2)\proj 1$}
\end{prooftree}

\begin{prooftree}
	\AxiomC{$M_2 \rightarrow M_2' $}
	\UnaryInfC{$(v_1, M_2)\proj 1 \rightarrow (v_1, M_2')\proj 1$}
\end{prooftree}

\begin{prooftree}
	\AxiomC{$M_1 \rightarrow M_1' $}
	\UnaryInfC{$(M_1, M_2)\proj 2 \rightarrow (M_1', M_2)\proj 2$}
\end{prooftree}

\begin{prooftree}
	\AxiomC{$M_2 \rightarrow M_2' $}
	\UnaryInfC{$(v_1, M_2)\proj 2 \rightarrow (v_1, M_2')\proj 2$}
\end{prooftree}

\begin{prooftree}
	\AxiomC{$M_1 \rightarrow M_1' $}
	\UnaryInfC{$(M_1, M_2) \rightarrow (M_1', M_2)$}
\end{prooftree}

\begin{prooftree}
	\AxiomC{$M_2 \rightarrow M_2' $}
	\UnaryInfC{$(v_1, M_2) \rightarrow (v_1, M_2')$}
\end{prooftree}

\noindent Mancano ancora delle regole generiche per quando ho un termine più grande:

\begin{prooftree}
	\AxiomC{$M \rightarrow M' $}
	\UnaryInfC{$M\proj 1 \rightarrow M'\proj 1$}
\end{prooftree}

\begin{prooftree}
	\AxiomC{$M \rightarrow M' $}
	\UnaryInfC{$M\proj 2 \rightarrow M'\proj 2$}
\end{prooftree}

\noindent Ovviamente si può fare di meglio \textbf{considerando solamente} le regole:

\begin{prooftree}
	\AxiomC{$M \rightarrow M' $}
	\UnaryInfC{$M\proj 1 \rightarrow M'\proj 1$}
\end{prooftree}

\begin{prooftree}
	\AxiomC{$M \rightarrow M' $}
	\UnaryInfC{$M\proj 2 \rightarrow M'\proj 2$}
\end{prooftree}

\begin{prooftree}
	\AxiomC{$M_1 \rightarrow M_1' $}
	\UnaryInfC{$(M_1, M_2) \rightarrow (M_1', M_2)$}
\end{prooftree}

\begin{prooftree}
	\AxiomC{$M_2 \rightarrow M_2' $}
	\UnaryInfC{$(v, M_2) \rightarrow (v, M_2')$}
\end{prooftree}

Con queste regole vengono prima valutate entrambi i termini e poi viene effettuata la proiezione su uno dei due valori. Volendo si può trovare una semantica alternativa che fa la valutazione lazy, calcolando solo il termine sul quale viene effettuata la proiezione.

Servono infine le regole di tipo:

\begin{prooftree}
	\AxiomC{$ \Gamma \vdash M_1 : T_1$}
	\AxiomC{$ \Gamma \vdash M_2 : T_2$}
	\BinaryInfC{$\Gamma \vdash (M_1, M_2) : T_1 * T_2$}
\end{prooftree}

\begin{prooftree}
	\AxiomC{$\Gamma \vdash M : T*S$}
	\UnaryInfC{$\Gamma M\proj 1 :T$}
\end{prooftree}

\begin{prooftree}
	\AxiomC{$\Gamma \vdash M : S*T$}
	\UnaryInfC{$\Gamma M\proj 2 :T$}
\end{prooftree}

\noindent Con queste regole di tipo è necessario andare a verificare che il teorema di safety regga.

\subsection{Record}

Sono termini del tipo

$$
(M_i\:^{i = 1 \ldots n})
$$

\noindent che ricordano le strutture del C++, dove vengono memorizzati come

$$
\{\text{pippo} =5 , \text{pluto} = \true \} : \{ \text{pippo}: \Nat , \text{pluto} : \Bool \}
$$

\noindent Dato che le funzioni sono valori di primo ordine, il tipo record inizia ad essere una sorta di oggetto.

La sintassi per i record nel nostro linguaggio è la seguente:

$$
M ::= \{l_i = M_i \:^{i = 1 \ldots n}\} \vbar M.l \vbar \ldots
$$

\noindent con $l_i$ viene indicata un'etichetta per il campo dati.

$$
v ::= \{l_i : v_i  \:^{i = 1 \ldots n} \} \vbar \ldots
$$

$$
T ::= \{ l_i : T_i  \:^{i = 1 \ldots n} \} \vbar \ldots
$$

\noindent Ad esempio:

$$
\Bigg(\bigg(\{ \text{a} = 3, \text{b} = \fn x.\fn y. (x+y) \}.\text{b} \bigg) 5 \Bigg) 6
$$

La semantica operazionale risulta essere:

\begin{prooftree}
	\AxiomC{$j \in \{1 \ldots m \} $}
	\UnaryInfC{$\{ l_i = v_i \:^{i = 1 \ldots n} \}.l_j \rightarrow v_j$}
\end{prooftree}

\begin{prooftree}
	\AxiomC{$M \rightarrow M'$}
	\UnaryInfC{$M.l \rightarrow M'.l$}
\end{prooftree}

\begin{prooftree}
	\AxiomC{$ M_j \rightarrow M'_j$}
	\UnaryInfC{$ \{ l_i = v_i \:^{i = 1 \ldots j-1}, l_j = M_j, l_k = M_k \:^{k = j+1\ldots n}\} \rightarrow \{ l_i = v_i \:^{i = 1 \ldots j-1}, l_j = M'_j, l_k = M_k \:^{k = j+1\ldots n}\}$}
\end{prooftree}

\noindent Le regole di tipo sono

\begin{prooftree}
	\AxiomC{$\Gamma \vdash M_i : T_i \: \: \forall \: i = 1\ldots n$}
	\UnaryInfC{$\Gamma \vdash \{ l_i = v_i \:^{i = 1 \ldots n} \} : \{ l_i : T_i \:^{i = 1 \ldots n} \}$}
\end{prooftree}

\begin{prooftree}
	\AxiomC{$ \Gamma \vdash M : \{ l_i : T_i \:^{i = 1 \ldots n}\} \:\: j \in \{1 \ldots n\} $}
	\UnaryInfC{$ \Gamma \vdash M.l_j : T_j$}
\end{prooftree}


\noindent Siamo già vicini al sistema di classi/oggetti in Java, solo che al momento il tipo dell'oggetto dipende dalla struttura del record che lo rappresenta.
Ma ci siamo anche abbastanza lontani, ad esempio ci mancano ancora i metodi.

\subsubsection{L'oggetto conto}

\begin{align*}
\big\{& \\
	\quad & \text{nConto} = 123,  \\
	\quad &\text{saldo} = 1000, \\
	\quad &\text{deposita} = \fn x_{this} : T_O . \fn y : \Nat  x_{this}.\text{saldo} + y  \\
\big\}&
\end{align*}

Ad esempio per chiamare il metodo \text{deposita} la sintassi sarebbe:

\begin{align*}
\quad &\bigg( (obj.\text{deposita}) \: obj \bigg) \: 50  \\
\rightarrow\: & \bigg( (\fn x_{this} : T_O . \fn y : \Nat  x_{this}.\text{saldo} + y )\: obj \bigg)\: 50  \\
\rightarrow\: & \bigg( \fn y : \Nat obj.\text{saldo} + y \bigg)\: 50  \\
\rightarrow\: & obj.\text{saldo} + 50 \\
\rightarrow\: & 1000 + 50 \\
\rightarrow\: & 1050
\end{align*}

\noindent Peccato che non viene aggiornato il saldo, perché non vengono effettuati side-effect.
Serve quindi un modo per creare un nuovo record che deve essere ritornato.

La definizione corretta del ``oggetto'' è quindi:

\begin{align*}
\big\{& \\
\quad & \text{nConto} = 123,  \\
\quad &\text{saldo} = 1000, \\
\quad &\text{deposita} = \fn x_{this} : T_O . \fn y : \Nat. \{ \text{nConto} = x_{this}.\text{nConto}, \text{saldo} = x_{this}.\text{saldo} + y, \text{deposita} = x_{this}.\text{deposita}\}  \\
\big\}&
\end{align*}

\noindent Con questa nuova sintassi, l'esecuzione del metodo diventa

\begin{align*}
\quad &\bigg( (obj.\text{deposita}) \: obj \bigg) \: 50  \\
\rightarrow\: & ... \\
\rightarrow\: & \{ \text{nConto} = 123, \text{saldo} = 1050, \text{deposita} = \fn \ldots \} 
\end{align*}

\noindent Questo modo di definire gli oggetti prende il nome di \textbf{functional objects} ed viene utilizzato da vari linguaggi perché combina i vantaggi della programmazione ad oggetti (riuso) con quelli della programmazione funzionale (immutabilità).

Tuttavia c'è ancora qualcosa da definire, perché al momento per indicare il tipo dell'oggetto stiamo utilizzando $T_O$, che in realtà è un segna posto per:

$$
T_O = \big\{ \text{nConto}: \Nat, \text{salto} : \Nat, \text{deposita}: T_O \rightarrow \Nat \rightarrow T_O \big\}
$$

\noindent che è un'equazione di tipo ricorsivo e non è supportata dal nostro sistema di tipi.

















% !TEX encoding = UTF-8
% !TEX program = pdflatex
% !TEX root = InformationRetrieval.tex
% !TEX spellcheck = it-IT

% 21 Ottobre 2016

%\section{Modello di reperimento}
%\subsection{Modello booleano}
%\subsubsection{Espressione della esigenza informativa}

\noindent Ad esempio con il nostro insieme di documento è possibile fare le seguenti query:

\begin{itemize}
	\item \textit{``pagine OR web''}: vengono forniti in risposta D1 e D3.
	\item \textit{``pagine AND web''}: solo il documento D1.
	\item \textit{``fasi AND web''}: nessun documento.
	\item \textit{``web AND NOT pagine''}: solo il documento D3. 	
\end{itemize}

\noindent Da notare che i documenti non vengono forniti con un ordinamento particolare perché la funzione di reperimento non calcola uno score dei documenti ma associa l'interrogazione al sotto-insieme di documenti che la rendono vera.

\subsubsection{Considerazioni sul modello booleano}

Il modello booleano è molto efficace in ambienti controllati e se l'utente è consapevole di come funziona il sistema e se sa quello che vuole. Magari se l'utente non è esperto può essere aiutato da un intermediario, che può essere un software o una persona.

Bisogna inoltre tenere conto che gli umanisti non conoscono la logica booleana e non sanno distinguere l'and dall'or.

Un altro problema di questo modello è legato alla dimensione dell'output, sulla quale non si ha controllo e può capitare sia di non avere risultati, che di averne troppi. Questo perché non ci sono misure di similarità/pesatura.

Il modello può inoltre essere esteso aggiungendo altri operatori, oltre a quelli booleani:

\begin{itemize}
	\item \textbf{operatori di prossimità}: per permettere la ricerca di frasi o di termini ad una certa distanza.
	\item \textbf{operatori di relazione}: $>, <, =, \leq, \ldots$.
	\item \textbf{operatori di troncamento dei descrittori}: per la ricerca utilizzando le radici dei termini.
\end{itemize}

\noindent Altri operatori possono essere quelli che fornisce Google: \url{https://support.google.com/websearch/answer/2466433?p=adv_operators\&hl=en\&rd=1}.

\subsection{Ordinamento dei risultati in base alla rilevanza}

Se il modello può fornire un ordinamento, quindi se si può effettuare il ranking dei documenti, si ha che i documenti formano una lista in cui a ciascun documento è assegnato un \textbf{rango} (o rank).

Il primo documento della lista ha un \textbf{rango alto} che corrisponde al \textbf{minimo valore intero di posizione nella lista} (ovvero al primo elemento della lista). L'ultimo documento della lista ha un \textbf{rango basso} che corrisponde al \textbf{massimo valore intero di posizione nella lista} (ovvero all'ultimo elemento della lista).

\subsection{Livello di coordinamento}

Il livello di coordinamento è un metodo e tecnica accessoria che può essere utilizzata per ordinare i documenti.
Più precisamente: il livello di coordinamento è una misura di quanto l'interrogazione è vera per un documento.

Il livello di coordinamento più semplice è quello \textbf{binario}, che vale 1 quando il documento in esame rende vera l'interrogazione e 0 in caso contrario.

\begin{figure}[ht]
	\centering
	\begin{minipage}[b]{0.45\linewidth}
		\centering
		\includegraphics[width=0.7\linewidth]{images/l8-set-1}
		\caption{Livello di coordinamento con \textit{AND}}
		
	\end{minipage}
	\quad
	\begin{minipage}[b]{0.45\linewidth}
		\centering
		\includegraphics[width=0.7\linewidth]{images/l8-set-2}
		\caption{Livello di coordinamento con \textit{OR}}
	\end{minipage}
\end{figure}

\noindent Volendo si può uscire dalla logica booleana a due valori, introducendo un livello di coordinamento non binario, andando a considerare il numero di termini della query che sono in presenti nel documento.
Si può quindi utilizzare questo livello per ordinare i documenti forniti in risposta (a volte detto anche \textbf{simple matching}).

\textit{Questo vale solo con l'operatore AND}

Ad esempio si consideri l'interrogazione \textit{``pagine AND web''} per il nostro insieme di documenti.

Si può dire che l'interrogazione è ``più vera'' per D1, perché rende vere entrambe le proposizioni atomiche ``pagine'' e ``web'', che formano l'interrogazione, mentre D3 ne rende vera solo una e D2 nessuna.

\textbf{Piccola nota}: l'AND non è proprio un AND, è una sorta di AND-rilassato, perché un documento viene valutato positivamente anche un documento che non soddisfa tutti i termini.

\begin{figure}[htbp]
	\centering
	\includegraphics[width=0.7\linewidth]{images/l8-cord-gen}
	\caption{Livello di coordinamento non binario.}
\end{figure}

\subsection{Modello vettoriale}
%pacchetto slide 8

Questo modello assume che gli \textit{n} documenti e le interrogazioni appartengano ad uno spazio vettoriale composto da \textit{t} dimensioni, dove \textit{t} è il numero di dei termini indice (parole, frasi, \ldots).

Di conseguenza il documento $D_i$ è rappresentato da un vettore

$$
D_i = (d_{i,1},d_{i,2},\ldots,d_{i,j}, \ldots d_{i,t})
$$

\noindent dove l'elemento $d_{i,j}$ rappresenta il peso del termine $j$-esimo nel documento $i$-esimo.

Anche la query viene considerata come un documento. L'idea alla base di questo è che l'utente vorrebbe trovare un documento uguale alla query che sta scrivendo.

Una collezione di $n$ documenti può quindi essere rappresentata come una matrice dove ogni riga rappresenta un documento e ogni colonna contiene i pesi che sono associati ad un termine in un determinato documento.

\begin{figure}[htbp]
	\centering
	\includegraphics[width=0.5\linewidth]{images/l8-mod-vet}
	\caption{Nell'esempio viene utilizzato la frequenza della parola come peso del termine. Sono state inoltre rimosse le stop-word e viene fatto lo stemming dei plurali \textbf{NB:} la matrice riportata è trasposta rispetto a quella definita precedentemente.}
\end{figure}

\noindent Un'altra cosa carina del modello vettoriale è che in casi semplici può essere rappresentato in modo grafico.

\begin{figure}[htbp]
	\centering
	\includegraphics[width=0.5\linewidth]{images/l8-vet}
\end{figure}

\subsubsection{Metodo di ordinamento}

Data la rappresentazione vettoriale, i documenti possono essere riordinati utilizzando il risultato del calcolo della distanza fra il vettore di ciascuno documento e quello della query.

La distanza viene calcolata con una \textbf{misura di similarità} che ha un valore maggiore nel caso il documento sia simile alla query.
La misura più comune è la \textbf{cosine correlation} che calcola il coseno dell'angolo tra i due vettori.
La cosa interessante di questa misura è che quando i vettori sono normalizzati (hanno lunghezza uguale):

\begin{itemize}
	\item Il coseno dell'angolo $\alpha$ tra due vettori identici vale 1.
	\item Il coseno dell'angolo $\alpha$ tra due vettori che non condividono nessun termine o caratteristica vale 0.
\end{itemize} 

$$
Cosine(D_i, Q) = \frac{\sum\limits_j (d_{i,j} \cdot q_j)}{\sqrt{\sum\limits_j d_{i,j}^2 \cdot \sum\limits_j q_{j}^2}}
$$

\noindent con $j$ che varia da 1 a $t$.

\`E importante precisare che non ci sono ragioni teoriche che fanno preferire questa misura, si è semplicemente visto che funziona bene.

Ad esempio, per i documenti $D_1 = (0.5, 0.8, 0.3)$ e $D_2 = (0.9, 0.4, 0.2)$ e per la query $Q = (1.5, 1, 0)$ si ottengono le similarità:

\begin{align*}
	Cosine(D_1,Q) &= 0.87 \\
	Cosine(D_2, Q) &= 0.97
\end{align*}

\subsubsection{Pesatura dei termini}

Rimane comunque da definire degli schemi di pesatura per i vari termini.

La maggior parte degli schemi si basano sul \textbf{TF-IDF}, dove \textbf{TF} è la frequenza normalizzata del termine all'interno del documento:

$$
tf_{i,k} = \frac{f_{i,k}}{\sum\limits_j f_{i,j}}
$$

\noindent dove $k$ è il termine d'interesse, $f_{i,j}$ è la frequenza del termine $j$ nel documento $i$.

Tuttavia, così facendo viene dato molto peso ai termini molto frequenti nel documento e si è visto che a livello pratico utilizzare il logaritmo produce risultati migliori.

$$
tf_{i,k} = \frac{\log (f_{i,k}+1}{\sum\limits_j \log (f_{i,j} +1)}
$$

% !TEX encoding = UTF-8
% !TEX program = pdflatex
% !TEX root = MEMOC.tex
% !TEX spellcheck = it-IT

\chapter{Il metodo del simplesso e la dualità}

Inizialmente abbiamo visto come le soluzioni di un problema di programmazione lineare intera si trovino sui vertici della regione ammissibile e come fosse possibile trovare una soluzione in modo grafico.

Con il metodo del simplesso vengono fatte delle considerazioni simili, però a livello algebrico, in modo che possano essere generalizzate a casi che utilizzano più di due variabili.

\section{Le basi del simplesso}

La struttura generale di un modello è:

\begin{align*}
	\min(\max)\: &f(x) \\
		\st &g_i(x) = b_i \\
			&g_i(x) \leq b_i \\
			&g_i(x) \geq b_i \\
			&x \in \mathbb{R}^n
\end{align*}

\noindent dove $x$ è un vettore colonna di $n$ variabili reali, $f$ è la funzione obiettivo, $g_i$ sono le funzioni che rappresentano i vincoli (sono tutte funzioni $\mathbb{R}^n \to \mathbb{R}$) e i vari $b_i$ sono valori reali.

La cosa chiave è che i valori \textbf{sono numeri reali} e i vincoli sono \textbf{funzioni lineari}.

Si ha quindi che una \textbf{soluzione ammissibile} $x \in \mathbb{R}^n$ è un vettore che soddisfa tutti i vincoli, mentre la \textbf{regione ammissibile} è data dall'insieme delle soluzioni ammissibili e la soluzione ottima è una particolare soluzione ammissibile che minimizza (o massimizza) la funzione obiettivo.

Il processo di risoluzione consiste quindi nel determinare se:

\begin{itemize}
	\item Il problema è non ammissibile.
	\item Il problema è illimitato.
	\item Il problema ha una soluzione ottima.
\end{itemize}

\noindent La regione ammissibile può essere rappresentata come un \textbf{poliedro}, un'intersezione di un numero finito di semi-spazi chiusi e iperpiani in $\mathbb{R}^n$.

Un problema può quindi essere modellato come:

$$
\min (\max) \{c^T x : x \in P\}
$$

\noindent dove $P$ è un poliedro in $\Rn$.

\begin{figure}[htbp]
	\centering
	\includegraphics[width=0.6\textwidth]{images/l9-fig-1.png}
\end{figure}

\noindent Un \textbf{vertice} $v \in P$ è un vertice del poliedro $P$ se non è una \textbf{strict convex combination} di due punti distinti di $P$.

Un vertice $z \in \Rn$ è una \textbf{combinazione convessa} di due punti quando

$$
\exists \lambda \in [0,1] : z = \lambda x + (1-\lambda)y
$$

\noindent Viene detta \textbf{strict} se i valori 0 e 1 sono esclusi dall'intervallo.

Una \textbf{combinazione convessa} può essere fatta anche con più di due punti e può essere utilizzata per rappresentare una regione.
$z \in \Rn$ è una combinazione convessa di $x^1, x^2, \ldots, x^k$ se $\exists \lambda_1, \lambda_2, \ldots \lambda_k \geq 0$ tale che 

$$
\sum\limits_{i=1}^{k} \lambda_i = 1 \quad \text{e}\quad z = \sum\limits_{i=1}^{k} \lambda_ix^i
$$

\noindent Per il teorema di \textbf{Minkowski-Weyl}, la combinazione convessa di tutti i vertici di un poliedro permettono di rappresentare tutti i punti che appartengo al poliedro.

Quindi, per il \textbf{teorema del vertice ottimo}, se un problema LP ha può essere rappresentato da un poliedro $P$, allora esiste almeno una soluzione ottima e una di queste è su un vertice.

Questo è un risultato importante perché possiamo limitare la ricerca della soluzione ottima sui vertici di $P$ e non su tutto lo spazio.

\subsection{Rappresentazione algebrica dei vertici}

Considerando tutti i vincoli come delle uguaglianze abbiamo che i vertici del poliedro sono ottenuti intersecando tra loro le rette rappresentate dai vincoli (considerando il caso con due variabili).

Per trasformare i vincoli da disuguaglianze in uguaglianze è necessario aggiungere delle variabili di \textit{slack} che simulano il maggiore o minore.

Ad esempio:

	\begin{align*}
	3x_1 + 4x_2 &\leq 24 \\
	x_1 + 4x_2 &\leq 20 \\
	3x_1 + 2x_2 &\leq 18
	\end{align*}


\noindent diventa



	\begin{align*}
	3x_1 + 4x_2 +s_1&= 24 \\
	x_1 + 4x_2 +s_2&= 20 \\
	3x_1 + 2x_2 +s_3&= 18
	\end{align*}



Se si crea un sistema con queste nuove equazioni si ottengono due gradi di libertà, ovvero possiamo porre due variabili qualsiasi a zero per ottenere una soluzione unica.
Ad esempio se $s_1 = s_2=0$, otteniamo un sistema che può essere risolto e la cui soluzione rappresenta il vertice nel quale vengono imposti saturi il primo e secondo vincolo.

\subsection{Forma standard di un problema}

Per rendere l'approccio generico assumiamo che tutti i problemi sono scritti secondo la forma standard:

\begin{align*}
\min \: &c_1x_1+ c_2x_2 +\ldots + c_nx_n \\
\st &a_{i1}x_1 + a_{i2}x_2 +\ldots+a_{in}x_n = b_i \quad (i = 1 \ldots m) \\
&x_i \in \mathbb{R}_+ \quad (i = 1 \ldots n)
\end{align*}

\noindent Ovvero un problema in quale si minimizza sempre la funzione obiettivo, con tutte le variabili $\geq 0$ (per rendere semplice identificare le soluzioni non ammissibili), con tutti i vincoli che sono uguaglianze e con tutte le parti destre $b_i$ dei vincoli $\geq 0$.

Si può sempre trasformare un problema generico in forma standard.

\begin{itemize}
	\item Se il problema iniziale è una massimizzazione, basta passare alla minimizzazione e moltiplicare la funzione obiettivo per $-1$.
	\item Se ci sono delle variabili negative è possibile sostituirle con un'altra variabile che è positiva, es $x_+ = - x_\text{-}$.
	\item Se ci sono delle disuguaglianze è possibile aggiungere le variabili slack.
	\item Se un $b_i$ è negativo si può moltiplicare tutto il vincolo per $-1$. 
\end{itemize}

\section{Il metodo del simplesso}

\noindent Assumiamo che il problema sia modellato dal sistema $Ax = b$, $A \in \R^{m\times n}, \rho(A) = m, m <n$.
Definiamo come \textbf{base di $A$} una sotto-matrice quadrata $B \in \R^{m \times m }$ tale che abbia rango massimo, ottenuta prendendo $n$ colonne linearmente indipendenti della matrice $A$. 

Abbiamo quindi che $A = [B|N]$ con $det(B) \neq 0$ e $x = \begin{bmatrix}
x_B\\ 
x_F
\end{bmatrix}$ con $x_B \in \R^m$ e $x_F \in \R^{n-m}$.

Così facendo possiamo riscrivere il sistema come 

\begin{align}
Ax &= b \\
[B|F]\begin{bmatrix}
x_B\\ 
x_F
\end{bmatrix} &= b\\
B x_B + F x_F &= b
\end{align}

\noindent Posso quindi ricavare i valori di $x_B$, ovvero delle variabili presenti nella base con

$$
x_B = B^{-1}b - B^{-1}F x_F
$$

\noindent Ponendo le variabili $x$ fuori base ($x_F$) uguali a 0, otteniamo una \textbf{soluzione base}.
Il nome \textbf{base} deriva dal fatto che l'insieme dei vettori che compongono la base sono tutti linearmente indipendenti.

Quindi in una soluzione base abbiamo almeno $n-m$ variabili uguali a 0 (se ce ne sono di più la base diventa degenere).

Tornando al nostro problema di partenza abbiamo che una soluzione base diventa \textbf{ammissibile} quando soddisfa tutti i vincoli di partenza, ovvero

$$
x_B = B^{-1}b \geq 0
$$

\noindent Inoltre, per la proprietà di \textbf{corrispondenza tra vertici e soluzioni di base} si ha che se $x$ è una soluzione di base, allora
\begin{equation}
Ax = b \Leftrightarrow x \text{ è vertice di }P 
\end{equation} 
 
Per ottenere delle basi diverse, e quindi cercare un altro vertice, basta cambiare quali variabili sono fissate a 0.

Combinando ciò con il teorema del vertice ottimo si ottiene il \textbf{Teorema fondamentale delle programmazione lineare}, il quale asserisce che se esiste una soluzione ottima allora esiste anche una soluzione di base ammissibile e ottima.

Da qui nasce l'idea del metodo del simplesso, il quale parte da un problema posto in forma standard e da una soluzione ammissibile per trovare un vertice ottimo.

\subsection{I passi dell'algoritmo}

L'algoritmo del simplesso considera un problema in forma standard e necessita di una soluzione ammissibile di base di partenza.
L'algoritmo passa iterativamente da una soluzione ammissibile di base ad una adiacente che permetta di migliorare il valore corrente della funzione obiettivo fino al raggiungimento dell'ottimo.

\subsection{Passo -1: Passaggio alla forma standard}

Trasformo il problema nella forma

\begin{align*}
\min \:& z=c^Tx \\
\st& Ax=b \\
&x\geq 0
\end{align*}

\subsection{Passo 0: Base iniziale e tableau}

Fisso una base iniziale di partenza e riscrivono il problema in modo che:

\begin{itemize}
\item Nei vincoli, le variabili di base siano espresse solamente nei termini delle variabili fuori base.
\item La funzione obiettivo contenga solo variabili non di base
\end{itemize}

\noindent ovvero il problema diventa:

\begin{align*}
\min \: z=\: &c_{B}^T B^{-1}b + (c_{F}^T - c_{B}^T B^{-1}F)x_F \\
\st& Ix_B + B^{-1}Fx_F = B^{-1}b \\
&x\geq 0
\end{align*}

\noindent che può essere sintetizzato in 

\begin{align*}
\min \: z=\: &z_B + \overline{c}_{F}^Tx_F \\
\st& Ix_B + \overline{F}x_F = \overline{b} \\
&x\geq 0
\end{align*}

\noindent dove:

\begin{itemize}
\item $ \overline{b} =  B^{-1}b $ rappresenta il valore delle variabili di base nella soluzione corrente.
\item $ z_B =  c_{B}^T B^{-1}b$ è il valore corrente della funzione obiettivo.
\item $ \overline{F} = B^{-1}F $ sono le colonne delle variabili fuori base espresse nei termini della base corrente.
\item $ \overline{c}_{F}^T = c_{F}^T - c_{B}^T B^{-1}F$ sono i costi ridotti per le variabili fuori base.
\end{itemize}

Questa forma prende il nome di \textbf{forma tableau} o forma canonica.

\begin{table}[htbp]
\centering
\begin{tabular}{c|c|c|c|}
\cline{2-4}
$-z$ & $-z_B$           & $0$     & $\overline{c}_{F}^T$ \\ \cline{2-4} 
   & $\overline{b}$ & $Ix_B$ & $\overline{F}x_F$         \\ \cline{2-4} 
\end{tabular}
\end{table}

\begin{figure}[htbp]
	\centering
	\includegraphics[width = .7\textwidth]{images/l9-fig-3.png}
	\caption{Il talbeau del simplesso}
\end{figure}

\subsection{Passo 1: Scelta della variabile da fare entrare in base}

L'operazione di cambio base permette di passare da una soluzione ammissibile di base ad una adiacente.

Considerando l'espressione della funzione obiettivo in termini di variabili fuori base, ovvero considerando la prima riga del tableau, abbiamo che:

\begin{align*}
z &= z_B + \overline{C}_{F}^Tx_F \\
&= z_B + \overline{C}_{F_1}^Tx_{F_1} + \ldots + \overline{C}_{F_n}^Tx_{F_n}
\end{align*}

Se entra in base una nuova variabile $x_{F_j}$, ovvero diventa diversa da 0, si ha che il nuovo valore della funzione obiettivo diventerà

$$
z = z_B + \overline{C}_{F_j}^Tx_{F_j}
$$

dove $\overline{C}_{F_j}$ rappresenta l'incremento marginale della funzione obiettivo all'aumentare di $x_{F_j}$ e viene detto costo ridotto. 

\subsection{Passo 2: Scelta della variabile uscente}

La nuova variabile che entra in base deve assumere il massimo valore possibile che non rende inammissibile la soluzione.
Pertanto viene assegnato alla variabile entrante $x_j$ il valore $\vartheta$ che preserva l'ammissibilità della soluzione, ovvero non rende le altre variabili negative.

Il valore $\vartheta$ viene scelto con la \textbf{regola del quoziente}

$$
\vartheta = \min\limits_{i | \overline{a}_{ij} > 0} \bigg\{ \frac{\overline{b}_i}{\overline{a}_{ij}} \bigg\}
$$

dove $\overline{a}_{ij}$ è il coefficiente di $x_j$ nel vincolo $i$.

La variabile uscente è quindi quella che corrisponde al minimo quoziente, ovvero a quella che viene azzerata ponendo $x_j = \vartheta$.

Da notare che se tutti gli $\overline{a}_{ij}$ sono $\leq 0$, il problema è illimitato, perché la variabile $x_j$ può entrare in base con un valore qualsiasi, migliorando sempre la funzione obiettivo.

\subsection{Passo 3: Cambio di base} 

Se $x_j$ è la variabile selezionata per entrare in base e $x_k$ è la variabile uscente è necessario effettuare un'operazione di pivot della matrice in modo che la colonna corrispondente a $x_j$ diventi della forma $ \begin{bmatrix}
0\\ 
e_k
\end{bmatrix} $ dove $e_j$ è un vettore di 0 con l'elemento di indice $j = 1$.

\subsection{Passo 4: Condizione di terminazione}

Il metodo del simplesso continua fino a che non ci sono più costi ridotti negativi e pertanto la soluzione di base è anche ottima.

Bisogna però stare attenti a non visitare sempre la stessa sequenza di vertici. Per evitare ciò è possibile scegliere di far entrare in base sempre la variabile di indice minimo tra tutte quelle che hanno costi ridotti negativi.



\subsection{Algoritmo del simplesso nel dettaglio}

\begin{enumerate}
	\item Trasforma il PL in forma standard e trova una base iniziale feasible $B$. La base può essere trovata risolvendo un problema artificiale.
	\item Ripeti finché non trovi una soluzione ottima o non ti accorgi che il problema è illimitato:
	\begin{itemize}
		\item Poni il problema in forma canonica rispetto la base $B$
		\begin{align*}
		z &= \overline{z}_B + \overline{c}_{F_1} x_{F_1} 
		+ \overline{c}_{F_2} x_{F_2} + \ldots + \overline{c}_{F_{n-m}} x_{F_{n-m}} \\
		x_{B_i} &= \overline{b}_i - \overline{a}_{iF_1} x_{F_1} - \overline{a}_{iF_2} x_{F_2} - \ldots - \overline{a}_{iF_{n-m}} x_{F_{n-m}} \qquad \forall \: i \in [1 \ldots m]
		\end{align*}
		\item Se $\overleftarrow{c}_j \geq 0 \forall \ j$, la base $B$ è ottima. Fermati.
		\item Se $\exists h : \overline{c}_h < 0 , \overline{a}_{ih}\leq 0 \ \forall i$, il problema è illimitato. Fermati.
		\item Scegli una variabile $x_h$ da portare in base tra tutte quelle con costi ridotti negativi.
		\item Trova la variabile $x_{B_t}$ con $ t = \arg \min_{i = 1 \ldots m} \bigg\{ \frac{\overline{b}_i}{\overline{a}_{ih}}  : \overline{a}_{ih} > 0 \bigg\}$
		\item $B \leftarrow B \oplus A_h \ominus A_{B_t}$, ovvero effettua il cambio di base.
	\end{itemize}
\end{enumerate}

Da notare che tutti i vari passaggi possono essere replicati utilizzando i calcoli matriciali al posto del tableau del simplesso.


\subsection{Metodo delle due fasi}

Questo metodo permette di trovare una soluzione ammissibile di partenza per il metodo del simplesso.

Dato un problema in forma standard

\begin{align*}
\min \:& z=c^Tx \\
\st& Ax=b \\
&x\geq 0
\end{align*}

\noindent viene creato un problema artificiale, aggiungendo al problema di partenza un nuovo vettore di variabili $y \in \R^{m}$ e cambiando la funzione obiettivo:

\begin{align*}
w^* = \min \:& w= I^Ty = y_1 + y_2 + \ldots + y_m \\
\st& Ax + Iy=b \\
&x, y\geq 0
\end{align*}

Per come è formulato questo problema ho già che $Iy$ è una base ammissibile ma non ancora espressa in forma canonica.
Inoltre, questo problema è sempre ammissibile, perché la soluzione $x = 0, y=b$ è ammissibile ed inoltre non è mai illimitato perché la funzione obiettivo è la minimizzazione di una sommatoria con termini tutti maggiori di 0, pertanto l'ottimo è 0.

Una volta messo in forma canonica il problema artificiale, posso risolverlo utilizzando il simplesso ottenendo un valore ottimo $w*$, che può essere:

\begin{itemize}
\item $w^* > 0$: il problema originale non è ammissibile e quindi non ha senso passare alla fase 2.
\item $w^* = 0$: in questo caso tutte le variabili artificiali sono necessariamente nulle e quindi possono essere tolte dai vincoli, rendendoli comunque soddisfatti, pertanto il problema di partenza è ammissibile. Se tutte le $y$ sono fuori base, allora il tableau finale può essere usato come tableau di partenza per il simplesso sul problema originale, altrimenti è necessario effettuare un cambio di base in modo da togliere dalla base tutte le $y$.


\end{itemize}

















% !TEX encoding = UTF-8
% !TEX program = pdflatex
% !TEX root = MEMOC.tex
% !TEX spellcheck = it-IT

\section{Definizione del problema duale}

Dato un problema di programmazione lineare in forma di standard, si vuole fornire una limitazione inferiore ai possibili valori che la funzione obiettivo può assumer nella regione di ammissibilità.

\begin{align*}
	(PL) \quad z* = &\min x = c^T x \\
	                &\st Ax = b \\
	                &\quad x \geq 0
\end{align*}

dove $A \in \Rmn$, $c \in \Rn$, $b \in \Rm$ e $x$ è un vettore di variabili in $\Rn$.

Un limite inferiore per un problema in forma standard è un valore $l \in \R$ che è minore o uguale del valore di tutte le soluzioni ammissibili per $PL$.
Più formalmente:

$$
l \leq c^T x \quad \forall x \text{ ammissibile}
$$

Per ottenere un lower bound  si può partire da un vettore $u \in \Rm$ e imporre la condizione di lower bound a partire dall'equazione $Ax = b$.

$$
u^TAx= u^Tb \quad\forall x \text{ ammissibile}
$$

Perché $u$ rappresenti un lower bound è necessario che $u^TA \leq c^T $ e questo deriva dal fatto che il valore della funzione obiettivo $c^Tx$ deve essere maggiore del lower bound:

$$
u^T A x \leq c^T x\quad\forall x \text{ ammissibile, inoltre } u^T A x = u^T b 
$$

Si noti che $x\geq0$ e quindi la disuguaglianza continua a valere.
Inoltre, una volta trovato un lower bound $l$ si ha che se c'è una soluzione ammissibile $\tilde{x}$ tale che $c^T\tilde{x} = l$, si ha che $\tilde{x}$ è anche una soluzione ottima.
Risulta quindi importante avere il lower bound più alto possibile e questo lo si fa scegliendo in modo opportuno il vettore $u$.

La scelta di questo vettore può essere vista come un problema di massimizzazione, in cui le variabili decisionali sono contenute nel vettore $u$:

\begin{align*}
		(DL) \quad \omega* = &\max \omega = u^T b \\
						&\st u^T A \leq c^T \\
						&\quad u \text{ libero}
\end{align*}

Questo nuovo problema prende il nome di \textbf{problema duale} ed è strettamente legato alla forma del problema primale.
Si può infatti notare che viene introdotta:

\begin{itemize}
	\item Una variabile duale in corrispondenza di ogni vincolo primale.
	\item Un vincolo duale in corrispondenza di ogni variabile primale.
\end{itemize}

Un esempio di coppia primale-duale è:


\begin{align*}
	(PL) \min\:& 2x_1 - 3x_2 +x_3 \\
	\st  & x_1 - x_2 = 6 \\
	 	 & 4x_1 +2 x_3 = 4 \\
	     & x_1, x_2, x_3 \geq 0 \\
	     &\\
	(DL) \max\:&6u_1 + 4u_2 \\
	     \st  & x_1 - x_2 = 6 \\
	     & u_1 + 4u_2 \leq 2 \\
	     & -u_1 \leq -3 \\
	     & 2u_2 \leq 1 \\
	     & u_1, u_2 \in \R
\end{align*}

\section{Teoremi della dualità}

\subsection{Dualità debole}

Data una soluzione $\tilde{x}$ ammissibile per il $PL$ e una soluzione $\tilde{u}$ ammissibile per il $DL$, allora si ha, per costruzione del problema duale, che:

$$
\underbrace{c^T \tilde{x}}_{\text{f.o. in }\tilde{x}} \geq \underbrace{\tilde{u}^T b}_{\text{lb associato a }\tilde{u}}
$$

Pertanto segue che se si ha un'uguaglianza, $\tilde{x}$ è una soluzione ottima per $PL$ e $\tilde{u}$ è una soluzione ottima per $DL$, perché $\tilde{x}$ è ammissibile ed è uguale al lower bound del problema e quindi è ottima e, in modo simile, $\tilde{u}$ è una soluzione ottima per $DL$ perché è uguale ad un upper bound del problema.
Infatti, con un ragionamento simile a quello fatto per ottenere il problema duale a partire dal primale, è possibile ottere quello primale a partire dal duale, ragionando sugli upper bound.

Vale inoltre che:

\begin{itemize}
	\item Se $PL$ è illimitato, non è possibile trovare un lower bound e quindi $DL$ è inammissibile.
	\item Se $DL$ è illimitato, non è possibile trovare un upper bound e quindi $PL$ è inammissibile.
	\item Il fatto che un problema sia inammissibile non implica che l'altro sia illimitato, infatti possono essere entrambi inammissibili.
\end{itemize}

\subsection{Dualità forte}

\begin{center}
	$PL$ ammette soluzione ottima $x^* \Leftrightarrow DL$ ammette soluzione ottima $u^*$ e $c^Tx^* = u^{*T}b$ 
\end{center}

Il verso $PL\Rightarrow DL$ si dimostra basandosi sulla teoria del simplesso.

Se $PL$ ha una ammette soluzione ottima, allora ne avrà una di base che può essere ricavata con il metodo del simplesso ($x^*$) a partire dalla quale si può costruire un vettore $u \in \Rm$ che è soluzione ammissibile e ottima per il duale.

$$
x^* = \begin{bmatrix}
x_{B}^* \\
x_{F}^*
\end{bmatrix}
$$

Essendo $x^*$ una soluzione trovata con il simplesso, i costi ridotti delle variabili in base saranno nulli ($\overline{c}_B = 0$) e quelli delle variabili fuori base saranno maggiori o uguali di 0 ($\overline{c}_F \geq 0$).

Poniamo

$$
u^T = c_{B}^TB^{-1}
$$

Per la definizione di costo ridotto si ha che

\begin{align*}
\overline{c}_{F}^T = &c_F - c_{B}^TB^{-1}F & \\
				   = &c_F - u^T F \geq 0 \\
					  &c_F \geq u^T F	                
\end{align*}

Possiamo quindi scrivere:

\begin{align*}
	u^T A &= u^T[B|F] \\
	      &= [u^TB |u^T F] \\
	      &= [c_{B}^TB^{-1} B |u^T F] \\
	      &= [c_{B}^T |u^T F] \leq [c_{B}^T | c_{F}^T] = c^T \\
    u^T A &\leq c^T
\end{align*}

Ovvero la soluzione $u^T$ è ammissibile per il problema duale e il suo corrispondente valore della funzione obiettivo è:

\begin{align*}
	u^T b &= c_{B}^T B^{-1} b \\
	      &= c_{B}^T x_{B}^* \\
	      &= c^Tx*
\end{align*}

I valori delle due funzioni obiettivo sono uguali e quindi per il corollario del teorema della dualità debole, le due soluzioni sono entrambe ottime.

\section{Trasformazione da Primale a Duale}

Per effettuare il passaggio dal problema primale a quello duale non è necessario che il problema primale sia in forma standard, l'unica cosa importante è che sia rispettata la catena di maggiorazioni:

$$
c^T x \geq u^T Ax \geq u^Tb
$$

\noindent Per semplificarci la vita, la trasformazione può essere fatta utilizzando la seconda tabella. Da notare che la tabella si legge da sinistra a destra se si ha un problema primale in forma di minimo e da destra a sinistra se si ha un problema primale in forma di massimo.

\begin{table}[htbp]
	\centering
	\begin{tabular}{|l|l|}
		\hline
		\multicolumn{1}{|r|}{\textbf{Primale} $\min c^T x$} & \textbf{Duale} $\max u^Tb$ \\ \hline
		$a_{i}^T x \geq b_i$    & $u_i \geq 0$    \\ \hline
		$a_{i}^T x \leq b_i$    & $u_i \leq 0$     \\ \hline
		$a_{i}^T x = b_i$       & $u_i \text{ libera}$     \\ \hline
		$x_j \geq 0$            & $u^TA_j \leq c_j$     \\ \hline
		$x_j \leq 0$            & $u^TA_j \geq c_j$       \\ \hline
		$x_j \text{ libera}$    & $u^TA_j = c_j$       \\ \hline
	\end{tabular}
\end{table}


\noindent Ad esempio:

\begin{align*}
(PL)\max\:& 3x_1 +2x_2 -2x_3 \\
     \st  & 4x_1 -6x_2 +x_3 = 1 \quad (u_1)\\
          & 2x_1 +2x_2 +3x_3 \leq 3 \quad (u_2) \\
          & x_1 \geq 0, x_2 \leq 0, x_3 \text{ libera} \\
          & \\
(DL)\min\:& 1u_1 + 3u_2 \\
	 \st  & 4u_1 +2u_2 \geq 3 \quad (x_1)\\
		  & -6u_1 +2u_2 \leq 2 \quad (x_2)\\
		  & 1 u_1 +3 u_2 = -2 \quad (x_3)\\
		  & u_1 \text{ libera}, u_2 \geq 0
\end{align*} 

\section{Condizioni di complementarietà primale-duale}

Il teorema della dualità forte fornisce delle \textbf{condizioni di ottimalità}: $x^*$ e $u^*$ sono soluzioni ottime per la coppia di problemi se e solo se ($\Leftrightarrow$)

\begin{itemize}
	\item $x^*$ è ammissibile primale, ovvero $Ax^* \geq b \wedge x^* \geq 0$
	\item $u^*$ è ammissibile duale, ovvero $u^{*T}A \leq c^T \wedge u^* \geq 0$
	\item vale la dualità forte, ovvero $c^Tx^* = u^{*T}b$
\end{itemize}

Si può quindi pensare di applicare direttamente le condizioni di ottimalità primale-duale, impostando un sistema di equazioni lineari contenete i vincoli del primale (uguaglianze), i vincoli del duale (sempre le uguaglianze) e aggiungere come ultimo vincolo l'uguaglianza delle funzioni obiettivo.

Più formalmente si possono riscrivere le condizioni di ottimalità come:

\begin{align*}
	\text{(ammisibilità primale)}\quad&Ax^* \geq b \wedge x^* \geq 0 \\
	\text{(ammisibilità duale)}\quad&u^{*T}A \leq c^T \wedge u^* \geq 0 \\
	\text{(ortogonalità)}\quad&\begin{cases}
	u^T(Ax -b) = 0 \\
	(c^T - u^T A) x = 0
	\end{cases}
\end{align*}

dove è possibile espandere i termini

\begin{align*}
	u^T(Ax -b) &= \sum\limits_{i=1}^{m} u_i (a_{i}^Tx -b_i) = 0\\
	(c^T - u^T A) x &=\sum\limits_{j=1}^{n} (c_j - u^TA-J)x_j = 0
\end{align*}

Tenendo presente che per l'ammissibilità dei problemi, tutti i fattori delle sommatorie devono essere $\geq 0$, si ha che all'ottimo vale:

\begin{align*}
u_i (a_{i}^Tx -b_i) &= 0 \quad \forall i = 1 \ldots m\\
(c_j - u^TA-J)x_j   &= 0\quad \forall j = 1 \ldots n
\end{align*}

Queste condizioni sono rispettate per ciascun vincolo/variabile primale/duale.
Ovvero, due soluzioni $x$ e $u$ sono ottime se e solo se:

\begin{enumerate}
	\item Ogni variabile primale positiva $x_j > 0$ implica il vincolo duale saturo $u^TA_j = c_j$
	\item Ogni vincolo duale lasco $u^TA_j < c_j$ implica la variabile primale nulla $x_j = 0$
	\item Ogni variabile duale positiva $u_i > 0$ implica il vincolo primale saturo $a_{i}^Tx = b_i$
	\item Ogni vincolo primale lasco $a_{i}^Tx > b_i$ implica la variabile duale nulla $u_i = 0$
\end{enumerate}

Grazie a queste condizioni possiamo andare a verificare che una soluzione data sia ottima o meno, provando a costruire una soluzione duale che è complementare a quella primale data.

\subsection{Esempio}

Dato il problema

\begin{align*}
    \min\:& 2x_1 + 3x_2 \\
     \st  & 3x_1 + x_2 \geq 11 \quad (u_1)\\
          & x_2 \geq 2 \quad (u_2) \\
          & x_1 \geq 1 \quad (u_3) \\
          & x_1, x_2 \geq 0
\end{align*}

vogliamo verificare se la soluzione $\overline{x} = (3,2)$ è ottima, applicando le condizioni di complementarietà.

Come prima cosa è necessario provare se la soluzione è ammissibile, andando a sostituire all'interno dei vincoli i valori della soluzione. Se tutti i vincoli sono rispettati la soluzione è ammissibile. Lo è.

Perché la soluzione sia anche ottima è quindi necessario trovare una soluzione complementare per il problema duale.

\begin{align*}
\max\:& 11u_1 + 2u_2 + u_3 \\
 \st  & 3u_1 + u_3 \leq 2 \\
      & u_1 + u_2 \leq 3\\
      & u_1, u_2, u_3 \geq 0
\end{align*}

Dobbiamo quindi trovare delle equazioni da mettere in sistema per trovare la soluzione duale.

\begin{itemize}
	\item Il vincolo $3x_1 + x_2 \geq 11$ è già all'uguaglianza con la soluzione $x_1 = 3$ e $x_2 = 2$, quindi \textbf{non da} informazioni aggiuntive.
	\item Anche $x_2 \geq 2$ è già all'uguaglianza con $x_2 = 2$.
	\item Il vincolo $ x_1 \geq 1 $ non è all'uguaglianza e quindi, per rendere soddisfatta la condizione di complementarietà, la variabile duale associata al vincolo deve essere nulla. Pertanto ricavo $u_3 = 0$.
	\item Entrambe le variabili primali sono strettamente maggiori di 0 e quindi posso imporre le due equazioni associate alle variabili a 0: $ 3u_1 + u_3 = 2$ e $u_1 + u_2 = 3$
\end{itemize}

Il sistema di equazioni finali che ho ottenuto è:

$$
\begin{cases}
u_3 = 0 \\
3u_1 + u_3 = 2 \\
u_1 + u_2 = 3 \\
\end{cases} \quad \text{da cui ricavo } \quad \begin{cases}
u_1 = 2/3 \\
u_2 = 7/3 \\
u_3 = 0
\end{cases}
$$

Non ho ancora finito, perché per essere sicuro che sia ottima, devo controllare che è ammissibile. In questo caso lo è, e quindi che entrambe le soluzioni primale/duale sono ammissibili e sono tra loro complementari, allora sono anche entrambe ottime.








\textbf{Sottostringa}: serie di caratteri vicini

\textbf{Sottosequenza}: serie di caratteri non necessariamente vicini.

\section{Pattern matching di base}\label{pattern-matching-di-base}

Effettua il pattern matching esatto, cercando la sotto stringa \emph{P}
dentro la stringa \emph{T}.

\begin{breakablealgorithm}
	\caption{Ingenuo: Pattern matching ingenuo}
	\begin{algorithmic}[1]
	\Function{Ingenuo}{$ (P,T) $}
	\State $  $ \Comment{\textit{T} ha lunghezza \textit{n} e \textit{P} ha lunghezza $ m \leq n $}
	\For{$ i = 1 \: \text{to} n-m+1 $}
		\State $ j \gets 1 $
		\While{$ j \leq m \:\text{and} \:P[j] = T[i+j-1] $}
	        \State $ j \gets j + 1 $
	    \EndWhile
	    \If{$ j > m $}
	        \State Segnala l'occorrenza del pattern
	    \EndIf
	\EndFor
	\EndFunction
\end{algorithmic}
\end{breakablealgorithm}

\subsection{Utilizzo della sentinella}\label{utilizzo-della-sentinella}

La prima modifica che si può fare all'algoritmo per migliorarne
l'efficienza è quello di ridurre il test del \texttt{while}, rimuovendo
il controllo sulla lunghezza del pattern, riducendo così le operazioni
da 3 a 2.

Questo viene fatto aggiungendo una sentinella alla fine del pattern,
ovvero viene aggiunto al pattern un carattere che non compare
nell'alfabeto della stringa.

Perché questo funzioni è necessario aggiungere un carattere diverso
dalla sentinella anche alla fine di \emph{T} per permettere il match del
pattern anche quando questo è un suffisso della stringa.

\begin{breakablealgorithm}
	\caption{Ingenuo: Pattern matching ingenuo}
	\begin{algorithmic}[1]
		\Function{Ingenuo-2}{$ (P,T) $}
        \State //\textit{T} ha lunghezza \textit{n} e \textit{P} ha lunghezza $ m \leq n $
        \State $ P[m+1] \gets \$ $
        \State $ T[n+1] \gets @$
        \For{$ i = 1 \: \text{to } n-m+1 $}
	        \State $ j \gets 1 $
	        \While{$ P[j] = T[i+j-1] $}
		        \State $ j \gets j + 1 $
	        \EndWhile
	        \If{$ j > m $}
		        \State Segnala l'occorrenza del pattern
	        \EndIf
        \EndFor
        \EndFunction
\end{algorithmic}
\end{breakablealgorithm}

Nel caso non sia possibile modificare il testo si può togliere il
\texttt{+1} del ciclo \texttt{for} e sostituirlo con un \texttt{if} che
verifica l'uguaglianza dell'ultimo carattere.

D'ora in avanti assumeremo la presenza dei due caratteri sentinella.

\subsection{Riduzione del numero di confronti}\label{riduzione-del-numero-di-confronti}

Se il pattern contiene delle sottostringhe uguali, è possibile ridurre
il numero di controlli.

Ad esempio nel caso sotto riportato, l'algoritmo \textsc{Ingenuo}
effettua 20 confronti.

\begin{figure}[htbp]
\centering
\includegraphics[width = .4\textwidth]{./notes/immagini/l11-fig1.png}
\end{figure}

Ovvero è possibile evitare il confronto tra l'inizio del pattern e il
terzo carattere del testo, perché si sa già che il terzo carattere del
testo è uguale al secondo carattere del pattern, il quale è diverso dal
primo carattere del pattern. Lo stesso ragionamento vale anche per i due
confronti successivi.

Ma si può fare di più, perché il pattern ha una sottostringa uguale al
suo prefisso, ovvero i caratteri 7-8-9 della del testo sono uguali ad
una sottostringa del pattern che coincide con il prefisso del pattern e
dal momento che questo è già questa uguaglianza è già stata verificata,
si possono ridurre ulteriormente i confronti.

\begin{figure}[htbp]
\centering
\includegraphics[width = .4\textwidth]{./notes/immagini/l11-fig2.png}
\end{figure}

Per poter applicare queste osservazioni ad un algoritmo è necessario
effettuare delle pre-elaborazioni delle stringhe.

\subsection{Pre-elaborazione fondamentale}\label{pre-elaborazione-fondamentale}

Data una stringa \emph{S} di lunghezza \emph{n}, la funzione
$\pi_i^S$ calcola la lunghezza del prefisso di \emph{S} più lungo
che occorre nella posizione \emph{i} di \emph{S}.

Quindi $\pi_i^S$ è il massimo \emph{h} tale che

$$
S[1,h] = S[i, i+h-1]
$$

Ad esempio:

\begin{figure}[htbp]
\centering
\includegraphics[width = .4\textwidth]{./notes/immagini/l11-fig3-bis.png}
\caption{$\pi_i$ per \textit{S=}\texttt{aabcaabdaae}.}
\end{figure}

Da notare che $Y = S[i, i+\pi_i -1]$ per definizione è un occorrenza in \textit{S} della stringa $S[1, \pi_i -1]$, pertanto si ha che \textit{Y} è un bordo della stringa $S[1, i +\pi_i -1]$. 
Per la relazione tra prefisso e bordo si ha quindi che $S[1, i +\pi_i -1]$ ha periodo $p = i -1$.

Si ha inoltre che la stringa $S[1, i +\pi_i -1]$ è il più lungo prefisso di \textit{S} con periodo $p= i-1$.

Questo perché se $i = 1$, si ha che il prefisso \textit{Y} coincide con \textit{S} e $p=0$, ottenendo periodo e bordo degeneri.

Se invece $ i \geq 2 $ si ha che, se $i + \pi_i -1 = n $, \textit{Y} è anche un suffisso di \textit{S}, pertanto non possono esserci altri prefissi con periodo $p = i -1$ più lunghi perché è la stringa \textit{S} è terminata.
Oppure se $i + \pi_i -1 \neq n $ vuol dire che il carattere $ S[\pi_i + 1] $ è diverso dal carattere $ S[i+\pi_i] $ per definizione di $ \pi_i $ e quindi non può esistere un prefisso $ S[1, \pi_i +1] $ con periodo $p = i -1$ perché $ S[\pi_i  + 1 + p] = S[\pi_i + i] $ che per ipotesi è diverso da $ S[\pi_i +1] $.

Le varie sottostringhe $ S[i, i + \pi_i -1 ] $ non sono necessariamente disgiunte ma possono sovrapporsi.

\subsubsection{Estremi massimi}

Fissato un $ i \geq 2 $, si ottengono varie stringhe $ S[j, j + \pi_j -1] $ con $ 2 \leq j \leq i $.

Tra tutte queste stringhe è possibile identificare la stringa che ha come valore dell' \textbf{estremo destro massimo}:

$$
r_i = \max \{ j + \pi_j -1 | \: 2 \leq j \leq i\}
$$

L'estremo sinistro associato viene indicato con 

$$
l_i = \arg\max\limits_{j} \{ j + \pi_j -1 | \: 2 \leq j \leq i\}
$$

Nel caso ci siano più sottostringhe con lo stesso estremo destro, viene scelto come estremo sinistro uno a caso tra quelli possibili.

La stringa $ S[l_i, r_i] $ risulta quindi essere la sottostringa di \textit{S} più lunga che è anche un prefisso e che inizia prima dell'indice \textit{i}.

\begin{figure}[htbp]
	\centering
	\includegraphics[width = .9\textwidth]{./notes/immagini/l11-fig4-bis.png}
	\caption{Alcune occorrenze di prefissi di \textit{S} che iniziano tra le posizioni \textit{2} e \textit{i}. L'occorrenza che termina più a destra è evidenziata in rosso.}
\end{figure}

Si ha quindi che $ S[1,r_i] $ è il più lungo prefisso di \textit{S} con periodo $ 1 \leq p < i $ e che la stringa $ S[l_i, r_i] $ è la sottostringa di \textit{S} che termina più a destra tra tutte quelle che sono uguali ad un prefisso di \textit{S}.

Ad esempio se 

$$
S = \text{\texttt{aabaabcadaabaabce}}
$$

il massimo destro con $ 2 \leq j \leq 15 $ è $ r_{15} = 10 + \pi_{10} -1 = 16$, perché $ \pi_{10} = 7$, e $ l_{15} = 10 $ che è la posizione in cui inizia la sottostringa \texttt{aabaabc}.

\subsection{Pre-elaborazione fondamentale in tempo lineare}\label{preambolazione-fondamentale-in-tempo-lineare}

Seguendo l'approccio di definizione il tempo richiesto è
$O(n^2)$, ma è possibile scendere a $O(n)$.

Supponiamo che \emph{S} termini con un carattere diverso da tutti gli
altri che compaiono nella stringa. Questo non è un problema perché si
può sempre aggiungere una sentinella.

L'algoritmo calcola $\pi_1 = n$ e poi calcola i valori $\pi_i, r_i, l_i$ per $i = 2,\ldots, n$, basandosi su un'array $ pref[1\ldots n] $ il quale conterrà i vari $ \pi_i $ e due variabili, \textit{r} e \textit{l}, le quali andranno a memorizzare gli estremi massimi tra gli indici precedente calcolati.

Come prima cosa viene effettuato il calcolo di $ \pi_2 $ confrontando da sinistra a destra i caratteri di $S[2,n]$, cosi facendo si ha che $ r = \pi_2 +1 $ e $l = 2$.

Assumendo induttivamente di aver calcolato $ \pi_j $ per ogni $ j = 2, \ldots, i-1$, si ha che $ r = r_{i-1} = \pi_{i-1} +1 $ e $ l = l_{i-1} $.

Durante il calcolo di $ \pi_i $ possono verificarsi 2 casi:

\begin{enumerate}
	\item $ i > r $: non si hanno informazioni riguardo ai caratteri che seguono \textit{i}, quindi viene effettuato il calcolo di $ \pi_i $ normalmente, andando ad effettuare i confronti da sinistra a destra tra $ S[i,n] $ e \textit{S}. Il valore di $ \pi_i $ è allora uguale alla lunghezza \textit{h} del massimo prefisso comune e $ r = i  + \pi_i +1 $ e $ l= i $.
	\item $ i \leq r$: il carattere \textit{S[i]} è contenuto nella sottostringa $ \alpha = S[l,r] $ la quale è anche prefisso di \textit{S}. Si ha quindi che il carattere \textit{S[i]} compare anche nella posizione $ i' = i - l +1 $ di \textit{S} e per lo stesso motivo la stringa $ \beta = S[i,r] $ compare anche in $ S[i', \pi_l] $.
	\begin{figure}[htbp]
		\centering
		\includegraphics[width = .8\textwidth]{./notes/immagini/l11-fig3.png}
	\end{figure}
	In \textit{i'} occorrerà un prefisso $ \gamma $ di \textit{S} di lunghezza $ \pi_{i'} $, che può anche essere degenere.
	Questo prefisso occorrerà a partire dalle posizioni \textit{1} e \textit{i'} e potrà essere contenuto o contenere la sottostringa $ \beta $.
	Ne segue che \textit{S} e \textit{S[i,n]} hanno un prefisso in comune di lunghezza uguale al minimo tra $ \pi_{i'} $ e $ |\beta| = r - i +1 $.
	\begin{enumerate}
		\item $ \pi_{i'} < |\beta| $: il prefisso che inizia in \textit{i} ha la stessa lunghezza di quello che inizia in \textit{i'}, ed avendo già calcolato la lunghezza di quel prefisso si ha che $ \pi_i = \pi_{i'} $ senza effettuare alcun confronto.
		\begin{figure}[htbp]
			\centering
			\includegraphics[width = .8\textwidth]{./notes/immagini/l11-fig5.png}
		\end{figure}
		\item $ \pi_{i'} \geq |\beta|$: l'intera sottostringa $ \beta = S[i,r] $ deve essere un prefisso di \textit{S} e quindi $\pi_i \geq |\beta| = r - i +1$. L'algoritmo calcola quindi la lunghezza \textit{h} del massimo prefisso comune tra \textit{S} e \textit{S[i,n]} a partire dai caratteri $ |\beta| + 1 $ e $ i + |\beta| $, fino a che non trova un mismatch. A questo punto vengono posti $ \pi_i =h,\: r= i + \pi_i -i \: \text{e} \: l = i $.
	\end{enumerate}
	
\end{enumerate}

\begin{breakablealgorithm}
	\caption{Prefisso: Preelaborazione del prefisso in tempo lineare}
	\begin{algorithmic}[1]
		\Function{Prefisso}{$ S $}
		    \State // S stringa di lunghezza n > 1 con sentinella alla fine
		    \State $ pref[1] \gets n $  \Comment{$ \pi_i, \pi_1 = n$}
		    \State $ h \gets 0 $
		    \While{$ S[1+h] = S[2+h] $}\Comment{Calcola $\pi_2$}
		        \State $ h = h +1 $
		    \EndWhile
		    \State $ pref[2] \gets h $
		    \State  $ l \gets 2 $
		    \State $ r \gets 2 + h - 1$
		    \For{$ i = 3 \: \text{to} \: n $}
			    \If{$ r < i $} \Comment{Caso 1}
			        \State $ h \gets 0$
		            \While{$ S[1+h] = S[i+h]$}
		                \State $ h \gets h + 1 $
		            \EndWhile
		            \State $ pref[i] \gets h$
		            \State $ l \gets i $
		            \State $ r \gets i + h -1 $
		        \Else \Comment{Caso 2}
				    \If{$pref[i-l+1] < r - i +1$} \Comment{Caso 2a}
		                \State $pref[i] \gets pref[i - l +1]$
		            \Else \Comment{Caso 2b}
		               \State $h \gets r - i +1$
		                \While{$S[1+h] = S[i +h]$}
		                    \State $ h \gets h +1 $
		                \EndWhile
		                \State $pref[i] \gets h$
		                \State $l \gets i$
		                \State $ r \gets i +h -i$
			         \EndIf
		          \EndIf
		    \EndFor
		    \State \Return $ pref $
		   \EndFunction
	\end{algorithmic}
\end{breakablealgorithm}

La correttezza dell'algoritmo deriva da quanto detto prima

\paragraph{Complessità}\label{complessituxe0}

Se non viene presa in considerazione la complessità dei cicli
\texttt{while} si ha che la complessità è data da \emph{O(n)}.

I cicli \texttt{while} terminano quando viene trovato un mismatch e al
massimo vengono trovati \emph{n-1} mismatch (1 dal \texttt{while}
esterno, $n-2$ dai \texttt{while} dentro il ciclo \texttt{for}).

Ad ogni confronto con successo, il carattere destro ($S[i+h]$)
viene spostato a destra di 1 e, una volta terminato il \texttt{while}, questo viene posto a $r = i + h - 1$, ovvero risulta essere il carattere alla posizione $S[r+1]$.

All'iterazione successiva il ciclo \texttt{while} inizia con carattere destro $S[i]$ se $i > r$\footnote{Viene quindi sposto a destra perché $ i \geq r+1 $} altrimenti inizia con $S[i + h]$ con $h = r - i + 1$. In entrambi i casi il carattere destro non si sposta mai a sinistra durante l'esecuzione dell'algoritmo. 
Pertanto vengono eseguiti al più $n-1$ confronti con successo. +
Si ottiene quindi una complessità per i \texttt{while} di $O(2n-2)$.

La complessità totale dell'algoritmo è data da $O(n) +\textit{ Complessità while }= O(n) + O(2n-2) = O(n)$.

\subsubsection{Matching esatto in tempo lineare}\label{matching-esatto-in-tempo-lineare}

Per effettuare il pattern matching in tempo $O(m+n)$ del pattern \emph{P} in \emph{T} è possibile utilizzare una versione leggermente modifica della funzione prefisso sulla stringa \emph{S = P\$T}, dove \$ è un simbolo che non è presente nelle due stringhe.

Questo viene fatto calcolando $ \pi_i^S $ per $ i = 2, \ldots n + m +1 $.
Siccome \$ non compare in nessuna delle due stringhe, si ha che $ \pi_i \leq m \: \forall \: i $ perché per ipotesi la sottostringa \textit{P\$} non può comparire all'interno di \textit{T}.
Inoltre, $ \forall \: i \geq m +1 : \pi_i = m $ si ha che $i - m - 1$ identifica l'inizio di un'occorrenza di \textit{P} in \textit{T}, perché $ \pi_i = m $ indica la presenza di prefisso di lunghezza \textit{m} a partire dalla posizione \textit{i}, ma la sottostringa prefissa di lunghezza \textit{m} coincide per costruzione con \textit{P}, quindi \textit{i} indica l'inizio di un'occorrenza del pattern \textit{P} nella stringa \textit{P\$T}.

Dal momento che il prefisso viene calcolato in tempo lineare rispetto la lunghezza della stringa, si ottiene una funzione di pattern matching con complessità lineare.

Altre caratteristiche di questo algoritmo sono:
\begin{itemize}
	\item il \textbf{consumo lineare di memoria} rispetto la lunghezza del pattern $ O(m) $, perché è possibile evitare di tenere in memoria tutti $ \pi_i $ con $ i >m $ perché tutti i valori \textit{i'} del caso 2 faranno sempre riferimento ai $ \pi_i \: \text{con}\: i \leq m $
	\item che non è necessario conoscere tutto l'alfabeto, basta avere la possibilità di confrontare i caratteri.
\end{itemize}

\subsubsection{Esercizio - Identificare una rotazione}

Date due stringhe \textit{X} ed \textit{Y} di uguale lunghezza \textit{n}, determinare in tempo lineare \textit{O(n)}, se \textit{Y} è una rotazione circolare di \textit{X}.
Ovvero se è possibile identificare due stringhe $ \alpha $ e $ \beta $ tali che $ X = \alpha\beta $ e $ Y = \beta\alpha $.

\paragraph{Soluzione}

Si concatenano le stringhe in modo da formare $ S = Y\$XX $, se eseguendo la preelaborazione trovo un prefisso di lunghezza \textit{n} a partire dal $n+1$ vuol dire che \textit{Y} compare tra le due \textit{X} e quindi le due stringhe sono la rotazione di una stessa stringa.

\subsubsection{Esercizio - Massima sottostringa comune}

Date due stringhe \textit{X} e \textit{Y} di lunghezza \textit{m} e \textit{n} calcolare, in tempo lineare, il più lungo suffisso di \textit{X} che è anche prefisso di \textit{Y}. 
Ovvero la sottostringa $ \gamma $ di lunghezza massima tale che $ X = \alpha\gamma \: \text{e} \: Y = \gamma\alpha$.

\paragraph{Soluzione}

L'idea è quella di concatenare due stringhe in modo da formare $ S= Y\$X $, dove \$ è un carattere che non compare in nessuna delle due stringhe, per poi effettuare la preelaborazione.

Se $ \pi_2^S = 0$ non c'è nessuna sottostringa uguale al prefisso di \textit{Y}, quindi $ \gamma = \epsilon $, altrimenti se  $ 0 < \pi_2^S = k \leq m$ si ha che c'è un'occorrenza all'interno di \textit{S} del prefisso di \textit{Y} lunga \textit{k}, ma non si ha alcuna garanzia che questa sia un suffisso per \textit{X}.

Se  $ \pi_{n+m+1-k}^S = k$ si ha che a partire dal carattere $ n+m+1-k $ c'è un match di una sottostringa di lunghezza \textit{k} che per costruzione è anche suffisso di \textit{X}, pertanto $ \gamma = Y[1,k] $, altrimenti la sottostringa precedentemente identifica si trova o all'interno di \textit{Y} o all'interno di \textit{X}, pertanto $ \gamma = \epsilon $.

\todo[inline]{Quasi corretto, bisogna andare a vedere i $ \pi_j $ per trovare il suffisso più lungo che è anche prefisso.}



% !TEX encoding = UTF-8
% !TEX TS-program = pdflatex
% !TEX root = computabilità e algoritmi.tex
% !TEX spellcheck = it-IT
\paragraph{Soluzione}\label{soluzione-esercizio}

La funzione \emph{f(x)} viene definita per casi e dal momento che i vari
predicati sono decidibili ed esaustivi, quindi almeno uno dei casi è
vero.

$$
f(\vec{x}) = f_1(\vec{x}))\cdot \mathcal{X}_{Q_1} + \ldots +	 f_m(\vec{x}))\cdot \mathcal{X}_{Q_m}
$$

Quando un caso è vero, viene calcolata la funzione associata, che è
totale. Se questa funzione non fosse totale, potrebbe essere che il
predicato vero su un certo \emph{x}, ma che la funzione su
quell'\emph{x} non sia definita e quindi neanche \emph{f(x)}
risulterebbe definita, perdendo così la totalità.

La calcolabilità deriva dal fatto che la definizione per casi viene
fatta con una serie di \emph{if} che è dimostrato essere calcolabile e
le funzioni $\mathcal{X}_{Q_i}$ sono calcolabili, perché i predicati sono
decidibili.

\subsubsection{Algebra della decidibilità}\label{algebra-della-decibilituxe0}

I predicati decidibili sono chiusi rispetto negazione, congiunzione e
disgiunzione.

Ovvero se \textit{Q} e \textit{Q'} sono due predicati in $\mathbb{N}^k$ decidibili allora anche:

\begin{enumerate}
\item $\neg Q(\vec(x))$
\item $ Q(\vec{x}) \wedge Q'(\vec{x})$
\item $ Q(\vec{x}) \vee Q'(\vec{x}) $
\end{enumerate}

Questo perché le relative funzioni che li calcolano possono essere definite come:

\begin{enumerate}
	\item $ \mathcal{X_{\neg Q}}(\vec{x}) = \overline{sg}( \mathcal{X_{Q}}(\vec{x})) $
	\item $ \mathcal{X_{Q \wedge Q'}}(\vec{x})= \mathcal{X_{Q}}(\vec{x}) \cdot \mathcal{X_{Q'}}(\vec{x})$
	\item $ \mathcal{X_{Q \vee Q'}}(\vec{x})= sg(\mathcal{X_{Q}}(\vec{x}) + \mathcal{X_{Q'}}(\vec{x}))$
\end{enumerate}

Tutte le funzioni così definite sono calcolabili perché ottenute da composizioni di funzioni calcolabili.

\subsubsection{Somma e prodotto dei valori di una funzione}

Data una funzione totale e $ f : \mathbb{N}^{k+1} \rightarrow  \mathbb{N}$ totale e calcolabile, è possibile definire le due funzioni che effettuano la somma e il prodotto dei primi \textit{y} valori della funzione.

\begin{align*}
	s(\vec{x}, y) &= \sum_{z < y} f(\vec{x},z) \\
	p(\vec{x}, y) &= \prod_{z < y} f(\vec{x},z) 
\end{align*}

Entrambe le funzioni sono calcolabili e totali perché possono essere definite induttivamente (ricorsivamente) a partire da delle funzioni calcolabili.

\begin{align*}
s(\vec{x}, y) &= \begin{cases} \sum_{z < 0}f(\vec{x},z) = 0, &\text{ se $ y = 0 $} \\
\sum_{z < y+1}f(\vec{x},z) = \sum_{z < y}f(\vec{x},z) + f(\vec{x},z), &\text{ altrimenti}
\end{cases} \\
p(\vec{x}, y) &=  \begin{cases} \prod_{z < 0}f(\vec{x},z) = 1, &\text{ se $ y = 0 $} \\
\prod_{z < y+1}f(\vec{x},z) = \prod_{z < y}f(\vec{x},z) \cdot f(\vec{x},z) &\text{ altrimenti}
\end{cases}
\end{align*}

La totalità deriva dal fatto che \textit{f} è totale.

\subsubsection{Quantificazione limitata}

Combinando algebra della decidibilità e quanto detto nel paragrafo precedente è possibile la decidibilità di $ \forall $ e $ \exists $.

Dato un predicato $ Q(\vec{x},z) $ per calcolare se $ \forall z < y, Q(\vec{x},z) $ è possibile utilizzare la funzione:

$$
\mathcal{X}_{Q_\forall} = \prod_{z < y} \mathcal{X}_Q(\vec{x},z)
$$

In modo simile è possibile calcolare $ \exists z < y, Q(\vec{x},z) $:

$$
\mathcal{X}_{Q_\exists} = sg(\sum_{z < y} \mathcal{X}_Q(\vec{x},z))
$$

Trattandosi della composizione di funzioni calcolabili e totali, le funzioni così ottenute sono a loro volta calcolabili e totali.

\section{Minimalizzazione limitata}

Data una funzione $ f(\vec{x},z) : \mathbb{N}^{k+1} \rightarrow \mathbb{N}$ calcolabile e totale è possibile definire una funzione 

$$
h(\vec{x},y) = \mu z< y | f(\vec{x},z) = 0
$$

$ h $ è ancora una funzione $ \mathbb{N}^{k+1} \rightarrow \mathbb{N} $ e viene calcolata come il minimo valore di \textit{z} minore di \textit{y} e tale che $ f(\vec{x},z) $ sia uguale a 0 (tipicamente l'uguale a 0 viene omesso). 

La definizione più precisa è:

$$
h(\vec{x}, y) = \mu z < y . f(\vec{x},z) = \begin{cases}
\text{minimo $ z < y $ tale che $f(\vec{x},z) = 0$ se questo esiste} \\
y, \text{altrimenti}
\end{cases}
$$

Per come è definita, questa funzione risulta essere \textbf{totale} e \textbf{calcolabile}.
Intuitivamente è calcolabile perché si tratta di calcolare \textit{f} per vari valori, serve però una dimostrazione più formale, fatta per ricorsione primitiva.

\begin{align*}
	h(\vec{x}, 0) &= 0\\
	h(\vec{x}, y+1) &= 	\begin{cases}
										h(\vec{x},y) < y, &\text{\textit{f} si annulla su un valore minore di \textit{y}, viene resituito $ h(\vec{x},y) $}\\
										h(\vec{x},y) = y, &\text{per tutti i valori di minori \textit{y} non c'è uno 0 } \begin{cases}
										\text{se } f(\vec{x},y) = 0 \rightarrow y \\ 
										\text{se } f(\vec{x},y) \neq 0 \rightarrow y+1
										\end{cases}
										\end{cases}
\end{align*}

La seconda parte può essere facilmente tradotta nell'espressione

$$
(y \dotminus h(\vec{x},y)) \cdot (h(\vec{x},y)) + \overline{sg}(y \dotminus h(\vec{x},y)) \cdot (y + sg(f(\vec{x},y)))
$$

Dal momento che la funzione \textit{h} può essere definita per ricorsione primitiva e per composizione di funzioni calcolabili, anche lei è calcolabile.

\subsection{Funzioni calcolabili per ricorsione limitata}

Utilizzando la ricorsione limitata è possibile dimostrare la calcolabilità di varie funzioni.

\subsubsection{Numero di divisori di $x$}

\begin{align*}
	D(x) &= \text{\# divisori di } x \\
			&= \sum_{y \leq x}(\overline{sg}(rm(y,x))) 
\end{align*}

Dove \textit{rm} è la funzione resto, precedentemente dimostrata calcolabile.

\subsubsection{Numeri primi}

Dimostrare la calcolabilità dei funzioni che lavorano con i numeri primi è importante perché torneranno utili in futuro.

\begin{align*}
	Pr(x) &= \text{``$x$ è primo ''} \\
			 &= \overline{sg}(|D(x) - 2|)
\end{align*}

\begin{align*}
	P_x &= \text{``$x$-esimo numero primo, per convezione: ''} P_0 = 0, P_1 = 2, \ldots \\
	&= \begin{cases}
	P_0 = 0& \\
	P_{x+1} = \mu z \leq (P_x! + 1) \: | Pr(z) \cdot \underbrace{\overline{sg}(P_x +1 \dotminus z)}_{1 \text{ se } z > P_x  } \dotminus 1| &
	\end{cases}
\end{align*}

\begin{align*}
	(x)_y 	&= \text{esponente di  } P_y \text{ nella decomposizione di } y \\
			   &= \text{max } z \: P_{y}^z \text{ divide } x \rightarrow \mu z \leq x \text{ tale che } P_{y}^{z+1} \text{ non divide } x \\
			   &= \mu z \leq x \: \overline{sg}(rm(P_{y}^{z+1},x))
\end{align*}

\subsubsection{Esercizio - mcm, MCD, radice di x}

Dimostrare che sono calcolabili:

\begin{itemize}
	\item $ floor(\sqrt{x}) $
	\item $mcm(x,y))$
	\item $MCD(x,y)$
\end{itemize}


\section{Codifica di coppie}

La funzione di \textit{fibonacci} non può essere definita per ricorsione primitiva, perché il passo induttivo richiede una coppia di valori precedenti.

\`{E} però possibile definire una funzione $\prod : \mathbb{N} \times \mathbb{N} \rightarrow \mathbb{N}$ che codifica il valore di una coppia in un unico numero:

$$
\prod (x,y) = 2^x(2y+1) \dotminus 1
$$

Questa funzione risulta essere biunivoca perché è possibile definire l'inversa:

$$
\prod^{-1}(x) = ((n+1)_1, \frac{1}{2}(\frac{n+1}{(n+1)_1})-1)
$$

La funzione inversa è effettiva, perché è definita in termini di componenti calcolabili, anche se la definizione di funzione calcolabile non è stata vista per funzioni $\mathbb{N} \rightarrow \mathbb{N} \times \mathbb{N}$.

Utilizzando la funzione accoppiamento, è possibile definire la funzione \textit{fib} per ricorsione primitiva:

\begin{align*}
	g(x) &= \prod(fib(x), fib(x+1)) \\
	g(0) &= \prod(fib(0), fib(1)) = \prod(1,1) = 5 \\
	g(x+1) &= \prod(fib(x+1), fib(x+2)) \\
				 &= \prod(fib(x+1, fib(x)+fib(x+1)) \\
				 &= \prod(\prod_2(g(x)),\prod(\prod_1(g(x))+\prod_2(g(x)))\\		 
\end{align*}

Dove $\prod_1$ e $\prod_2$ sono rispettivamente le funzioni per il calcolo del primo e del secondo elemento della coppia.

Così facendo è stata dimostrata la calcolabilità di \textit{g} per ricorsione primitiva, ma $fib(x) = \prod_1(g(x))$ e di conseguenza \textit{fib} è calcolabile per composizione di funzioni calcolabili.

\section{Minimalizzazione illimtata}

Data $ f : \mathbb{N}^{k+1} \rightarrow \mathbb{N} $, si vuole definire $ h : \mathbb{N}^k \rightarrow \mathbb{N} $ tale che calcoli il minimo \textit{z} che azzera la funzione \textit{f}, ovvero:

$$
h(\vec{x}) = \mu z.f(\vec{x},z)
$$

Ci sono però dei problemi se $ f(\vec{x}, z) $ è sempre diversa da zero, perché in questo caso $ h $ è $ \uparrow $.

Un altro problema si ha se la funzione è indefinita per un valore $ z' $ minore dello $ z $ che azzera la funzione. Anche in questo caso si ha che $ h $ è $ \uparrow $.

$$
h(\vec{x}) = \mu z.f(\vec{x},z) = \begin{cases}
\text{minimo \textit{z} tale che } f(\vec{x},z) = 0 \text{ se esiste e se } \forall z' < z, f(\vec{x},z')\downarrow \neq 0 \\
\uparrow \text{ altrimenti}
\end{cases}
$$

Alternativamente, definendo $ Z_{f, \vec{x}} = \{z | f(\vec{x},z) = 0 \wedge \forall z' < z f(\vec{x},z') \downarrow \} $, si ha che \textit{h} è definita come

$$
h(\vec{x}) = \begin{cases}
\min Z_{f,\vec{x}} \text{ se } Z_{f,\vec{x}} \neq \emptyset \\
\uparrow \text{ altrimenti}
\end{cases}
$$

\subsection{Esercizi}

\subsubsection{Esercizio - Radice quadrata}
Dimostrare la calcolabilità di 

$$
f(x) = \begin{cases}
\sqrt{x} \text{ se \textit{x} è un quadrato} \\
\uparrow \text{ altrimenti}
\end{cases}
$$

\paragraph{Soluzione}

L'idea è quella di trovare un \textit{y} che elevato al quadrato è uguale a \textit{x}: $ y^2 - x = 0 $.

Si ha quindi che

$$ 
f(x) = \mu y.|y^2-x|
$$

ed è calcolabile perché minimizza illimitatamente una composizione di funzione calcolabile.

\subsubsection{Esercizio teorico}

Dimostrare che se $ f : \mathbb{N} \rightarrow \mathbb{N} $ è iniettiva, calcolabile e totale, anche la sua inversa è calcolabile.

\paragraph{Soluzione}

$$
f^{-1}(x) = \begin{cases}
y, &\text{ tale che } f(y) = x \\
\uparrow, &\text{ altrimenti}  
\end{cases}
$$

$$ 
f^{-1}(x) = \mu y.|f(y)-x|
$$

Perché l'uguaglianza sia rispettata è necessario che \textit{f} sia totale, dal momento che se per un certo \textit{y}, \textit{f} non è definita il programma che la calcola non termina, rendendo indefinita anche $ f^{-1} $.

C'è un barbatrucco per gestire anche la non totalità di \textit{f}, ovvero quello di eseguire contemporaneamente il calcolo per ogni \textit{y}, eseguendo tot passi alla volta per ognuno dei calcoli (\textit{dovrebbe essere dimostrato in futuro}).

Dal momento che \textit{f} è calcolabile si ha che anche l'inversa è calcolabile per minimizzazione illimitata di una composizione di funzioni calcolabili.

L'iniettività\footnote{Una funzione $ f: X\rightarrow Y $ si dice iniettiva se due elementi distinti del dominio hanno immagini distinte, ovvero $a_1\neq a_2$ implica $f(a_1)\neq f(a_2)$.} garantisce che il valore trovato sia quello corretto.


\subsubsection{Esercizio - Divisione}

$$
f(x,y) = \begin{cases}
\frac{x}{y}, &\text{ se } y \neq 0 \text{ e \textit{x} divisibile per \textit{y}}\\
\uparrow, &\text{altrimenti} 
\end{cases}
$$

\paragraph{Soluzione}

$$
f(x,y) = \mu k. |x \dotminus y\cdot k|
$$

C'è però un problema, perché la funzione così definita risulta calcolabile se \textit{x} e \textit{y} sono uguali a 0.

$$
f(x,y) = \mu k.(|x - y\cdot k| + \underbrace{\overline{sg}(y)}_{\text{vale 1 se \textit{y} è uguale a 0}})
$$

Così facendo se $ y=0 $ la minimalizzazione non converge.

Questo porta ad un discorso un po' più ampio sulla possibilità di aggiustare una funzione che si comporta quasi come un'altra funzione (§\ref{hotfix}).

\section{Lezione 13 - Support Vector Machine}\label{lezione-13---support-vector-machine}

Nelle precedenti puntate:

\begin{itemize}
\item
  Sappiamo che un iperpiano in uno spazio di dimensione \textit{m} ha VC
  dimension \textit{m+1}.
\item
  Si può aggiungere un vincolo di classificazione relativo al margine.
\item
  Per ottenere l'iperpiano con margine ottimo è necessario considerare
  l'ipotesi che minimizza la norma di \emph{w}.
\item
  Il tutto si fa con un polinomio di Lagrange e il suo duale.
\end{itemize}

\subsection{Dati non separabili linearmente}\label{dati-non-separabili-linearmente}

Tutto quello visto finora funziona se i dati sono linearmente
separabili.

Nel caso questi non lo siano è necessario permettere che alcuni vincoli possano essere violati e per fare ciò vengono introdotte delle nuove variabili $\xi_i \geq 0$, una per ogni vincolo (ovvero per ogni esempio del training set), tale che:

$$ y_i (\vec{w} \cdot \vec{x}_i + b) \geq 1 - \xi_i $$

Queste nuove variabili rappresentano la distanza dell'esempio \textit{i}-esimo dal margine.

\begin{figure}[htbp]
\centering
\includegraphics[width = 0.6\textwidth]{./notes/immagini/l13-non-linear.png}
\caption{SVM con dati non linearmente separabili.}
\end{figure}

L'idea è quindi quella di andare a sommare alla funzione costo la sommatoria di tutti i $\xi_i$ dei vari esempi presenti nel training set, moltiplicata per un coefficiente di penalizzazione \textit{C} che rappresenta un iper-parametro dell'algoritmo di apprendimento, da ottimizzare con le tecniche di model selection.

La nuova formula da minimizzare diventa:

$$ \frac{1}{2}||\vec{w}||^2 + C \sum\limits_{i=1}^n \xi_i $$

In pratica vengono penalizzati (aumentato il costo) gli esempi che non rispettano il margine.

La minimizzazione avviene considerano il problema duale, che risulta essere definito come:

$$max_\alpha \sum\limits_{i=1}^n \alpha_i - \frac{1}{2}\sum\limits_{i,j = 1}^n y_i y_j \alpha_i \alpha_j (\vec{x}_i \cdot \vec{x}_j)$$

$$ \text{s.t.: } \forall i \in \{1, \ldots, n\} : 0 \leq \alpha_i \leq C \text{ e } \sum\limits_{i=1}^n y_i \alpha_i = 0$$

Da notare che le $\xi_i$ sono variabili del problema primale e che quindi non compaiono nel problema duale.

Questa strategia per esempi non linearmente separabili non sempre
garantisce buone prestazioni perché un iper-piano può solo rappresentare
dicotomie dello spazio delle istanze.

Per questo motivo, quando gli esempi non sono linearmente separabili su
usa una strategia divisa in due passi:

\begin{enumerate}
\item
  Si mappano i dati di ingresso (input space) in uno spazio a dimensione
  molto superiore (feature space). Quindi a partire dalle feature degli
  elementi dell'input space vengono creati nuovi esempi nel feature
  space che utilizza combinazioni non lineari delle feature del primo
  spazio.
\item
  Si calcola poi l'iper-piano ottimo per il nuovo spazio usando la
  formulazione precedente (che prende il nome di variabili slack).
\end{enumerate}

Perché dovrei farlo?

\begin{enumerate}
\item
  Perché il \textbf{teorema sulla separabilità di Cover} afferma che un problema di classificazione complesso, formulato
  attraverso una trasformazione non lineare dei dati in uno spazio ad
  alta dimensionalità, ha maggiore probabilità di essere linearmente
  separabile che in uno spazio a bassa dimensionalità.
\item
  Perché l'iper-piano ottimo minimizza la VC-Dimension e quindi la
  capacità di generalizzazione migliora.
\end{enumerate}

Viene quindi utilizzata una trasformazione $\varphi(\cdot)$ non lineare, da applicare ai dati originari del problema $\{(\vec{x}_i, y_i)\}_1^n$ tale che:

$$ \forall i \: \vec{x}_i \in R^m, \varphi(\vec{x}_i) = \vec{Z}, \vec{Z} \in R^M, M \gg m $$

Il vettore ottenuto può essere rappresentato come $\vec{\varphi}(\vec{x}) = [ \varphi_1(\vec{x}), \ldots , \varphi_M(\vec{x}) ] $.

Con questa notazione è possibile andare a definire l'iper-piano nel nuovo spazio con:

$$ \sum\limits_{j=1}^M w_j \varphi_j(\vec{x}) + b = 0$$

che se si considera il termine noto $b = w_0$ e si aggiunge $\varphi_0(\vec{x}) = 1$, risulta essere

$$ \sum\limits_{j=0}^M w_j \varphi_j(\vec{x}) = \vec{w} \cdot \vec{\varphi}(\vec{x}) = 0$$

Andando a sostituire il $\vec{w}$ dell'equazione precedente con $ \vec{w} = \sum\limits_{k=1}^n j_k \alpha_k \vec{\varphi}(\vec{x}_k)$ si ottiene:

$$ \sum\limits_{j=1}^M w_j \varphi(\vec{x_k}) \cdot \varphi(\vec{x}) = 0 $$

Con il termine $\varphi(\vec{x}_k) \cdot \varphi(\vec{x})$ rappresenta il prodotto scalare tra un vettore del training set $\vec{x}_k$ e il vettore in input $\vec{x}$ calcolato nello spazio $R^M$.

\subsection{Funzioni Kernel}\label{funzioni-kernel}

Lo spazio di dimensione superiore serve solo per calcolare il prodotto scalare tra i due vettori, si può quindi definire una funzione $K(\cdot, \cdot)$ che prende il nome di kernel e che calcola il prodotto scalare dei due vettori senza passare esplicitamente nello spazio di dimensione superiore.

$$ K(\vec{x}_k, \vec{x}) = \varphi(\vec{x}_k) \cdot \varphi(\vec{x})$$

Assumendo di avere una di queste funzioni, l'iper-piano risulta essere:

$$\sum\limits_{k=1}^n  y_k \alpha_k K(\vec{x}_k, \vec{x}) = 0$$

Per il teorema di Mercer esistono delle funzioni di questo tipo, ma è necessario che soddisfino determinate condizioni.

Alcune di queste sono:

\begin{itemize}
\item \textbf{Polinomiale di grado \textit{p}}: $K(\vec{x},\vec{y}) = (\vec{x} \cdot \vec{y} +1) ^p$
\item \textbf{RBF}: $ K(\vec{x},\vec{y}) = exp(-\frac{1}{2\sigma^2}||\vec{x}-\vec{y}||^2)$
\end{itemize}

La formulazione duale del problema risulta quindi essere:

$$max_\alpha \sum\limits_{i=1}^n \alpha_i - \frac{1}{2}\sum\limits_{i,j = 1}^n y_i y_j \alpha_i \alpha_j K(\vec{x}_i ,\vec{x}_j)$$

$$ \text{s.t.: } \forall i \in \{1, \ldots, n\} : 0 \leq \alpha_i \leq C \text{ e } \sum\limits_{i=1}^n y_i \alpha_i = 0$$

Per fare la classificazione viene utilizzato il \textbf{segno} della funzione:

\begin{align*}
f(\vec{u})  &= \sum\limits_{i = 1}^n y_i  \alpha_i  K(\vec{x}_i,  \vec{u}) + b
\end{align*}

\subsection{Regressione}\label{regressione}

Quando si considera il problema di approssimazione di funzioni a valori
reali (regressione) si utilizza l'$\epsilon$-tubo: output che differiscono dai
valori di target per più di $\epsilon$ in valore assoluto vengono penalizzati
linearmente, altrimenti non vengono considerati errori. In pratica
aggiungo un intervallo di tolleranza al iper-piano che partiziona lo
spazio.

\begin{figure}[htbp]
\centering
\includegraphics[width = 0.8\textwidth]{./notes/immagini/l13-primale-duale.png}
\caption{Regressione in forma primale (a sinistra) e duale (a destra)}
\end{figure}

\todo[inline]{Mancano formule (ultime due slide) http://www.math.unipd.it/\~{}aiolli/corsi/1516/aa/SVM.pdf}
% !TEX encoding = UTF-8
% !TEX program = pdflatex
% !TEX root = MEMOC.tex
% !TEX spellcheck = it-IT

% 15 Dicembre 2016

\todo[inline]{Da recuperare}
\section{Metodi seminumerici}\label{metodi-seminumerici}

Famiglia di algoritmi per il pattern matching che utilizzano operazioni
sui bit e aritmetiche al posto del confronto di caratteri.

\subsection{Metodo Shift-And}\label{metodo-shift-and}

Metodo che funziona molto bene per pattern corti.

L'algoritmo calcola una matrice booleana $M_i[j]$ con
$n+1$ righe e \emph{m} colonne, tale che

$$
M_i[j] = 1 \text { se e solo se } P[1,j] = T[i-j+1, i]
$$

ovvero, $ M_i[j] = 1 $ se e solo se in $ T[i-j+1] $ inizia il prefisso del pattern $ P[1,j] $, il che implica che se nell'ultima colonna della riga \emph{i}-esima
della matrice c'è un 1, allora c'è un'occorrenza del pattern a partire
dalla posizione $i-m+1$.

Prima di effettuare la ricerca l'algoritmo effettua una pre-elaborazione
nella quale calcola un vettore booleano $U_x$ di lunghezza
\emph{m} per ogni carattere dell'alfabeto, tale che

$$U_x[j] = \begin{cases} 
1,& \text{ se } P[j] = x\\
0, & \text{ altrimenti} 
\end{cases}$$

Ovvero $U_x$ memorizza tutte le posizioni del pattern in cui
compare il carattere \emph{x}.

L'algoritmo poi si basa sull'operazione di \textsc{Shift} a destra, la
quale modifica una riga di bit spostando tutti i bit a destra di un
posto, scartando il bit più a destra e inserendo un 1 nella posizione più
a sinistra.

\begin{align*}
R &= 0111001000101 \\
\textsc{Shift}(R) &= 1011100100010 \\
\textsc{Shift(Shift}(R)) &= 1101110010001
\end{align*}

Il vantaggio di questo approccio è che se la parola è sufficientemente
corta, questa istruzione viene eseguita da una singola istruzione
macchina.

La prima riga della matrice $M_0$ ha tutti i bit uguali a 0 perché
nessun prefisso non nullo del pattern può occorrere nel testo prima del
carattere 0.

Le righe successive vengono calcolate utilizzando

$$M_i = \textsc{And}\big(\textsc{Shift}(M_{i-1}), U_{T[i]}\big) $$

dove \textsc{And} è l'operazione macchina che effettua l'\textit{and} logico bit
a bit.

\subsubsection{Dimostrazione di correttezza}\label{dimostrazione-di-correttezza}

La dimostrazione viene effettuata per induzione.

Il caso base riguarda $M_0$, ma questo è corretto per definizione, in quanto nessun prefisso non nullo del pattern può occorrere prima del carattere 0.

Nel caso induttivo si ha che:

$$M_i{[1]} = 1 \Leftrightarrow \big( \underbrace{1}_{\text{lo shift inserisce a sinistra un 1}} \wedge\: U_{T[i]}[1] = 1  \big)$$

ma, per costruzione di $ U_x $, $U_{T[i]}[1] = 1 \Leftrightarrow P[1] = T[i] $, pertanto il primo bit della riga risulta calcolato correttamente. 

Per i bit successivi al primo si ha:

$$
M_i[j] = 1 \Leftrightarrow \big( M_{i-1}[j-1] = 1 \wedge U_{T[i]}[j] = 1 \big)
$$

Per ipotesi induttiva $M_{i-1}[j-1]$ è stata calcolata correttamente, quindi se $M_{i-1}[j-1] = 1$, si ha che:

$$
P[1, j-1] = T[i -j +1, i-1]
$$

Dal momento che $U_{T[j]} = 1$ se e solo se $P[j] =T[i]$, si ha che

$$
P[1,j] = T[i-j+1,i]
$$

e quindi il calcolo di $M_{i}[j]$ viene effettuato in modo
corretto.

\subsubsection{Algoritmo}\label{algoritmo}

\begin{breakablealgorithm}
	\caption{ShiftAnd: Algoritmo }
	\begin{algorithmic}[1]
\Function{ShiftAnd}{$P,T,\Sigma$}
    \For{$ \forall x \in \Sigma$}
        \State $ U_x \gets \vec{0} $
    \EndFor
    \For{$ j = 1 \text{ to } m$}
        \State $U_{P[j]}[j] =1$
    \EndFor
    \State $M_0 \gets \vec{0}$
    \For{$i = 1 \text{ to } n$}
        \State $M_i \gets \text{\textsc{And}}(\text{\textsc{BitShift}}(M_{i-1}), U_{T[i]})$
        \If{$M_i[m] = 1$}
            \State occorrenza in $i -m +1$
        \EndIf
    \EndFor
\EndFunction
\end{algorithmic}
\end{breakablealgorithm}

La complessità dell'algoritmo è $O(\Sigma + m + n)$, tuttavia può essere ridotta andando ad inizializzare $U_x$ solo
per i caratteri che compaiono nel pattern.

\subsection{Metodo dell'impronta di Karp e Rabin}\label{metodo-dellimpronta-di-karp-e-rabin}

\textbf{Semplificazione}: consideriamo solo stringhe che contengono 0 o
1, tanto l'algoritmo può comunque essere esteso ad alfabeti con più
caratteri.

L'algoritmo considera quindi le stringhe di bit come dei numeri,
rendendo così possibile utilizzare le operazioni di confronto tra numeri
anziché quello tra caratteri.

Con $T_i$ viene indicata la porzione di testo che inizia dal
carattere \emph{i} e che è lunga quanto il pattern

$$
T_i = T[i, i + m -1]
$$

Data una stringa binaria \emph{S} di lunghezza \emph{m}, questa può
essere vista come la codifica di un numero \emph{H(S)}

$$
H(S) = \sum\limits_{j = 1}^m 2^{m-j}S[j]
$$

Così facendo è possibile calcolare sia \emph{H(P)} che $H(T_i)$ e
i due numeri risultano essere uguali se e solo se \emph{P = T}.

Perché le operazioni aritmetiche possano essere eseguite in tempo
costante è necessario che i numeri utilizzati siano rappresentabili con
$O(\log n)$ bit dove \emph{n} è la dimensione dell'input
(\textbf{ipotesi del modello RAM}). Pertanto se il pattern è lungo,
questa ipotesi non vale.

Rabin e Karp hanno pubblicato il \textbf{metodo dell'impronta
randomizzata}: viene scelto casualmente un numero \emph{p} e viene
calcolata

$$H_p(S) = H(S) \mod p$$

Le quantità così calcolate prendono il nome di impronta.

Se c'è un'occorrenza è ovvio che le due impronte sono uguali, ma possono
verificarsi dei casi in cui le due impronte sono uguali e non c'è
un'occorrenza. 

Tuttavia è possibile scegliere \emph{p} in modo che la
probabilità che ci sia una ``\emph{falsa occorrenza}'' sia molto bassa.

C'è però ancora un problema, serve un modo per effettuare questi conti
utilizzando numeri piccoli in modo da avere l'esecuzione delle
istruzioni in tempo costante.

Per risolvere questo problema è possibile utilizzare l'aritmetica
modulare e la regola di Horner: data una stringa $S = b_1 \ldots b_m$, $H_p(S)$ può essere calcolato nel seguente modo

\begin{breakablealgorithm}
	\begin{algorithmic}[1]
\Function{H$ _p$}{$S$}
    \State $ H = b_1 $
    \For{$ j = 2 \text{ to } m$}
        \State $H \gets (2H + b_j) \mod p$
    \EndFor
    \State \Return $H$
\EndFunction
\end{algorithmic}
\end{breakablealgorithm}


Così facendo non ci sono risultati intermedi di lunghezza maggiore di
$2p-1$ (\emph{2H} è lungo al massimo $2p-2$).

Nel calcolo di $H_p(T_i)$ con $i > 2$ è possibile effettuare delle ottimizzazioni, andando a calcolarlo a
partire da $H_p(T_{i-1})$, riducendo la complessità da
$O(mn)$ a $O(n)$:

\begin{verbatim}
|------ T i-1 ------|
X|------ T i   ------|

|------ T i-1 ------|0  <-- 2 T[i-1]
X|------ T i   ------|

 |----- T i-1 ------|0  <-- 2 T[i-1] - 2^m T[i-1]
X|------ T i   ------|

 |----- T i-1 -------|  <-- 2 T[i-1] - 2^m T[i-1] + T[i+m-1]
X|------ T i   ------| 
\end{verbatim}

\begin{align*}
	H(T_i) &= \sum\limits_{j=1}^m 2^{m-j} T[i+j -1] \\
	           &= T[i+m -1] + \sum\limits_{j=1}^{m-1} 2^{m-j} T[i+j -1] \\
	           &= T[i+m -1] + \sum\limits_{j=2}^{m-1} 2^{m-j+1} T[i+j -2] \\
	           &= T[i+m -1] - 2^m T[i-1] + \sum\limits_{j=1}^{m-1} 2^{m-j+1} T[i+j -2] \\
	           &= T[i+m -1] - 2^m T[i-1] + 2\sum\limits_{j=1}^{m-1} 2^{m-j} T[i+j -2] \\
	           &= T[i+m -1] - 2^m T[i-1] + 2H(T_{i-1})
\end{align*}

Pertanto $ H_p(T_i) $ può essere calcolato come

$$
H_p(T_i) = \big( T[i+m -1] - (2^m \mod p) T[i-1] + 2H_p(T_{i-1}) \big) \mod p
$$

Da notare che il tutto non viola il modella RAM perché il valore è compreso tra $ -p+1 $ e $ 2p-1 $ e $ 2^m \mod p $ può essere calcolato come

$$
2^m \mod p  \begin{cases}
2 &\text{ se } m = 1\\
2 (2^{m-1} \mod p) & \text{ se } m > 1
\end{cases}
$$

e quindi i calcoli vengono fatti in tempo costante.

\subsubsection{Algoritmo}\label{algortimo}

\begin{breakablealgorithm}
	\caption{RabinKarp: Algoritmo di RabinKarp}
	\begin{algorithmic}[1]
\Function{RabinKarp}{$P,T$}
    \State $z \gets 2$
    \For{$ j = 2 \text{ to } m$}
        \State $ z \gets (2z) \mod p$
    \EndFor
    \State // $ z = 2^m \mod p$
    \State $ x \gets P[1]$
    \For{$j = 2 \text{ to } m$}
        \State $ x \gets (2x + P[j]) \mod p$
    \EndFor
    \State // $x = H_p(P)$
    \State $y \gets T[1]$
    \For{$j = 2 \text{ to } m$}
        \State $y = (2y + T[j]) \mod p$
    \EndFor
    \State // $y = H_p(T_1)$
    \If{$x = y$}
        \State Segnala una possibile occorrenza in posizione 1
    \EndIf
    \For{$i = 2 \text{ to } n-m+1$}
        \State $y \gets (T[i+m-1] - zT[i-1] + 2y) \mod p$
        \State // $y = H_p(T_i)$
        \If{$x = y$}
            \State Segnala una possibile occorrenza in posizione $i$
        \EndIf
    \EndFor
\EndFunction
\end{algorithmic}
\end{breakablealgorithm}

\subsubsection{La scelta di \textit{p}}\label{la-scelta-di-p}

I risultati migliori si ottengono scegliendo come \emph{p} un numero
primo in un opportuno intervallo, in modo da minimizzare i falsi
positivi.

La scelta si basa su alcune proprietà dei numeri primi.

\paragraph{Numero di numeri primi minori di un certo \emph{n}}

Sia $\pi(n)$ il numero di primi minori o uguali a \emph{n}.

Chebyschev ha dimostrato che per ogni $n \geq 11$, si
ha che

$$
\frac{n}{\ln (n)} \leq \pi (n) \leq 1,26 \frac{n}{\ln (n)}
$$

\paragraph{Prodotto dei numeri primi}\label{prodotto-dei-numeri-primi}

Per ogni $n \geq  29$, il prodotto di tutti i numeri
primi minori o uguali ad \emph{n} è strettamente maggiore di
$2^{n}$.

$$
q_1 \cdot q_2 \cdot \ldots \cdot q_{\pi (n)} > 2^n
$$

Come conseguenza si ha che se $n \geq 29$, qualsiasi
numero \emph{x} minore o uguale di $2^n$ ha meno di $\pi(n)$
divisori primi distinti.

Questo si dimostra per assurdo, perché se \emph{x} ha $m \geq \pi(n)$ divisori primi distinti $p_1, \ldots , p_m$, si ha che

$$
q_1 \cdot \ldots \cdot q_k \leq p_1 \cdot \ldots \cdot p_m \leq x \leq 2^n
$$

Dove $q_1,\ldots , q_k$ sono i primi \emph{k} numeri primi. 
Il loro prodotto è strettamente maggiore di $2^n$ e questo è assurdo.

\paragraph{Teorema fondamentale per Rabin e Karp}\label{teorema-fondamentale-per-rabin-e-karp}

Siano \emph{P} e \emph{T} due stringhe di lunghezza \emph{m} e \emph{n}
tali che $mn \geq 29$ e sia \emph{N} un intero positivo
qualsiasi maggiore o uguale a \emph{mn}.

Se \emph{p} è scelto casualmente tra tutti i numeri primi minori o
uguali di \emph{N} la probabilità di una falsa occorrenza tra \emph{P} e
\emph{T} è minore di

$$
\frac{\pi (nm)}{\pi (N)}
$$

\subparagraph{Dimostrazione}\label{dimostrazione}

Sia \emph{R} l'insieme di tutti gli indici di tutte le posizioni in
\emph{T} in cui non c'è un'occorrenza del pattern \emph{P}, ossia
$H(T_i) \neq H(P)$.

Consideriamo il prodotto

$$
\Pi = \prod\limits_{i \in R} \big(\big|H(T_i) - H(P)\big|\big)
$$

Tale prodotto deve essere minore di $2^{mn}$ in quanto $\big|H(T_i) - H(P)\big| \leq 2^m$ per ogni
\emph{i}.

Per quanto precedentemente enunciato si ha che $ \Pi $ ha al più
$\pi(nm)$ divisori primi distinti.

Supponendo che ci sia una falsa occorrenza del pattern \emph{P} in
qualche posizione \emph{i} del testo \emph{T}, ovvero $H(T_i) \neq
H(P)$ e $H_p(T_i) = H_p(P)$.

Si ha che \emph{p} è un divisore della differenza $H(T_i) - H(P)$ e
quindi è divisore anche di $ \Pi $.

Siccome \emph{p} è stato scelto casualmente tra i primi $\pi(N)$
numeri primi, la probabilità di una falsa occorrenza è minore o uguale
di $\frac{\pi (nm)}{\pi (N)}$

Quindi se \emph{N} è sufficientemente la probabilità di un falso
positivo è bassa.

Tipicamente viene scelto $N = n^2m$ perché

$$
\frac{\pi (nm)}{\pi (n^2m)} \leq \frac{1,26 \frac{nm}{\ln (nm)}}{ \frac{n^2 m}{\ln (n^2 m)}} = 1,26 \frac{\ln (n^2 m) }{n \ln(nm)} = \frac{1,26}{n} \frac{ 2\ln (n) + \ln (m)}{\ln (n) + \ln (m)} \leq \frac{2,52}{n}
$$

% !TEX encoding = UTF-8
% !TEX program = pdflatex
% !TEX root = AALP.tex
% !TEX spellcheck = it-IT

% 29 Novembre 2016

%\section{OOP meets FP}

Il pattern viene utilizzato anche per la gestione delle funzioni.


Le funzioni vengono implementate come oggetti e per ogni arietà è definita una classe che le rappresenta. Non sono proprio classi, ma \texttt{trait}, una sorta di interfacce.

\begin{lstlisting}[language=Scala]
trait Function1[-A, +B] {
	def apply(x:A) : B
}
\end{lstlisting}

Nella definizione dei parametri di tipo sono presenti anche dei simboli \texttt{+} e \texttt{-}. Si tratta di due operatori che permetto di annotare la varianza dei tipi. Con questa configurazione si a che:

\begin{center}
\texttt{Fun[A',B']} $<:$ \texttt{Fun[A,B]} se $A <: A'$ e $B' <: B$
\end{center}

Ovvero il sub-typing per le funzioni segue la classica definizione.

Dato che le funzioni sono classi, potrei anche pensare di definire un array come una funzione.


\section{Static vs dynamic typing}

In Scala viene adottato un sistema di typing statico, ma la sua verbosità viene ridotta dalla type-inference effettuata dal compilatore.

Tuttavia si può anche introdurre il dynamic typing per alcune classi. Ciò viene fatto utilizzando il trait \texttt{Dynamic}. Così facendo, quando il compilatore deve valutare la chiamata su un riferimento \texttt{Dynamic}, prova a fare la verifica statica, ma se non ci riesce predispone un controllo a run-time della presenza.
Ad esempio la chiamata \texttt{ref.sel(arg)} viene riscritta in byte-code come \texttt{ref.applyDynamic("sel")(args)}.

\chapter{Java 8}

Con Java 8 sono state aggiunte in Java delle caratteristiche della programmazione funzionale, come la possibilità di definire delle funzioni anonime. 

Queste caratteristiche permettono di rinnovare il framework delle collezioni, ad esempio per supportare la parallelizzazione delle operazioni sulle collezioni.

Ad esempio per effettuare l'iterazione con il for-each si può utilizzare la valutazione \textbf{interna} della collezione.

\begin{lstlisting}[language=Java]
Collection<Person> people = ...
int maxAge = people.stream()
                          .filter(p -> p.getGender() == MALE)
                          .mapToInt(p -> p.getAge())
                          .max();
\end{lstlisting}

Da notare che così facendo non si perde efficienza, perché ad ogni chiamata non viene creata una nuova collezione, ma viene effettuata una valutazione lazy che può anche essere parallelizzata.

Per ottenere questo risultato è stato necessario aggiungere al linguaggio:

\begin{itemize}
	\item Lambda Expression
	\item Stream
	\item Method References
	\item Type inferece
	\item Default methods
	\item Functional Interfaces
\end{itemize}

Il tutto mantenendo la retro-compatibilità con le precedenti versioni.

\section{Lambda Expressions}

Non è stato possibile aggiungere un tipo freccia in Java perché altrimenti si sarebbe persa la retro-compatibilità.

Pertanto è stato necessario definire delle interfacce che hanno un solo metodo come

\begin{lstlisting}
interface Fun {double apply(int i);}
\end{lstlisting}

Tutte queste interfacce sono definite nel package \texttt{java.util.function} e prendono il nome di \textbf{functional interfaces}.
Da notare che anche le interfacce già esistenti, come \texttt{Runnable}, rientrano in questa categoria.

\begin{lstlisting}[language=Java]
public interface Function<T,R> {R apply(T t);}
public interface Predicate<T> {boolean test(T t);}
public interface Supplier<T> {T get();}
...
public interface Runnable {public void run();}
\end{lstlisting}

Si perde però il sub-typing diretto tra i tipi freccia, ma si può ottenere qualcosa di simile utilizzando le wild-card per le classi generiche.

Alcuni esempi:

\begin{lstlisting}[language=Java]
ToIntFunction<Integer> f1 = z -> z*10;
int x1 = f1.applyAsInt(18);

ToIntBiFunction<Integer, Integer> f2 = (x,y) -> x -y; // C'è un maggior lavoro di type-inference
int x2 = f2.applyAsInt(20,4);

BiPredicate<String, String> f3 = (s1,s2) -> s1.equals(s2);
boolean x3 = f3.test("pippo", "pluto");
\end{lstlisting}

Da notare che con questa definizione anche le interfacce dei vari action listener possono essere istanziate con delle lamba expression.

\begin{lstlisting}[language=Java, caption=Da notare che la variabile frame della lamba expression è libera e prende un valore dal contesto di invocazione del metodo.]
button.addActionListener(event -> JOptionPane.showMessageDialog(frame, "Hai cliccato"));

// mentre prima bisognava scrivere
button.addActionListener(new ActionListener() {
	public void actionPerformed(ActionEvent event) {
		JOptionPane.showMessageDialog(frame, "Hai cliccato");
	}
})
\end{lstlisting}

Con le lambda expression possono essere utilizzate delle variabili libere, che utilizzando i riferimenti del contesto di esecuzione per dare un valore ai riferimenti liberi (\textbf{Clojure}).

Lo stesso vale per Scala. Ad esempio sul REPL di Scala si possono fare le seguenti operazioni:

\begin{lstlisting}[language = Scala]
> (x : Int) => x + more
> Error: not found : value more

> val more = 1
more : Int = 1 
> val addMore = (x: Int) => x + more

// todo copiare anche l'esempio con var
\end{lstlisting}

Sia in Java che in Scala, le \textbf{clojure}, ovvero l'espressione e i riferimenti all'ambienti di esecuzione vengono tenuti assieme. In particolare i riferimenti non sono dereferenziati ma rimangono tali. Quindi se in una clojure vengono utilizzate delle \texttt{var} esterne 




%\appendix


%----------------------------------------------------------------------------------------
% BIBLIOGRAPHY
%----------------------------------------------------------------------------------------
%\bibliographystyle{unsrt}
%\bibliography{sample}
%----------------------------------------------------------------------------------------

\end{document}