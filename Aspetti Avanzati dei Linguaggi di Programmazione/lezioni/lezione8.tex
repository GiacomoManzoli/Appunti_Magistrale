% !TEX encoding = UTF-8
% !TEX program = pdflatex
% !TEX root = AALP.tex
% !TEX spellcheck = it-IT

% 27 Ottobre 2016
%\section{Estensioni del nostro linguaggio}
%\subsection{Record}
% \subsubsection{L'oggetto conto}

\section{Tipi variante}

Se i record possono essere visti come tipi congiunzione, che combinano più tipi, i tipi variante possono essere visti come disgiunzione.

Ad esempio possiamo pensare ad un contatto in una rubrica che può avere un indirizzo fisico o virtuale:

$$
< \text{fisico}: \underbrace{\{ \text{nome : String}, \text{indirizzo: String} \}}_{T_{fisico}} , \: \text{virtuale}: \underbrace{\{ \text{login : String}, \text{email : String} \}}_{T_{virt}} >= T_{ind}
$$

\noindent Un esempio di valore che ha questo tipo è:

$$
< \text{fisico} = \{ \text{nome = ``pippo''}, \text{indirizzo =  ``Via rosa''} \} >
$$

\noindent L'utilità di questi tipi si ha con l'operatore di pattern matching.
Ad esempio in Scala è possibile definire delle funzione generiche che lavorano sui tipi variante:

\begin{lstlisting}[language=Scala, caption=Utilizzo del pattern matching in Scala]
def getName(a : Tind) : String = a match{
	case <fisico = x> => x.nome
	case <virtuale = x> => x.login
	/** case <l = x> => ... da errore di compilazione! */
	/** allo stesso modo viene segnalato se il pattern non copre tutte le possibili etichette */
}

def getAll(a : List[Tind] ) : List[String] = {
	var l = new List[String]
	for (x <- a) l.add(getName(x))
}
\end{lstlisting}

\noindent Per inserire nel nostro linguaggio questi valori servono dei nuovi termini:

$$
M ::= < l = M > \vbar M \text{ match } \{\text{case }l_i = x_i => M_i \:^{i = 1\ldots n}\} \vbar \ldots
$$

\noindent Da notare che la $x_i$ che viene utilizzata nel \text{match} lega le eventuali occorrenze all'interno di $M_i$.

$$
<l_1 = 3> \text{ match } \{\text{case } l_1 = x => x+1, \ldots \} \rightarrow 3+1
$$

\noindent Più formalmente:

\begin{itemize}
	\item $fv(<l =M>) = fv(M)$
	\item $fv( M \text{ match } \{\text{case }l_i = x_i => M_i \:^{i = 1\ldots n}\}) = fv(M) \cup \bigcup_{i = 1 \ldots n}\bigg( fv(M_i) \setminus \{x_i\} \bigg) $
\end{itemize}


\noindent Servono inoltre dei nuovi valori finali e dei tipi:

$$
v ::= <l = v> \vbar \ldots \qquad T ::= < l_i : T_i \:^{i = 1 \ldots n}>
$$

\noindent La semantica operazionale viene espressa con 3 nuove regole:

\begin{itemize}
	\item Riduzione del termine interno
	\begin{prooftree}
		\AxiomC{$M \rightarrow M'$}
		\LL{Variant}
		\UnaryInfC{$<l = M> \rightarrow <l = M'>$}
	\end{prooftree}
	\item Avanzamento in un match:
	\begin{prooftree}
		\AxiomC{$M \rightarrow M'$}
		\LL{Red-Match}
		\UnaryInfC{$M \text{ match } \{\text{case } l_i = x_i => M_i \:^{i = 1 \ldots n}\} \to M' \text{ match } \{\text{case } l_i = x_i => M_i \:^{i = 1 \ldots n}\}$}
	\end{prooftree}
	\item Assioma per il match effettivo
	\begin{prooftree}
		\AxiomC{$j \in \{1 \ldots n\}$}
		\LL{Match}
		\UnaryInfC{$<l_j = v> \text{ match } \{\text{case } l_i = x_i => M_i \:^{i = 1 \ldots n}\} \rightarrow M_j\{x_j = v\}$}
	\end{prooftree}
\end{itemize}

\noindent Servono inoltre delle regole di tipo, una per l'invariante e l'altra per il match.

\begin{prooftree}
	\AxiomC{$ \Gamma \vdash M: T_j $}
	\AxiomC{$ j \in \{1 \ldots n \} $}
	\LL{Type-Variant}
	\BinaryInfC{$\Gamma \vdash <l_j = M> : <l_i : T_i \:^{i=1\ldots n}>$}
\end{prooftree}

\noindent Con questa regola posso assegnare ad un valore infiniti tipi, l'importante è che in questi infiniti tipi ci sia l'etichetta $l_j$ in modo da avere la garanzia di riuscire ad effettuare il pattern matching.

\begin{prooftree}
	\AxiomC{$\Gamma \vdash M : < l_i : T_i  \:^{i=1\ldots n}>$}
	\AxiomC{$\Gamma, x_i : T_i \vdash M_i : T \:\: \forall \: i = 1 \ldots n $}
	\LL{Type-Match}
	\BinaryInfC{$\Gamma \vdash M \text{ match }\{\text{case }l_i = x_i => M_i \:^{i=1\ldots n}\} : T$}
\end{prooftree}

\noindent Perché un match sia ben tipato è necessario che il termine $M$ sia un valore di tipo variante e che tutti i ``rami'' del match siano termini con lo stesso tipo, aggiungendo anche al contesto la sostituzione che viene effettuata quando viene applicato il match.

Se nel pattern matching ho come premessa della regola $\Gamma \vdash M : < l_i : T_i  \:^{i=1\ldots m}>$ con $m \geq n$ c'è un problema perché possono capitare delle etichette che il costrutto \text{match} non riesce a gestire. Se invece $m \leq n$ non ci sono problemi. Ma anche in questo caso, dato che ho a disposizione infiniti tipi, conviene forzare $m = n$.

Anche se non sembra questi tipi sono presenti nella maggior parte dei linguaggi main stream, solo che questi vengono nascosti da dello zucchero sintattico. Ad esempio le liste possono essere viste come un tipo variante in quanto sono o una lista vuota o la concatenazione di un elemento e un'altra lista.

$$
\text{List} = < \text{nil : Unit}, \text{cons} : (\Nat * \text{List}) >
$$

\noindent Si tratta di un tipo ricorsivo che non è supportato nella nostra grammatica, ma in altre grammatiche è possibile gestirlo.

I valori per questo tipo sono:

$$
<\text{nil } = \text{unit}>  \quad <\text{cons} = (5, <\text{nil } = \text{unit}>)>
$$

\noindent Un altro caso d'uso dei valori variante è l'analisi delle dereferenziazioni dei valori null.
Ad esempio in Java possiamo definire una variabile e assegnarle null:

\begin{lstlisting}[language=Java]
C c = null;

C find(List<C> l, C a) { ... } // se non trova ritorna null

C c = find(list, s);
// c può essere null, quindi devo controllare
// se non controllo potrei finire in un NullPointerException, anche perché il compilatore non controlla questo tipo di eccezioni
if (c != null) print(c.info());
else print("non trovato");
\end{lstlisting}

\noindent Questo approccio non è dei migliori, perché dovrebbe utilizzare le eccezioni personalizzate, ma si è visto che nessuno le usa. Con C\# si sono invece inventati i tipi \texttt{Nullable} che possono avere anche come valore \texttt{Null}, anche per i tipi primitivi.

Un'altra idea è stata quella di introdurre i tipi \texttt{!}: una classe di tipo \texttt{C!} non può assumere come valore null, in modo da sfruttare di più l'analisi statica.

Linguaggio che vai, soluzione che trovi. Altri linguaggi utilizzano i così detti \textbf{null objects}:

\begin{lstlisting}[language=Java]
class C {
	...
	String info() {...}
	...
}
class NullC extends C {
	...
	String info() {}
}
\end{lstlisting}

\noindent Così facendo c'è un metodo definito da chiamare anche sul valore null, in modo da evitare la NullPointerExcpetion.

Scala e Java8 (e Swift) implementano un'ulteriore versione, gli \textbf{option type}, ovvero un tipo variante che prevede due possibilità: c'è l'oggetto oppure non c'è.
La definizione di questo tipo è la seguente:

$$
Option[C] = < \text{none} : \text{Unit}, \text{some} : C >
$$

\noindent Il codice di prima può essere quindi riscritto come

\begin{lstlisting}[language=Scala]
...
def find(l : List[C], s : String) : Option[C] = {
	for(x <- l) 
		if (x.info() == s) return Some(x)
	return None 
}
...
find(l, "pippo") match {
	case Some(x) => print(x.info())
	case None => print("non trovato")
}
\end{lstlisting}

\noindent Si può notare come questa versione del codice è più espressiva e si riesce subito a capire che la funzione \texttt{find} può non trovare l'elemento cercato.


