% !TEX encoding = UTF-8
% !TEX program = pdflatex
% !TEX root = AALP.tex
% !TEX spellcheck = it-IT

% 8 Novembre 2016
%\section{Estensioni del nostro linguaggio}
%\subsection{Le eccezioni}

\section{Sub-typing}

Partiamo dalla versione del linguaggio che ha i record.

$$
(\fn x : \{ l : \Nat \} . x.l + 2 ) \: \{l=0, l' = 1\}
$$

\noindent Al momento il programma sopra riportato non tipa.

$$
(\fn x : \{ l : \Nat, l': \Nat \} . M ) \: \{ l = 5 \}
$$

\noindent Anche in questo caso il tipo è diverso e il programma non compila.

Tuttavia sembra un po' una forzatura, specialmente riguardo al primo caso, dato che per il parametro formale è richiesta solamente l'etichetta $l$.
Il secondo caso è un po' più delicato, perché il valore fornito come parametro attuale funzionerebbe solamente in alcuni casi, in base alla struttura di $M$.

In generale è possibile far tipare, senza incorrere in errori, il passaggio di parametri quando il parametro attuale ha più informazioni (più etichette) rispetto al parametro formale.

Quindi se $S <: T$, ovvero $S$ è un sotto tipo di $T$, posso usare un valore di tipo $S$ dove ne è richiesto uno di tipo $T$ se riesco a provare vera la regola:

\begin{prooftree}
\AxiomC{$\Gamma \vdash M : S$}
\AxiomC{$S <: T $}
\LeftLabel{\myrule{Subsumption}}
\BinaryInfC{$\Gamma \vdash M : T$}
\end{prooftree}

\noindent Esempio:

\begin{prooftree}

\AxiomC{$\checkmark$}
\UnaryInfC{$x:\{ l:\Nat\}$}
\LeftLabel{\myrule{Select}}
\UnaryInfC{$x:\{l : \Nat \} \vdash x.l: T$}
\AxiomC{$ \checkmark$}
\UnaryInfC{$x: \{l : \Nat\}  \vdash 2 : \Nat$}
\LeftLabel{\myrule{Sum}}
\BinaryInfC{$x : \{l : \Nat\} \vdash x.l+2 : T$}
\LeftLabel{\myrule{Fun}}	
\UnaryInfC{$ \emptyset \vdash \fn x : \{ l : \Nat \} . x.l + 2 : \{l : \Nat\} \to T$}


\AxiomC{$ \checkmark$}
\UnaryInfC{$\emptyset \vdash 0 : \Nat$}
\AxiomC{$ \checkmark$}
\UnaryInfC{$\emptyset \vdash 1 : \Nat$}
\LeftLabel{\myrule{Record}}
\BinaryInfC{$\emptyset \vdash \{l=0, l'=1\} : \{l: \Nat, l':\Nat \}$}
\AxiomC{$ $}
\UnaryInfC{$\{l:\Nat , l' : \Nat\}$}

\LeftLabel{\myrule{Sumbumption}}
\BinaryInfC{$ \emptyset \vdash \{l=0, l'=1\}: \{l:\Nat\}$}
\LeftLabel{\myrule{App}}
\BinaryInfC{$\emptyset \vdash (\fn x : \{ l : \Nat \} . x.l + 2 ) \: \{l=0, l' = 1\}: T $}
\end{prooftree}

\noindent Serve ora una regola per decidere quando vale o meno la relazione di sub-typing.

Un tipo è sotto-tipo di se stesso.
\begin{prooftree}
	\AxiomC{ }
	\UnaryInfC{$T <: T$}
\end{prooftree}

\noindent La proprietà di sub-typing è transitiva.
\begin{prooftree}
	\AxiomC{$T_1 <: T_2$}
	\AxiomC{$T_2 <: T_3$}
	\BinaryInfC{$T_1 <: T_3$}
\end{prooftree}

\noindent Sub-typing in larghezza, il sottotipo ha più etichette, ma contiene anche tutte quelle del tipo di partenza.

\begin{prooftree}
	\AxiomC{}
	\LeftLabel{\myrule{Sub-Width}}
	\UnaryInfC{$ \{l_i : T_i \:^{i = 1\ldots n+k} \} <: \{l_i : T_i \:^{i = 1\ldots n} \}$}
\end{prooftree}

\noindent C'è anche il sub-typing in profondità, che permette di cambiare i tipi anche all'interno dei campi dati.

\begin{prooftree}
	\AxiomC{$ S_i <: T_i \quad i = 1 \ldots n $}
	\LeftLabel{\myrule{Sub-Depth}}
	\UnaryInfC{$ \{ l_i : S_i \:^{i = 1 \ldots n} \} <: \{ l_i : T_i \:^{i = 1 \ldots n} \}$}
\end{prooftree}

\noindent Possono però capitare delle situazioni in cui c'è bisogno di effettuare una permutazione delle etichette, ad esempio per far valere la relazione $\{a : \Bool, b: \Bool \} <: \{ b: \Bool\}$
\begin{prooftree}
	\AxiomC{$ \{ k_j : S_j \:^{i = j \ldots n} \} \text{ è una permutazione di }\{ l_i : T_i \:^{i = 1 \ldots n} \}  $}
	\LeftLabel{\myrule{Permute}}
	\UnaryInfC{$ \{ k_j : S_j \:^{i = j \ldots n} \} <: \{ l_i : T_i \:^{i = 1 \ldots n} \} $}
\end{prooftree}

\noindent Tutte le regole possono essere riunite in un'unica regola, ma noi le teniamo separate perché alla fine ognuna giustifica il sub-typing con una motivazione diversa.

Da notare che in Java non è presenta la regola \myrule{Sub-Depth}, ma ce n'è una molto più forte che rende vero se i valori di ritorno sono tra di loro sotto-tipi.

Nel nostro linguaggio ci sono anche dei valori di tipo funzione, e anche li si può definire una relazione di sub-typing.

\begin{prooftree}
	\AxiomC{$ T_1 <: S_1 $}
	\AxiomC{$ S_2 <: T_2$}
	\BinaryInfC{$S_1 \to S_2 <: T_1 \to T_2 $}
\end{prooftree}

\noindent Se una funzione accetta argomenti di tipo $S_1$, vuol dire che può accettare argomenti di tipo $T_1$ se $T_1 <: S_1$ ($T_1$ è più specifico). In modo simile, una funzione che ritorna un valore di tipo $S_2$, può essere utilizzata nel contesto in cui è utilizzata una funzione che ritorna un $T_2$ più generico di $S_2$ ($S_2 <: T_2$). Ovvero il sub-typing delle funzioni è \textbf{contro variante} sull'input e \textbf{co-variante} sull'output.

\begin{prooftree}
	\AxiomC{$T_1 <: S_1$}
	\AxiomC{$T_2 <: S_2$}
	\BinaryInfC{$ S_1 \to S_2 <: T_1 \to T_2$}
	\LeftLabel{\myrule{Sub-Depth}}
	\UnaryInfC{$ \{m : S_1 \to S_2\} <: \{ m : T_1 \to T_2 \} $}
\end{prooftree}

\noindent Nel nostro caso il sub-typing sui record e funzioni è più permissivo rispetto a quello del Java, perché in Java c'è sempre in input il parametro implicito \texttt{this} che crea dei problemi (semplificazione dell'overloading resolution).

\begin{prooftree}
	
	\AxiomC{$\vdots$}
	\LeftLabel{\myrule{Fun}}
	\UnaryInfC{$ \emptyset \vdash \fn x.\{l:\Nat\} . x.l+2 : \{l:\Nat\} \to \Nat $}
	
		\AxiomC{$\checkmark$}
		\UnaryInfC{${l:\Nat, l':\Nat} <: \{l:\Nat\}$}
		
		\AxiomC{$\checkmark$}
		\UnaryInfC{$ \Nat <: \Nat $}
		\LeftLabel{\myrule{Subsumption}}
		\BinaryInfC{$\{ l: \Nat \} \to \Nat <: \{ l : \Nat, l':\Nat \}$}
	
	\LeftLabel{\myrule{Subsumption}}	
	\BinaryInfC{$ \emptyset \vdash \fn x : \{ l : \Nat \} . x.l + 2 : \{l : \Nat , l':\Nat \} \to \Nat$}
	
	
	\AxiomC{$\vdots$}
	\LeftLabel{\myrule{Record}}
	\UnaryInfC{$ \emptyset \vdash \{l=0, l'=1\}: \{l:\Nat, l': \Nat\}$}
	\LeftLabel{\myrule{App}}
	\BinaryInfC{$\emptyset \vdash (\fn x : \{ l : \Nat \} . x.l + 2 ) \: \{l=0, l' = 1\}: \Nat $}
\end{prooftree}

\noindent Con queste nuove regole di typing si perde l'unicità dell'albero di derivazione del giudizio di tipo, perché la regola \myrule{Subsumption} può essere applicata in qualsiasi momento.

A causa di queste nuove regole è necessario andare ad estendere il \textbf{Lemma di inversione}:

\begin{enumerate}
	\item Se $\Gamma \vdash x : T $, allora $\exists S. x: S \in \Gamma \text{ e } S<: T$. ($S$ può anche essere uguale a $T$).
	\item Se $\Gamma \vdash \fn x.S_1.M : T$, allora $\exists T_1, T_2 . T = T_1 \to T_2, T_1 <: S_1, \Gamma, x:S_1 \vdash M : T_2$.
	\item (Continua sulle note)
\end{enumerate}

\noindent Ci sono poi dei \textbf{Lemmi di Inversione del Sub-Typing}:

\begin{enumerate}
	\item Se $S <: T_1 \to T_2$, allora $\exists S_1, S_2. S = S_1 \to S_2, T_1 <: S_1, S_2 <: T_2 $.
	\item Se $S <: \{l_i : T_i \:^{i = 1 \ldots n} \}$, allora $S = \{ k_j : S_j \:^{j = 1 \ldots n} \}$ con gli $S_j$ corretti (vedi note).
\end{enumerate}

\noindent Anche il \textbf{Lemma delle forme canoniche} deve essere aggiustato:

\begin{enumerate}
	\item Se $v$ è un valore di tipo $T_1 \to T_2$, allora $v = \fn x : S_1.M$
	\item \textit{c'è anche una cosa simile per i record}
\end{enumerate}

\noindent Con queste modifiche ai lemmi, continuano a valere sia progressione che subject-reduction, e quindi continua a valere il teorema di safety.

Una volta introdotto il sub-typing risulta utile definire un tipo generico che funziona da super-tipo per tutti gli altri tipi, sia del linguaggio che definiti dagli utenti.
Infatti, in Scala c'è il tipo \texttt{Any} che è sopra-tipo di \texttt{AnyRef}, il sopra-tipo di tutti gli oggetti, e \texttt{AnyVal}, il sopra-tipo di tutti i valori.

In modo simile, può tornare utile avere un sotto-tipo \texttt{Bot} che funziona da sotto-tipo per tutti gli altri tipi. Questo tipo può tornare utile per dare un tipo al termine $\throw{v}: \text{Bot}$. In Scala esiste questo tipo e si chiama \texttt{Nothing} e c'è un metodo predefinito, \texttt{def error(msg : String): Nothing}, che viene utilizzato per lanciare le eccezioni run-time. 
Da notare che non possono esserci valori di tipo \texttt{Nothing}, perché sennò non sarebbe un tipo bottom. Serve solo per far tipare le espressioni.



