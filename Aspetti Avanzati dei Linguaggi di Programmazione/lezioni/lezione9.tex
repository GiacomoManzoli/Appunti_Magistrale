% !TEX encoding = UTF-8
% !TEX program = pdflatex
% !TEX root = AALP.tex
% !TEX spellcheck = it-IT

% 3 Novembre 2016
%\section{Estensioni del nostro linguaggio}
%\subsection{Tipi variante}

\section{Le eccezioni}

$$
M ::= \text{throw} \: M \vbar \text{try} \: M \: \text{catch} \: M \vbar \ldots
\qquad
T ::= \TExc \vbar \ldots
$$

Il termine $\throw{M}$ indica il sollevamento di un'eccezione con delle informazioni aggiuntive, rappresentate dal termine $M$, che vengono utilizzate dal gestore dell'eccezione.

La cattura di un'eccezione viene fatta dal \text{catch} $N$, il quale sarà una funzione dato che dovrà ricevere le informazioni relative all'eccezione che si è verificata.

\noindent Alcuni esempi di termini sono:

\begin{itemize}
	\item $M = \fn x. \IF{pari(x)}{x/2}{\throw{x}}$.
	\item $\trycatch{(M \: 3)}{\fn y.(y+y)}$: il blocco catch deve essere una funzione, perché in qualche modo deve ricevere l'eccezione.
	\item $\trycatch{(\fn y.y-z \: (M-5)))}{\fn z.print(z)}$: in questo caso è necessario fare un salto e scartare alcuni valori per arrivare al blocco catch.
	\item $\trycatch{(\fn y.y-2 \: (\trycatch{(M 5)}{\fn w.w+w}))}{\fn z.print(z)}$: nel caso di handler innestati, l'eccezione deve essere gestita da quella più vicina.
\end{itemize}

\noindent Abbiamo quindi un po' di regole di semantica da definire.

\begin{itemize}
	\item Avanzamento del termine dentro il \text{try}:
	\begin{prooftree}
	\AxiomC{$M \to M'$}
	\LeftLabel{\myrule{Try}}
	\UnaryInfC{$\trycatch{M}{N} \to \trycatch{M'}{N}$}
	\end{prooftree}
	
	\item Assioma per l'handling dell'eccezione:
	\begin{prooftree}
	\AxiomC{$ $}
	\LeftLabel{\myrule{Try-Handle}}
	\UnaryInfC{$\trycatch{\throw{v}}{M} \to (M \: v)$}
	\end{prooftree}
	
	\item Assioma per l'avanzamento nel caso in cui non si sono verificate eccezioni all'interno del blocco \text{try}:
	\begin{prooftree}
	\AxiomC{$ $}
	\LeftLabel{\myrule{Try-Val}}
	\UnaryInfC{$\trycatch{v}{M} \to v$}
	\end{prooftree}
	
	\item Regole per l'avanzamento del \text{throw}:
	\begin{center}
		\begin{bprooftree}
		\AxiomC{$M \to M'$}
		\LeftLabel{\myrule{Raise-1}}
		\UnaryInfC{$\throw{M} \to \throw{M'}$}
		\end{bprooftree}
		$\quad$
		\begin{bprooftree}
			\AxiomC{$M \to M'$}
			\LeftLabel{\myrule{Raise-2}}
			\UnaryInfC{$\throw{\throw{v}} \to \throw{v}$}
		\end{bprooftree}
	\end{center}

	\item Regole per la sollevazione delle eccezioni durante l'applicazione di una funzione:
	\begin{center}
		\begin{bprooftree}
			\AxiomC{$ $}
			\LeftLabel{\myrule{Raise-App-1}}
			\UnaryInfC{$ \throw{v} \: M \to \throw{v}$ }
		\end{bprooftree}
		$\quad$
		\begin{bprooftree}
			\AxiomC{$ $}
			\LeftLabel{\myrule{Raise-App-2}}
			\UnaryInfC{$ v_1 \: \throw{v_2} \to \throw{v_2}$ }
		\end{bprooftree}
	\end{center}

	\item Regola per l'eccezione nella guardia dell'\text{if}:
	\begin{prooftree}
		\AxiomC{$ $}
		\LeftLabel{\myrule{Raise-If}}
		\UnaryInfC{$ \IF{\throw{v}}{M_1}{M_2} \to \throw{v}$ }
	\end{prooftree}
	
	\item Regole per la propagazione nella somma (ci sono anche quelle per la sottrazione):
	\begin{center}
		\begin{prooftree}
			\AxiomC{$ $}
			\LeftLabel{\myrule{Raise-Sum-1}}
			\UnaryInfC{$ \throw{v} + M \to \throw{v}$ }
		\end{prooftree}
	
		\begin{prooftree}
			\AxiomC{$ $}
			\LeftLabel{\myrule{Raise-Sum-2}}
			\UnaryInfC{$ v_1 +\throw{v_2} \to \throw{v_2}$ }
		\end{prooftree}
	\end{center}

\end{itemize}

\noindent Con queste nuove regole l'avanzamento del programma rimane deterministico, perché $\throw{v}$ è un termine in forma normale (non riducibile) e non un valore.

Pertanto in alcune situazioni potenzialmente ambigue, come:

$$
\fn x:\Nat.0 \: \throw{5}
$$

\noindent In questo caso la decisione della regola da applicare rimane deterministica, perché $\throw{5}$ non è un valore e quindi sono costretto ad applicare \myrule{Raise-App-2} e non posso applicare \myrule{Beta}. 
Se invece il termine $\throw{5}$ fosse anche un valore ci sarebbe l'ambiguità.

\subsection{Regole per il sistema di typing}

\begin{prooftree}
	\AxiomC{$ \Gamma \vdash M : \TExc $}
	\LL{Raise}
	\UnaryInfC{$ \Gamma \vdash \throw{M} : T$}
\end{prooftree}

\begin{prooftree}
	\AxiomC{$ \Gamma \vdash M : T $}
	\AxiomC{$ \Gamma \vdash N : \TExc \to T$}
	\LL{Try}
	\BinaryInfC{$ \Gamma \vdash \trycatch{M}{N} : T$}
\end{prooftree}

\noindent Da notare che al $\throw{}$devo poter dare qualsiasi tipo, altrimenti non avrei possibilità di inserire il termine in ovunque, perdendo così l'unicità di tipo.

Questa è la strategia usata da Java, dato che nella compilazione dell'\text{if} richiede che entrambi i rami abbiano lo stesso tipo, ma il codice sotto riportato compila correttamente, anche se il ramo \text{else} non ha tipo \texttt{C}.

\begin{lstlisting}[language=Java]
C foo() throws MyException{
	if (...){
		return new C();
	} else {
		throw new MyException();
	}
}
\end{lstlisting}

\noindent Se invece l'eccezione fosse di tipo run-time, ovvero una di quelle che non viene controllata dal compilatore, quest'ultimo compila qualsiasi cosa, senza controllare che il metodo dichiari nel suo prototipo che possono essere sollevate delle eccezioni.

Un'ultima cosa strana è il tipo $\TExc$, che in C++ corrisponde agli interi, in Java è una qualsiasi sotto-classe di \texttt{Throwable} e in ML viene implementato come un tipo variante.

\subsubsection{Modifiche ai teoremi}

\`E necessario modificare il teorema di \textbf{Subject Reduction} aggiungendo i nuovi casi induttivi relativi alle nuove regole.

Il teorema di \textbf{Progressione} deve essere modificato perché adesso abbiamo che $\throw{5}$ è un termine chiuso, ben tipato, che non fa nessun passo e non è un valore.
\`E quindi necessario estendere il teorema, aggiungendo anche la possibilità che $M$ sia un $\throw{}$:

\begin{center}
	\textit{Sia $M$ un termine chiuso, ben tipato ed in forma normale ($M \not\to$). Allora $M$ è un valore oppure $M = \throw{v}$ per un qualche valore $v$.}
\end{center}

\noindent Anche il teorema di Safety cambia e viene aggiunto il caso in cui $M\to^* \throw{v}$:

\begin{center}
	\textit{Sia $M$ un termine chiuso e ben tipato. Allora $M$ non evolve ad un termine stuck, ma $\exists v$ tale che $M \to^* v$ oppure $M \to^* \throw{v}$. } 
\end{center}

\noindent Da notare che quando un programma si ferma su un termine $\throw{}$ vuol dire che si è verifica un'eccezione non gestita.

Questa modifica al teorema di Safety è necessaria perché, anche se impongo che tutti i programmi siano chiusi dentro un grande \text{try-catch}, il vecchio teorema di Safety non funziona in quanto può esserci una \text{throw} dentro il \text{catch}.









