\section{Appello 2016-07-18}

\todo[inline]{La soluzione non è del tutto corretta}

\subsection{Esercizio 1}

Dimostrare il teorema di Rice.

\subsubsection{Soluzione}

Già fatto in altri appelli.

\subsection{Esercizio 2}

Data una funzione $f : \mathbb{N} \rightarrow \mathbb{N}$ si definisce il predicato $P(x,y) \equiv \text{``}f(x) = y\text{''}$, ovvero $P(x,y)$ è vero se e solo se $x \in dom(f)$ e $f(x) = y$. Dimostrare che la funzione $f$ è calcolabile se e solo se $P(x,y)$ è semi-decidibile.

\subsubsection{Soluzione}

Se $f$ è calcolabile c'è un programma $e$ che la calcola.

$$
SC_P(x,y) = \begin{cases}
1 &f(x) = y \\
\uparrow
\end{cases}  = \mathbb{1}\bigg( \mu w. S(e,x,y,w) \bigg)
$$

$SC_P$ è definita in modo calcolabile ed è corretta.

Se $P$ è semi-decidibile, $SC_P$ è calcolabile.

$$
f(x) = \mu w. SC_P(x,w)
$$

$f$ è calcolabile. C'è qualche problema se la $f$ non è iniettiva, ma se non ricordo male noi assumiamo che tutte le funzioni lo siano.

\paragraph{Possibile errore}

Sulla definizione di $f$ può esserci un'errore. Se per un $w$ non $SC_P$ è $\uparrow$, $f$ risulta indefinita su $x$ anche quando non lo è. 

Forse così va meglio ($e$ è il programma che calcola $SC_P$):

$$
f(x) = \bigg(\mu w . S\Big(e, x, (w)_1, (w)_2\Big)\bigg)_1
$$

perché vengono fatti un po' di passi alla volta.

\subsection{Esercizio 3}

$$
A = \{ x | x \in W_x \wedge \phi_x(x) = x^2 \}
$$

\subsubsection{Soluzione}

$A$ RE.

$$
SC_A(x) = \mathbb{1} \Bigg( \mu w. \bigg( S\Big(x,x,(w)_1, (w)_2\Big) \wedge (w)_1 = x^2 \bigg) \Bigg)
$$

$A$ non ricorsivo, $K \leq_m A$.

$$
g(x,y) = \begin{cases}
y^2 & x \in K \\
\uparrow
\end{cases} = y^2 \cdot SC_K(x)
$$

SMN.

\begin{itemize}
	\item $x \in K$: $\phi_{f(x)} = y^2 \forall y$. $\phi_{f(x)}(f(x)) = f(x)^2 \rightarrow f(x) \in A$.
	\item $x \notin K$: $\phi_{f(x)}(y) = \uparrow \forall y$.  $\phi_{f(x)}(f(x)) = \uparrow \rightarrow f(x) \notin A$.
\end{itemize}

$A$ è RE, non ricorsivo, $\overline{A}$ è quindi non-RE.

\paragraph{Note}
Forse da qualche parte nel pdf c'è una soluzione migliore di questo esercizio.

\subsection{Esercizio 4}

$$
B = \{ x | \forall y \in W_x, \exists z \in W_x. (y < z) \wedge f(y) < f(z) \}
$$

\subsubsection{Soluzione}

$B$ è saturo, contiene tutte le funzioni strettamente crescenti.

$id, \emptyset \in \beta$, la funzione gradino e quella definita in un punto solo $\in \overline{\beta}$.

$$
f(x) = \begin{cases}
0 & x = 0 \\
1 & \text{altrimenti}
\end{cases} = sg(x)
$$

$$
\vartheta(x) = \begin{cases}
0 & x= 0 \\
1 & x = 1 \\
\uparrow 
\end{cases}
$$

$f \notin \beta$, $\vartheta \in \beta$, $\vartheta \subseteq f$, $B$ è non-RE per Rice-Shapiro.

$$
g(x) = \begin{cases}
0 & x= 0 \\
\uparrow & \text{altrimenti}
\end{cases}
$$


$id \notin \overline{\beta}$, $g \in \overline{\beta}$, $g \subseteq id$, $\overline{B}$ è non-RE per Rice Shapiro.

\subsection{Esercizio 5}

Dimostrare che non è saturo

$$
C = \{x | x \in E_x \}
$$

\subsubsection{Soluzione}

$$
g(x,y) = \begin{cases}
x & x= y \\
\uparrow
\end{cases} = x \cdot \mathbb{1}\bigg( \mu w . |x-y| \bigg)
$$

SMN + Secondo teorema

$$
\phi_e(y) = \phi_{f(e)}(y) = g(e,y)
$$

e

$$
E_e = E_{f(e)} = \{ e\}
$$

Sia $e'$ tale che $\phi_e =\phi_{e'}$, $E_{e'} = \{e\}$, $e' \notin C$, $e \in C$, l'insieme non è saturo.