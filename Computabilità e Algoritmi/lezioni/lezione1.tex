% !TEX encoding = UTF-8
% !TEX TS-program = pdflatex
% !TEX root = computabilità e algoritmi.tex
% !TEX spellcheck = it-IT
\chapter{Funzioni calcolabili e Modelli di calcolo}
\section{Introduzione}\label{lezione-1---computabilituxe0-e-algoritmi}

Ci sono dei problemi che non possono essere risolti in modo algoritmico,
come la terminazione o la prova di correttezza di un programma, lo studio di questi problemi prende il nome di teoria della computabilità.

In questa teoria non viene preso in considerazione il consumo di
risorse in modo che le dimostrazioni effettuate siano indipendenti dal
modello di calcolo adottato.

Notoriamente, i problemi appartengono a varie classi di difficoltà:

\begin{itemize}
\item
  \textbf{P}: problemi che possono essere risolti da un algoritmo in
  tempo polinomiale
\item
  \textbf{NP}: problemi che possono essere risolti in tempo polinomiale
  ma in modo non deterministico
\item
  \textbf{EXP}: problemi che possono essere risolti da un algoritmo in
  tempo esponenziale
\end{itemize}

\subsection{L'informatica e la computabilità}\label{linformatica-e-la-computabilituxe0}

\emph{Computer science is no more about computers tha astronomy is about
telescopes. Dijkstra}.

L'idea dell'informatica nasce dalla logica, ricercando un procedimento
generale (macchina) su base combinatoria per trovare tutte le verità.

Libro: \emph{Nigel Cutland ``Computability. An Introduction to Recursive
Function Theory'' Cambridge University Press}.
