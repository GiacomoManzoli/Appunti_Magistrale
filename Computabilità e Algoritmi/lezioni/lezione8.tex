% !TEX encoding = UTF-8
% !TEX TS-program = pdflatex
% !TEX root = computabilità e algoritmi.tex
% !TEX spellcheck = it-IT

\section{Predicati decidibili}\label{predicati-decidibili}

Nel mondo matematico spesso ci si chiede se valgono determinate
proprietà come:

$$ div(n,m) = \exists k : n \cdot k =m$$

La stessa cosa può essere vista come una relazione sui numeri naturali
$div \subseteq \mathbb{N} \times \mathbb{N}$. Tipicamente questo prende il nome di
\textbf{problema} o \textbf{predicato} e viene formalmente indicato con

$$Q(x_1, \ldots , x_k)$$

$$Q : \mathbb{N}^k \rightarrow \{vero,falso\} \: \text{oppure} \: Q \subseteq \mathbb{N}^k $$

Una volta formalizzato il predicato, ci si chiede se questo predicato è
\textbf{calcolabile} o \textbf{decidibile}

Un predicato $Q \subseteq \mathbb{N}^k$ si dice \textbf{decidibile} quando
$\mathcal{X}_A : \mathbb{N}^k \rightarrow \mathbb{N}$ è calcolabile, ovvero 

$$\mathcal{X}(\vec{x}) = \begin{cases}
1, \:& \text{ se } Q(\vec{x}) \\
0, \:& \text{altrimenti}
\end{cases} $$

Un predicato è decidibile quando la sua funzione $\mathcal{X}_Q$ è totale.

\subsection{Esempi di predicati decidibili}\label{esempi-di-predicati-decidibili}

\subsubsection{Uguaglianza}\label{uguaglianza}

Ad esempio $Q_1(x,y) = ``x=y''$ (uguaglianza) viene calcolato da

$$\mathcal{X}_{Q_1}(x,y) = \begin{cases}
1, \:& \text{se}\: x=y\\
0, \:& \text{altrimenti}
\end{cases}$$ 

Risulta triviale trovare un programma che calcola questa funzione

\begin{lstlisting}[language=URM]
J(1,2,SI)
Z(1)
J(1,1,FINE)
Z(1) #SI
S(1)
#FINE
\end{lstlisting}

\subsubsection{Pari e dispari}\label{pari-e-dispari}

Altro esempio $Q_2(x) = ``x \:\text{è pari}''$

\begin{lstlisting}[language=URM]
/*
    1: x
    2: 0, lo incremento per vedere se x è pari o meno
    3: 0, lo uso per tenere il codice compatto
*/
J(1,2,SI) #PARI?
S(2)
J(1,2,NO) #DISPARI?
S(2)
J(1,1,PARI?)
S(3) #SI
T(3,1) #NO
\end{lstlisting}

\subsubsection{Esercizio - Minore e uguale}\label{esercizio---minore-e-uguale}

$Q_3(x,y) = x \leq y$

\todo[inline]{todo}

\subsubsection{Esercizio - Divisibile}\label{esercizio---divisibile}

$Q_4(x,y) = div(x,y)$

\todo[inline]{todo}
\section{Computabilità su altri domini}\label{computabilituxe0-su-altri-domini}

Ci sono tanti modelli di calcolo che funzionano su svariati domini e noi
ci siamo concentrati solamente su quelli che utilizzano i numeri. 
Può però essere interessante vedere come cambiano le regole del gioco se viene
utilizzato un dominio diverso.

\textbf{Codifica effettiva}: Dato un dominio \emph{D}, si può trovare
una funzione $\alpha : D \rightarrow \mathbb{N}$ biunivoca e effettiva,
con inversa effettiva.

Alcuni domini che vanno bene sono $\mathbb{Z}, \mathbb{Q}, \mathbb{A}*$ ma non è possibile
utilizzare $\mathbb{R}$ perché non è numerabile.

Una volta fissato un dominio con una codifica effettiva, una funzione
$f: D \rightarrow D$ è \textbf{calcolabile} se $f^* = \alpha \circ f \circ \alpha^{-1}$ è URM-Calcolabile.

Ad esempio: $Q = \mathbb{Z},\: f : \mathbb{Z} \rightarrow \mathbb{Z},\: f(z) =
|z|$ è calcolabile perché è possibile trovare una funzione $\alpha : \mathbb{Z}
\rightarrow \mathbb{N}$ che codifica i numeri positivi con i numeri pari e
quelli negativi con i numeri dispari.

Così facendo viene stabilita una corrispondenza \textit{1-1} tra $\mathbb{Z}$ e $\mathbb{N}$.


$$\alpha(z) = \begin{cases}
2z\:& \text{se \textit{z} è positivo} \\
-2z-1\:& \text{se z è negativo}
\end{cases}$$

L'inversa viene definita come

$$\alpha^{-1}(n) = \begin{cases} \frac{n}{2} \:& \text{se \textit{n} è pari}\\
 \frac{-(n+1)}{2}\:& \text{se \textit{n} è dispari}\end{cases}$$

Pertanto $f^*$ risulta essere

\begin{align*}
	f^* &=\begin{cases}
	\alpha(f(\frac{n}{2})), \:\:\:& \text{se \textit{n} è pari}\\
	\alpha(f(\frac{-(n+1)}{2})), \:\:\:& \text{se \textit{n} è dispari}
	\end{cases} \\	
	f^* &= \begin{cases}
	n, \qquad\qquad\qquad\qquad& \text{se \textit{n} è pari} \\
	n+1, \qquad\qquad\qquad\qquad& \text{se \textit{n} è dispari}
	\end{cases} 
\end{align*}


\section{Composizione di funzioni calcolabili}\label{composizione-di-funzioni-calcolabili}

Si possono ottenere delle funzioni calcolabili utilizzando delle
composizioni di funzioni calcolabili più semplici.

La combinazione può essere fatta mediante:

\begin{itemize}
\item
  \textbf{Composizione funzionale}
\item
  \textbf{Ricorsione primitiva}, ovvero che termina sempre
\item
  \textbf{Minimizzazione illimitata}, ovvero l'iterazione fino a
  quando una certa condizione non viene soddisfatta.
\end{itemize}

In termini più tecnici, la classe $\mathcal{C}$ è \textbf{chiusa}
rispetto alle operazioni sopra riportate.

Ci si può spingere oltre, dimostrando che tutte le funzioni calcolabili
possono essere ottenute da combinazioni di funzioni calcolabili
elementari.

\textbf{Funzioni di base}: insieme delle funzioni calcolabili
elementari:

\begin{itemize}
\item
  $\underline{0} : \mathbb{N} \rightarrow \mathbb{N}^k,\: \underline{0}(x_1,\ldots{},x_k) = 0$
\item
  $succ: \mathbb{N} \rightarrow \mathbb{N},\: succ(x) = x+1$
\item
  $U_i^k : \mathbb{N}^k \rightarrow \mathbb{N},\: U_i^k(x_1,\ldots,x_k) = x_i$
\end{itemize}

e possono essere tutte calcolate da delle singole istruzioni URM.

\subsection{Notazioni e ipotesi per i programmi}\label{notazioni-e-ipotesi-per-i-programmi}

Per dimostrare la composizione di funzioni calcolabili viene fatta una
composizione dei programmi che le calcolano.

Per rendere le varie dimostrazioni più semplici si utilizzando varie
notazioni/ipotesi:

\begin{itemize}
\item
  Dato un programma \emph{P} in URM, $\rho(P) = max(\{i |  R_i \text{ è usato in }P\})$
\item
  \emph{P} è in forma standard, ovvero tutti i salti che fanno terminare
  il programma vanno a finire all'istruzione \emph{L(P) + 1}.
\item
  La concatenazione dei programmi viene indicata con \emph{P Q} e si
  assume che questa sia corretta rispetto ai salti a fine programma,
  ovvero che quando viene effettuata la concatenazione, le istruzioni di
  salto vengano aggiornate in modo opportuno.
\item
  Dato \emph{P}, $P[i_1,\ldots, i_k \rightarrow l]$ indica
  il programma che prende in input i registri $i_1\ldots, i_k$ e
  mette l'output in \emph{l} e non assume niente sugli altri registri,
  ovvero modifica \emph{P} in

\begin{lstlisting}[language=URM]
T(i1,1)
...
T(ik,k)
Z(k+1)
...
Z(rho(P))
P
T(1,l)
\end{lstlisting}

  C'è un piccolo problema con i vari trasferimenti \texttt{T(ij,j)} che
  possono sovrascrivere alcuni dati, pertanto è necessario stare attenti che
  questo non succeda utilizzando dei registri di supporto.
\end{itemize}
