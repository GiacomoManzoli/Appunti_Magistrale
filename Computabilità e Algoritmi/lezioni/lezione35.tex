% !TEX encoding = UTF-8
% !TEX TS-program = pdflatex
% !TEX root = computabilità e algoritmi.tex
% !TEX spellcheck = it-IT
\subsubsection{Inserimento del segmento $s_i$}\label{inserimento-del-segmento-si}

\begin{enumerate}
\item   Cerca le regioni intersecate dal segmento $s_i$.
\item   Suddividi le regioni intersecate dal segmento $s_i$. Se vengono trovate due regioni adiacenti che hanno gli stessi segmenti superiori e inferiori, uniscile.
\end{enumerate}

La ricerca delle regioni intersecate può essere fatta in $O(\log(i))$ e può essere fatta in modo analogo alla ricerca di un punto nella regione. In totale per questa fase si ha $O(n \log n)$.

Per quanto riguarda il secondo passo, ogni regione può essere suddivisa in al più quattro parti con un costo costante $O(1)$. 
Si può quindi effettuare un analisi ammortizzata attribuendo ad ogni regione che viene creata e successivamente suddivisa un costo ammortizzato che comprende anche il costo della sua successiva suddivisione.

Otteniamo quindi un costo ammortizzato costante per ogni regione creata e poi suddivisa mentre il costo delle regioni finali risulta nullo, in quanto prepagato con il costo ammortizzato delle regioni suddivise.

Dobbiamo quindi calcolare il numero di regioni che venogno suddivise durante la costruzione del trapezioidale.

In una partizione i lati verticali sono in corrispondenza dei vertici dei segmenti inseriti. Per ogni vertice di un segmento inserito ci sono al più due lati sulla sua verticale: quello che congiunge il vertice con il segmento immediatamente sopra e quello che lo congiunge con quello immediatamente sotto.

Siccome i vertici dei segmento sono $2n$, il numero totale di vertici della partizione trapezioidale è minore o uguale di $6n$. \todo{Da dove viene il 6n?}

Possiamo vedere la partizione trapezioidale come un grafo planare con al più $6n+1$ vertici, facendo convergere tutti i lati illimitati ad un vertice \textit{z}.

Per la formula di Eulero in un grafo planare il numero di regioni \textit{f}, il numero di lati \textit{e} ed il numero di vertici \textit{v} sono legati dalla relazione

$$
v -e +f = 2
$$ 

Inoltre in un grafo planare con almeno 3 vertici vale la disuguaglianza

$$
e \leq 3v -6
$$

Combinando le due cose si ottiene

$$
f = 2 + e -v \leq 2v - 4 \leq 12n -2 = O(n)
$$

Dunque il numero di regioni della partizione trapezioidale di partenza finale è $O(n)$ e quindi anche le regioni suddivise durante la costruzioni sono $O(n)$.

Siccome abbiamo attribuito un costo ammortizzato costante ad ogni regione che viene suddivisa, il coso totale dell'esecuzione dei secondi passi è $O(n)$ e quindi l'interno costo della procedura è dato da

$$
\underbrace{O(n \log n)}_{\text{tot fase 1}} + \underbrace{O(n)}_{\text{tot fase 2}} = O(n \log n)
$$

\todo[inline]{la parte relativa all'implementazione è stata fatta ma non sarà chiesta.}


\chapter{Algoritmi Randomizzati}

\section{Problema del Rendering}\label{problema-del-rendering}

C'è una serie di segmenti che deve essere proiettata su un altro segmento secondo un determinato punto d'osservazione.

L'idea è quella di effettuare prima la proiezione dei segmenti più lontani, spostandosi via via su quelli più vicini.

Possiamo assumere che effettuare la proiezione di un segmento sia fatta in tempo costante dalla scheda grafica.

Se ci sono dei segmenti che si intersecano è necessario suddividere uno dei due in due segmenti, stando attenti che con le coordinate intere non possiamo calcolare il punto di intersezione di un segmento.

Si può quindi supporre di non avere segmenti che si intersecano.

Il test di vicinanza può essere fatto con il prodotto scalare in tempo costante e pertanto l'ordinamento può essere fatto in $O(n \log n)$ e poi possono essere proiettati dal più lontano al più vicino.

Se però il punto di osservazione si sposta, sarebbe carino pre-elaborare la scena in modo da poter fare l'aggiornamento in tempo costante.
