\section{Ordinamento topologico}\label{ordinamento-topologico}

Un \textbf{ordinamento topologico} definito su un grafo orientato
aciclico (\textbf{DAG}) è un ordinamento dei suoi vertici tale che per
ogni arco $uv \in E$, il vertice \emph{u} precede il vertice
\emph{v}.

\begin{figure}[htbp]
\centering
\includegraphics[width=0.6\textwidth]{./notes/immagini/l7-fig1.png}
\caption{Esempio di DAG}
\end{figure}

L'algoritmo effettua una ricerca in profondità e una volta completata la
visita di un nodo, lo inserisce in cima alla pila, ottenendo così una
pila ordinata per tempo di finitura decrescente.

\begin{breakablealgorithm}
	\caption{TS: ordinamento topologico di un grafo}
	\begin{algorithmic}[1]
		\Function{TS}{$ G $}
			\For{$ \forall v \in G.V $}
				\State $ v.color \gets bianco $
			\EndFor
			\State $ P \gets \emptyset $
			\For{$ \forall v \in G.V $}
				\If{$ v.color = bianco $}
					\State \textsc{TS-Visit}$ (G,v,P) $
				\EndIf
			\EndFor
			\State \Return $ P $
		\EndFunction
		\Statex
		\Function{TS-Visit}{$ G,u,P $}
			\State $ u.color \gets grigio $
			\For{$ \forall v \in Adj[u] $}
				\If{$ v.color = bianco $}
					\State \textsc{TS-Visit}$ (G,v,P) $
				\EndIf
			\EndFor
			\State $ u.color \gets nero $
			\State \textsc{PUSH}$ (P,u) $
		\EndFunction
	\end{algorithmic}
\end{breakablealgorithm}

La complessità dell'algoritmo è la stessa della visita in profondità,
ovvero \emph{O(n+m)}.

\subsection{Correttezza di TS}\label{correttezza-di-ts}

Per dimostrare la correttezza di \textsc{TS} è necessario dimostrare che
per ogni arco \emph{uv} il vertice \emph{v} viene finito prima del
vertice \emph{u}, ovvero \emph{u} si trova più in alto nella pila
rispetto a \emph{v}.

Prima di tutto è necessario dimostrare che un grafo è un \textbf{DAG} se
e solo se nella visita in profondità non si trova nessun arco
all'indietro e questo può essere fatto in modo analogo a come avviene
per un grafo non orientato.

Ovvero, se in una visita in profondità si trova un arco all'indietro \textit{vu}, 
allora tale arco aggiunto al cammino da \textit{u} a \textit{v} forma un ciclo.
Il cammino tra \textit{u} e \textit{v} esiste perché \textit{v} è discendete di \textit{u}.

Viceversa, supponendo che il grafo abbia un ciclo. 
Sia \textit{v} il primo vertice del ciclo ad essere scoperto e
sia \textit{uv} l’arco del ciclo che entra in \textit{v}.
Quando \textit{v} viene scoperto esiste un cammino bianco da \textit{v} ad \textit{u} e quindi, per la proprietà del cammino bianco, \textit{u} è discendente di \textit{v}.
Di conseguenza \textit{uv} è un arco all’indietro.

Tornando alla correttezza di \textsc{TS}, quando viene esplorato l'arco
\emph{uv}, il vertice \emph{u} è grigio e il vertice \emph{v} non può
essere grigio, altrimenti \emph{uv} sarebbe un arco all'indietro.

Se \emph{v} è nero, vuol dire che è già stato finito, mentre \emph{u}
non lo è, quindi \emph{u} viene inserito più in alto nella pila rispetto
a \emph{v}.

Se \emph{v} è bianco, per il teorema del cammino bianco, è anche un
discendente di \emph{u} e quindi viene finito prima di \emph{u} e di
conseguenza \emph{v} si trova più in basso nella pila rispetto a
\emph{u}.

\section{Componenti fortemente connesse}\label{componenti-fortemente-connesse}

La visita in profondità può essere utilizzata per individuare le
componenti fortemente connesse, ovvero i gruppi di nodi del grafico che
sono mutualmente raggiungibili tra di loro.

L'algoritmo che calcola le componenti fortemente connesse lavora in 3
passi:

\begin{enumerate}
\item
  Visita in profondità del grafo \emph{G} per ordinare i vertici in
  ordine di finitura decrescente, un po' come avviene nell'ordinamento
  topologico, con la differenza che non è garantita l'assenza di cicli.
\item
  Calcola il grafo trasposto $G^T$.
\item
  Visita in profondità il grafo trasposto $G^T$ usando l'ordine
  dei vertici nell'ordine calcolato al punto 1.
\end{enumerate}

Gli alberi della foresta così calcolata rappresentano le componenti
fortemente connesse.

La complessità dell'algoritmo è sempre \emph{O(n+m)}, perché tutti e 3 i
passi hanno complessità \emph{O(n+m)}.

\begin{figure}[htbp]
\centering
\includegraphics[width=0.6\textwidth]{./notes/immagini/l7-fig2.png}
\caption{Ricerca in profondità sul grafo di partenza e su quello di
trasposto. In verde la relazione padre-figlio. I numeri dei nodi nel
grafo trasposto indicano l'ordine di visita.}
\end{figure}

La dimostrazione di correttezza dell'algoritmo viene dopo perché si basa
su determinate proprietà che devono essere dimostrate.

\subsection{Proprietà dei cammini}\label{proprietuxe0-dei-cammini}

Siano \emph{C} e \emph{C'} due \textbf{CFC} distinte. Se esiste un
cammino $P_{uu'}$ da un vertice \emph{u} di \emph{C} ad un vertice
di \emph{u'} di \emph{C'}, non esiste nessun cammino $P_{vv'}$ da un
vertice di \emph{v'} di \emph{C'} a un vertice \emph{v} di \emph{C}.

La dimostrazione è banale, perché se ci fosse il cammino $P_{vv'}$
ci sarebbe un ciclo tra le due \textbf{CFC} e quindi queste non
sarebbero distinte, il che contraddice l'ipotesi.

\subsection{Proprietà dei tempi di fine}\label{prorpietuxe0-dei-tempi-di-fine}

Dato un insieme di vertici $ U \subseteq V $, \textit{d(U)} indica il tempo in cui viene scoperto il primo vertice in \textit{U}, mentre \textit{f(U)} indica il tempo in cui viene terminata l'esplorazione dell'ultimo vertice in \textit{U} durante la visita in profondità.

Si ha quindi che:

$$
d(U) = \min_{u \in U} (u.d) \: \text{e} \: f(U) = \max_{u \in U}(u.f)
$$

Siano \emph{C} e \emph{C'} due \textbf{CFC} distinte. Se esiste un arco
\emph{uv} da $u \in C$ a $v \in C'$, allora $f(C) > f(C')$, ovvero tutti i vertici di \emph{C}
vengono finiti prima dei vertici di \emph{C'}.

Possono quindi verificarsi due casi, ovvero che viene prima scoperto un
vertice di \emph{C} o di \emph{C'}.

Se $d(C) < d(C')$, quando viene scoperto il primo vertice
\emph{x} tutti i vertici, sia di \emph{C} che di \emph{C'} sono bianchi.
Quindi c'è un cammino bianco da \emph{x} a tutti i vertici di \emph{C}
e, a causa dell'arco \emph{uv}, c'è anche un cammino bianco da \emph{x}
a tutti i vertici di \emph{C'}. Per il teorema del cammino bianco, tutti
i vertici di \emph{C} e \emph{C'} diventeranno discendenti di \emph{x} e
quindi $x.f = f(C) > f(C')$.

Se $ d(C) > d(C') $, quando viene scoperto il primo vertice \textit{y} di \textit{C'}, tutti i vertici di \textit{C} e di \textit{C'} sono bianchi.
C'è quindi un cammino bianco da \textit{y} ad ogni vertice di \textit{C'} e quindi $ y.f = f(C') $.
Siccome esiste l'arco \textit{uv} non può esister nessun cammino da un vertice di \textit{C'} ad un vertice di \textit{C}, quindi \textit{C} non è raggiungibile da \textit{y}, dunque $ d(C) > f(C') $ e a maggior ragione $ f(C) > f(C') $

Come conseguenza di ciò si ha che se \textit{C} e \textit{C'} sono due \textbf{CFC} distinte del grafo \textit{G} e se nel grafo trasposto $ G^T $ esiste un arco \textit{uv} da $ u \in C$ a $ v \in C' $, allora $ f(C) < f(C') $, ovvero durante la visita in profondità del grafo trasposto viene visitata per prima la \textbf{CFC} che nel grafo originale viene finita per prima.
Questo perché le \textbf{CFC} del grafo originale sono le stesse del grafo trasposto e la presenza di $ uv \in G^T:E $ implica che $ vu \in G.E$, quindi per quanto precedentemente dimostrato $ f(C) < f(C') $ (considerando il grafo \textit{G}).

\subsection{Correttezza dell'algoritmo}\label{correttezza-dellalgoritmo}

La visita in profondità di $G^T$ parte dal vertice $x_1$
terminato per ultimo dalla visita in profondità di \emph{G}. Questo
vertice farà parte di una certa componente connessa $C_1$.

Per ogni altra \textbf{CFC} \emph{C} si ha, per la proprietà dei tempi
di fine, $x_1.f = f(C_1) > f(C)$ perché non esiste
nessun arco \emph{vu} in $G^T$ da $v \in C_1$ a $u \in C$ e quindi l'albero costruito a partire da $x_1$ conterrà tutti e
soli i verti di $C_1$.

Dopodiché l'algoritmo continua da $x_2$ tra quelli terminati per
ultimi e che non sono $C_1$, sul quale è possibile fare lo
stesso ragionamento.

Questo continua fino a che non vengono visitati tutti i nodi, ottenendo
così una foresta di alberi che sono componenti fortemente connesse.

\subsection{(Esercizio) Componenti fortemente connesse e modifiche agli archi}\label{esericizio-componenti-fortemente-connesse-e-modifiche-agli-archi}

Come variano le componenti fortemente connesse aggiungendo un arco?
Trovare un esempio in cui il numero di \textbf{CFC} non cambia e un
esempio in cui il numero di \textbf{CFC} diminuisce di 1 ed un esempio
in cui il numero di \textbf{CFC} da 10 diventa 1.

\begin{itemize}
	\item \textbf{Non cambia} se aggiungo un arco che collega due nodi della stessa componente connessa oppure se aggiungo un arco \textit{uv} tra due vertici appartenenti a \textbf{CFC} distinte, $ v \in C $ e $ u \in C' $, tali che \textbf{non esiste} un cammino che collega un vertice di \textit{C'} a  uno di\textit{C}.
	\item \textbf{Diminuisce di 1} se aggiungo un arco \textit{uv} tra due vertici appartenenti a \textbf{CFC} distinte, $ v \in C $ e $ u \in C' $, tali che \textbf{esiste} un cammino che collega un vertice di \textit{C'} a uno di \textit{C}.
	\item \textbf{Da 10 diventa 1} se ad un cammino di 10 vertici aggiungo un arco che collega l'ultimo vertice del cammino con il primo.
\end{itemize}

\subsection{Grafo delle componenti fortemente connesse}\label{grafo-delle-componenti-fortemente-connesse}

Sia \emph{G} un grafo orientato per il quale sono state calcolate le
componenti fortemente connesse, indicate con $C_i$.

Si può definire il grafo delle componenti fortemente connesse, il quale ha come vertici i vari $C_i$ e come archi:

$$ 
E = \{xy | \exists uv \in G.E, \: x,u \in C_i, \: y,v \in C_j\}
$$

Il grafo così ottenuto astrae i cicli del grafo di partenza e risulta essere un \textbf{DAG}.

\subsubsection{Esercizio - Grafi semi-connessi}\label{esercizio---grafi-semiconnessi}

Un grafo orientato è semi-connesso se per ogni due vertici \emph{u} e
\emph{v} esiste o un cammino da \emph{u} a \emph{v} oppure un cammino da
\emph{v} a \emph{u}. Trovare un algoritmo efficiente per verificare se
un grafo è semi-connesso.

Questo può essere fatto costruendo il grafo delle componenti fortemente
connesse.

Se questo grafo contiene un cammino che connette tutti i suoi vertici allora il grafo è semplicemente connesso.

Questo perché quando c'è un cammino tra tutte le CFC, si ha esiste un cammino tra tutti i nodi di una CFC a tutti i nodi di tutte le altre CFC.

\todo[inline]{verificare se è corretto}