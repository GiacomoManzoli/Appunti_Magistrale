\section{Metodi seminumerici}\label{metodi-seminumerici}

Famiglia di algoritmi per il pattern matching che utilizzano operazioni
sui bit e aritmetiche al posto del confronto di caratteri.

\subsection{Metodo Shift-And}\label{metodo-shift-and}

Metodo che funziona molto bene per pattern corti.

L'algoritmo calcola una matrice booleana $M_i[j]$ con
$n+1$ righe e \emph{m} colonne, tale che

$$
M_i[j] = 1 \text { se e solo se } P[1,j] = T[i-j+1, i]
$$

ovvero, $ M_i[j] = 1 $ se e solo se in $ T[i-j+1] $ inizia il prefisso del pattern $ P[1,j] $, il che implica che se nell'ultima colonna della riga \emph{i}-esima
della matrice c'è un 1, allora c'è un'occorrenza del pattern a partire
dalla posizione $i-m+1$.

Prima di effettuare la ricerca l'algoritmo effettua una pre-elaborazione
nella quale calcola un vettore booleano $U_x$ di lunghezza
\emph{m} per ogni carattere dell'alfabeto, tale che

$$U_x[j] = \begin{cases} 
1,& \text{ se } P[j] = x\\
0, & \text{ altrimenti} 
\end{cases}$$

Ovvero $U_x$ memorizza tutte le posizioni del pattern in cui
compare il carattere \emph{x}.

L'algoritmo poi si basa sull'operazione di \textsc{Shift} a destra, la
quale modifica una riga di bit spostando tutti i bit a destra di un
posto, scartando il bit più a destra e inserendo un 1 nella posizione più
a sinistra.

\begin{align*}
R &= 0111001000101 \\
\textsc{Shift}(R) &= 1011100100010 \\
\textsc{Shift(Shift}(R)) &= 1101110010001
\end{align*}

Il vantaggio di questo approccio è che se la parola è sufficientemente
corta, questa istruzione viene eseguita da una singola istruzione
macchina.

La prima riga della matrice $M_0$ ha tutti i bit uguali a 0 perché
nessun prefisso non nullo del pattern può occorrere nel testo prima del
carattere 0.

Le righe successive vengono calcolate utilizzando

$$M_i = \textsc{And}\big(\textsc{Shift}(M_{i-1}), U_{T[i]}\big) $$

dove \textsc{And} è l'operazione macchina che effettua l'\textit{and} logico bit
a bit.

\subsubsection{Dimostrazione di correttezza}\label{dimostrazione-di-correttezza}

La dimostrazione viene effettuata per induzione.

Il caso base riguarda $M_0$, ma questo è corretto per definizione, in quanto nessun prefisso non nullo del pattern può occorrere prima del carattere 0.

Nel caso induttivo si ha che:

$$M_i{[1]} = 1 \Leftrightarrow \big( \underbrace{1}_{\text{lo shift inserisce a sinistra un 1}} \wedge\: U_{T[i]}[1] = 1  \big)$$

ma, per costruzione di $ U_x $, $U_{T[i]}[1] = 1 \Leftrightarrow P[1] = T[i] $, pertanto il primo bit della riga risulta calcolato correttamente. 

Per i bit successivi al primo si ha:

$$
M_i[j] = 1 \Leftrightarrow \big( M_{i-1}[j-1] = 1 \wedge U_{T[i]}[j] = 1 \big)
$$

Per ipotesi induttiva $M_{i-1}[j-1]$ è stata calcolata correttamente, quindi se $M_{i-1}[j-1] = 1$, si ha che:

$$
P[1, j-1] = T[i -j +1, i-1]
$$

Dal momento che $U_{T[j]} = 1$ se e solo se $P[j] =T[i]$, si ha che

$$
P[1,j] = T[i-j+1,i]
$$

e quindi il calcolo di $M_{i}[j]$ viene effettuato in modo
corretto.

\subsubsection{Algoritmo}\label{algoritmo}

\begin{breakablealgorithm}
	\caption{ShiftAnd: Algoritmo }
	\begin{algorithmic}[1]
\Function{ShiftAnd}{$P,T,\Sigma$}
    \For{$ \forall x \in \Sigma$}
        \State $ U_x \gets \vec{0} $
    \EndFor
    \For{$ j = 1 \text{ to } m$}
        \State $U_{P[j]}[j] =1$
    \EndFor
    \State $M_0 \gets \vec{0}$
    \For{$i = 1 \text{ to } n$}
        \State $M_i \gets \text{\textsc{And}}(\text{\textsc{BitShift}}(M_{i-1}), U_{T[i]})$
        \If{$M_i[m] = 1$}
            \State occorrenza in $i -m +1$
        \EndIf
    \EndFor
\EndFunction
\end{algorithmic}
\end{breakablealgorithm}

La complessità dell'algoritmo è $O(\Sigma + m + n)$, tuttavia può essere ridotta andando ad inizializzare $U_x$ solo
per i caratteri che compaiono nel pattern.

\subsection{Metodo dell'impronta di Karp e Rabin}\label{metodo-dellimpronta-di-karp-e-rabin}

\textbf{Semplificazione}: consideriamo solo stringhe che contengono 0 o
1, tanto l'algoritmo può comunque essere esteso ad alfabeti con più
caratteri.

L'algoritmo considera quindi le stringhe di bit come dei numeri,
rendendo così possibile utilizzare le operazioni di confronto tra numeri
anziché quello tra caratteri.

Con $T_i$ viene indicata la porzione di testo che inizia dal
carattere \emph{i} e che è lunga quanto il pattern

$$
T_i = T[i, i + m -1]
$$

Data una stringa binaria \emph{S} di lunghezza \emph{m}, questa può
essere vista come la codifica di un numero \emph{H(S)}

$$
H(S) = \sum\limits_{j = 1}^m 2^{m-j}S[j]
$$

Così facendo è possibile calcolare sia \emph{H(P)} che $H(T_i)$ e
i due numeri risultano essere uguali se e solo se \emph{P = T}.

Perché le operazioni aritmetiche possano essere eseguite in tempo
costante è necessario che i numeri utilizzati siano rappresentabili con
$O(\log n)$ bit dove \emph{n} è la dimensione dell'input
(\textbf{ipotesi del modello RAM}). Pertanto se il pattern è lungo,
questa ipotesi non vale.

Rabin e Karp hanno pubblicato il \textbf{metodo dell'impronta
randomizzata}: viene scelto casualmente un numero \emph{p} e viene
calcolata

$$H_p(S) = H(S) \mod p$$

Le quantità così calcolate prendono il nome di impronta.

Se c'è un'occorrenza è ovvio che le due impronte sono uguali, ma possono
verificarsi dei casi in cui le due impronte sono uguali e non c'è
un'occorrenza. 

Tuttavia è possibile scegliere \emph{p} in modo che la
probabilità che ci sia una ``\emph{falsa occorrenza}'' sia molto bassa.

C'è però ancora un problema, serve un modo per effettuare questi conti
utilizzando numeri piccoli in modo da avere l'esecuzione delle
istruzioni in tempo costante.

Per risolvere questo problema è possibile utilizzare l'aritmetica
modulare e la regola di Horner: data una stringa $S = b_1 \ldots b_m$, $H_p(S)$ può essere calcolato nel seguente modo

\begin{breakablealgorithm}
	\begin{algorithmic}[1]
\Function{H$ _p$}{$S$}
    \State $ H = b_1 $
    \For{$ j = 2 \text{ to } m$}
        \State $H \gets (2H + b_j) \mod p$
    \EndFor
    \State \Return $H$
\EndFunction
\end{algorithmic}
\end{breakablealgorithm}


Così facendo non ci sono risultati intermedi di lunghezza maggiore di
$2p-1$ (\emph{2H} è lungo al massimo $2p-2$).

Nel calcolo di $H_p(T_i)$ con $i > 2$ è possibile effettuare delle ottimizzazioni, andando a calcolarlo a
partire da $H_p(T_{i-1})$, riducendo la complessità da
$O(mn)$ a $O(n)$:

\begin{verbatim}
|------ T i-1 ------|
X|------ T i   ------|

|------ T i-1 ------|0  <-- 2 T[i-1]
X|------ T i   ------|

 |----- T i-1 ------|0  <-- 2 T[i-1] - 2^m T[i-1]
X|------ T i   ------|

 |----- T i-1 -------|  <-- 2 T[i-1] - 2^m T[i-1] + T[i+m-1]
X|------ T i   ------| 
\end{verbatim}

\begin{align*}
	H(T_i) &= \sum\limits_{j=1}^m 2^{m-j} T[i+j -1] \\
	           &= T[i+m -1] + \sum\limits_{j=1}^{m-1} 2^{m-j} T[i+j -1] \\
	           &= T[i+m -1] + \sum\limits_{j=2}^{m-1} 2^{m-j+1} T[i+j -2] \\
	           &= T[i+m -1] - 2^m T[i-1] + \sum\limits_{j=1}^{m-1} 2^{m-j+1} T[i+j -2] \\
	           &= T[i+m -1] - 2^m T[i-1] + 2\sum\limits_{j=1}^{m-1} 2^{m-j} T[i+j -2] \\
	           &= T[i+m -1] - 2^m T[i-1] + 2H(T_{i-1})
\end{align*}

Pertanto $ H_p(T_i) $ può essere calcolato come

$$
H_p(T_i) = \big( T[i+m -1] - (2^m \mod p) T[i-1] + 2H_p(T_{i-1}) \big) \mod p
$$

Da notare che il tutto non viola il modella RAM perché il valore è compreso tra $ -p+1 $ e $ 2p-1 $ e $ 2^m \mod p $ può essere calcolato come

$$
2^m \mod p  \begin{cases}
2 &\text{ se } m = 1\\
2 (2^{m-1} \mod p) & \text{ se } m > 1
\end{cases}
$$

e quindi i calcoli vengono fatti in tempo costante.

\subsubsection{Algoritmo}\label{algortimo}

\begin{breakablealgorithm}
	\caption{RabinKarp: Algoritmo di RabinKarp}
	\begin{algorithmic}[1]
\Function{RabinKarp}{$P,T$}
    \State $z \gets 2$
    \For{$ j = 2 \text{ to } m$}
        \State $ z \gets (2z) \mod p$
    \EndFor
    \State // $ z = 2^m \mod p$
    \State $ x \gets P[1]$
    \For{$j = 2 \text{ to } m$}
        \State $ x \gets (2x + P[j]) \mod p$
    \EndFor
    \State // $x = H_p(P)$
    \State $y \gets T[1]$
    \For{$j = 2 \text{ to } m$}
        \State $y = (2y + T[j]) \mod p$
    \EndFor
    \State // $y = H_p(T_1)$
    \If{$x = y$}
        \State Segnala una possibile occorrenza in posizione 1
    \EndIf
    \For{$i = 2 \text{ to } n-m+1$}
        \State $y \gets (T[i+m-1] - zT[i-1] + 2y) \mod p$
        \State // $y = H_p(T_i)$
        \If{$x = y$}
            \State Segnala una possibile occorrenza in posizione $i$
        \EndIf
    \EndFor
\EndFunction
\end{algorithmic}
\end{breakablealgorithm}

\subsubsection{La scelta di \textit{p}}\label{la-scelta-di-p}

I risultati migliori si ottengono scegliendo come \emph{p} un numero
primo in un opportuno intervallo, in modo da minimizzare i falsi
positivi.

La scelta si basa su alcune proprietà dei numeri primi.

\paragraph{Numero di numeri primi minori di un certo \emph{n}}

Sia $\pi(n)$ il numero di primi minori o uguali a \emph{n}.

Chebyschev ha dimostrato che per ogni $n \geq 11$, si
ha che

$$
\frac{n}{\ln (n)} \leq \pi (n) \leq 1,26 \frac{n}{\ln (n)}
$$

\paragraph{Prodotto dei numeri primi}\label{prodotto-dei-numeri-primi}

Per ogni $n \geq  29$, il prodotto di tutti i numeri
primi minori o uguali ad \emph{n} è strettamente maggiore di
$2^{n}$.

$$
q_1 \cdot q_2 \cdot \ldots \cdot q_{\pi (n)} > 2^n
$$

Come conseguenza si ha che se $n \geq 29$, qualsiasi
numero \emph{x} minore o uguale di $2^n$ ha meno di $\pi(n)$
divisori primi distinti.

Questo si dimostra per assurdo, perché se \emph{x} ha $m \geq \pi(n)$ divisori primi distinti $p_1, \ldots , p_m$, si ha che

$$
q_1 \cdot \ldots \cdot q_k \leq p_1 \cdot \ldots \cdot p_m \leq x \leq 2^n
$$

Dove $q_1,\ldots , q_k$ sono i primi \emph{k} numeri primi. 
Il loro prodotto è strettamente maggiore di $2^n$ e questo è assurdo.

\paragraph{Teorema fondamentale per Rabin e Karp}\label{teorema-fondamentale-per-rabin-e-karp}

Siano \emph{P} e \emph{T} due stringhe di lunghezza \emph{m} e \emph{n}
tali che $mn \geq 29$ e sia \emph{N} un intero positivo
qualsiasi maggiore o uguale a \emph{mn}.

Se \emph{p} è scelto casualmente tra tutti i numeri primi minori o
uguali di \emph{N} la probabilità di una falsa occorrenza tra \emph{P} e
\emph{T} è minore di

$$
\frac{\pi (nm)}{\pi (N)}
$$

\subparagraph{Dimostrazione}\label{dimostrazione}

Sia \emph{R} l'insieme di tutti gli indici di tutte le posizioni in
\emph{T} in cui non c'è un'occorrenza del pattern \emph{P}, ossia
$H(T_i) \neq H(P)$.

Consideriamo il prodotto

$$
\Pi = \prod\limits_{i \in R} \big(\big|H(T_i) - H(P)\big|\big)
$$

Tale prodotto deve essere minore di $2^{mn}$ in quanto $\big|H(T_i) - H(P)\big| \leq 2^m$ per ogni
\emph{i}.

Per quanto precedentemente enunciato si ha che $ \Pi $ ha al più
$\pi(nm)$ divisori primi distinti.

Supponendo che ci sia una falsa occorrenza del pattern \emph{P} in
qualche posizione \emph{i} del testo \emph{T}, ovvero $H(T_i) \neq
H(P)$ e $H_p(T_i) = H_p(P)$.

Si ha che \emph{p} è un divisore della differenza $H(T_i) - H(P)$ e
quindi è divisore anche di $ \Pi $.

Siccome \emph{p} è stato scelto casualmente tra i primi $\pi(N)$
numeri primi, la probabilità di una falsa occorrenza è minore o uguale
di $\frac{\pi (nm)}{\pi (N)}$

Quindi se \emph{N} è sufficientemente la probabilità di un falso
positivo è bassa.

Tipicamente viene scelto $N = n^2m$ perché

$$
\frac{\pi (nm)}{\pi (n^2m)} \leq \frac{1,26 \frac{nm}{\ln (nm)}}{ \frac{n^2 m}{\ln (n^2 m)}} = 1,26 \frac{\ln (n^2 m) }{n \ln(nm)} = \frac{1,26}{n} \frac{ 2\ln (n) + \ln (m)}{\ln (n) + \ln (m)} \leq \frac{2,52}{n}
$$
