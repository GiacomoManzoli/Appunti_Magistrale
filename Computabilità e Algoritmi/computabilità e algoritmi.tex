%----------------------------------------------------------------------------------------
% PACKAGES AND OTHER DOCUMENT CONFIGURATIONS
%----------------------------------------------------------------------------------------

% !TEX encoding = UTF-8
% !TEX TS-program = pdflatex
% !TEX root = computabilità e algoritmi.tex
% !TEX spellcheck = it-IT

\documentclass[a4paper, 11pt]{report} % Font size (can be 10pt, 11pt or 12pt) and paper size (remove a4paper for US letter paper)
\usepackage[italian]{babel}      							% Lingua italiana
\usepackage[margin=1.2in]{geometry}             % Imposta i margini del documento

\usepackage[T1]{fontenc} % Required for accented characters
\usepackage[mathletters]{ucs}    % Caratteri matematici come UTF8
\usepackage[utf8]{inputenc}      % Ancora utf8

\usepackage{eurosym}                %simbolo dell'euro
\usepackage{listings}
\usepackage[usenames,dvipsnames,svgnames,table]{xcolor}
% Imposta lo spazio nella list of listing in modo simile alla list of figures/tables
%\makeatletter
%\let\my@chapter\@chapter
%\renewcommand*{\@chapter}{%
%  \addtocontents{lol}{\protect\addvspace{10pt}}%
%  \my@chapter}
%\makeatother


\definecolor{codegreen}{rgb}{0,0.6,0}
\definecolor{codegray}{rgb}{0.5,0.5,0.5}
\definecolor{backcolor}{rgb}{0.98,0.98,0.98}

\renewcommand{\lstlistingname}{Codice}% Listing -> codice
\renewcommand{\lstlistlistingname}{Elenco dei frammenti di codice}% List of Listings -> Frammenti di codice

\lstdefinestyle{mystyle}{
    backgroundcolor=\color{backcolor},   
    commentstyle=\color{Peach}\ttfamily,
    keywordstyle=\color{RoyalBlue},
    numberstyle=\tiny\color{codegray},
    stringstyle=\color{SeaGreen}\ttfamily,
    basicstyle=\footnotesize\ttfamily,
    breakatwhitespace=false,         
    breaklines=true,                 
    captionpos=b,                    
    keepspaces=true,                 
    numbers=left,                    
    numbersep=5pt,                  
    showspaces=false,                
    showstringspaces=false,
    showtabs=false,                  
    tabsize=2,
    frame=trbl, % draw a frame at the top, right, left and bottom of the listing
	frameround=ftff, % angolo in basso a destro curvo
	framesep=4pt, % quarter circle size of the round corners,
	inputencoding=utf8,
    extendedchars=true,
    literate={á}{{\'a}}1 {à}{{\`a}}1 {é}{{\'e}}1 {è}{{\`e}}1 {ù}{{\`u}}1 {ò}{{\`o}}1,
    belowskip=1em,
    aboveskip=1em,
}

 
\lstset{style=mystyle}

\lstdefinelanguage{JavaScript}
{
  % list of keywords
  morekeywords={ true, false, catch, function, break,	new, class, extends, var, require, switch, return, import, if, while, for, this, View, Text, StyleSheet},
  sensitive=false, % keywords are not case-sensitive
  morecomment=[l]{//}, % l is for line comment
  morecomment=[s]{/*}{*/}, % s is for start and end delimiter
  morestring=[b]' % defines that strings are enclosed in double quotes
}

\lstdefinelanguage{JSON}
{
  % list of keywords
  morekeywords={string, boolean, int, Array, Node, Asset, AssetDetail, Filter, FilterItem},
  sensitive=false, % keywords are not case-sensitive
  morecomment=[l]{//}, % l is for line comment
  morecomment=[s]{/*}{*/}, % s is for start and end delimiter
  morestring=[b]" % defines that strings are enclosed in double quotes
}

%\tightlist per compatibilità con pandoc
\providecommand{\tightlist}{%
  \setlength{\itemsep}{0pt}\setlength{\parskip}{0pt}}


\usepackage[labelfont=bf]{caption}

\usepackage[protrusion=true,expansion=true]{microtype} % Better typography
\usepackage{graphicx} % Required for including pictures
\usepackage{wrapfig} % Allows in-line images


\usepackage{subfig}
\usepackage{hyperref}
\usepackage{mathpazo} % Use the Palatino font

\linespread{1.05} % Change line spacing here, Palatino benefits from a slight increase by default
\usepackage{placeins}
\usepackage{sourcecodepro}
\usepackage{hyperref}                   % collegamenti ipertestuali

\usepackage[colorinlistoftodos,prependcaption]{todonotes} %todo

\usepackage{amsmath}
\usepackage{mathtools}

\usepackage{float}
\usepackage{algorithm}
\usepackage{algpseudocode} % https://en.wikibooks.org/wiki/LaTeX/Algorithms#Typesetting_using_the_algorithmicx_package
\usepackage{amssymb}  %$\mathbb{N}$ per il simbolo dei numeri naturali 

\makeatletter
\renewcommand\@biblabel[1]{\textbf{#1.}} % Change the square brackets for each bibliography item from '[1]' to '1.'
\renewcommand{\@listI}{\itemsep=0pt} % Reduce the space between items in the itemize and enumerate environments and the bibliography

\renewcommand{\maketitle}{ % Customize the title - do not edit title and author name here, see the TITLE block below
\begin{flushright} % Right align
{\LARGE\@title} % Increase the font size of the title

\vspace{50pt} % Some vertical space between the title and author name

{\large\@author} % Author name
\\\@date % Date

\vspace{100pt} % Some vertical space between the author block and abstract
\end{flushright}
}

%% breakablealgorithm http://tex.stackexchange.com/questions/33866/algorithm-tag-and-page-break
\makeatletter
\newenvironment{breakablealgorithm}
{% \begin{breakablealgorithm}
	\begin{center}
		\refstepcounter{algorithm}% New algorithm
		\hrule height.8pt depth0pt \kern2pt% \@fs@pre for \@fs@ruled
		\renewcommand{\caption}[2][\relax]{% Make a new \caption
			{\raggedright\textbf{\ALG@name~\thealgorithm} ##2\par}%
			\ifx\relax##1\relax % #1 is \relax
			\addcontentsline{loa}{algorithm}{\protect\numberline{\thealgorithm}##2}%
			\else % #1 is not \relax
			\addcontentsline{loa}{algorithm}{\protect\numberline{\thealgorithm}##1}%
			\fi
			\kern2pt\hrule\kern2pt
		}
	}{% \end{breakablealgorithm}
	\kern2pt\hrule\relax% \@fs@post for \@fs@ruled
\end{center}
}
\makeatother

%----------------------------------------------------------------------------------------
% TITLE
%----------------------------------------------------------------------------------------

\title{\textbf{Computabilità e Algoritmi}\\ % Title
A.A. 2015-2016} % Subtitle

\author{\textsc{Giacomo Manzoli}
\\ 1130822 % Author
\\{\textit{Università degli Studi di Padova}}} % Institution

\date{\today} % Date

%----------------------------------------------------------------------------------------

\begin{document}

\maketitle % Print the title section

%----------------------------------------------------------------------------------------
% ABSTRACT AND KEYWORDS
%----------------------------------------------------------------------------------------

%\renewcommand{\abstractname}{Summary} % Uncomment to change the name of the abstract to something else

\clearpage
\tableofcontents
\listofalgorithms

%\hspace*{3,6mm}\textit{Keywords:} lorem , ipsum , dolor , sit amet , lectus % Keywords

\vspace{30pt} % Some vertical space between the abstract and first section

%----------------------------------------------------------------------------------------
% ESSAY BODY
%----------------------------------------------------------------------------------------
\clearpage

\part{Computabilità}
% !TEX encoding = UTF-8
% !TEX program = pdflatex
% !TEX root = InformationRetrieval.tex
% !TEX spellcheck = it-IT

% 29 Settembre 2016

\chapter{Introduzione}

Lo scopo del reperimento dell'informazione è quello di essere in grado di soddisfare il bisogno informativo di una determinata categoria di utenti, mediante tecniche di ricerca e di memorizzazione dei dati.
Tali dati possono essere in qualsiasi formato: testo, foto, audio, ecc. e possono anche esserci problemi riguardo l'internazionalizzazione dei dati, legati all'utilizzo di lingue diverse.

Si ha quindi che c'è un utente che ha un'esigenza informativa che deve essere colmata reperendo delle informazioni da una collezione di documenti.

\section{Terminologia}

\begin{itemize}
	\item \textbf{Esigenza informativa}: mancanza di informazione che un utente vuole colmare per risolvere un suo problema e prendere delle decisioni.
	\item \textbf{Documento}: inteso come ``materializzazione'' dell'informazione, è quasi un oggetto che contiene informazioni utili a soddisfare l'esigenza informativa.
	\item \textbf{Rilevanza}: proprietà di un ``documento'' di contenere l'informazione utile e necessaria a soddisfare una specifica esigenza informativa dell'utente.
	\item \textbf{Sistema di Reperimento dell'informazione}: sistema che a partire da una query e da un insieme di documenti, utilizza delle specifiche rappresentazione per valutare la similarità tra i documenti e la query, in modo da fornire in risposta all'utente un insieme ordinato di documenti secondo un determinato ranking. Può esserci anche un sistema di feedback basato sulle preferenze dell'utente (assessment).
\end{itemize}


\section{Informazioni organizzative}

Orario delle lezioni

\begin{itemize}
	\item Giovedì dalle 10:30 - Aula Oe
	\item Venerdì dalle 12:30 - Aula Oe
\end{itemize}

Ci saranno degli homework che riguardano:
\begin{itemize}
	\item Un progetto individuale di valutazione di un sistema di reperimento dell'informazione. 
	\item Readings:  si studia uno o due articoli e si fa una presentazione stile seminariale.
\end{itemize}

Il ricevimento è per appuntamento.

Comunicazioni via mail a \url{maristella.agosti@unipd.it}, specificando come oggetto \texttt{[IR2016-2017]} e il nome/cognome/matricola.


\subsection{Modalità d'esame}

\begin{itemize}
	\item Esame scritto su tutto il programma svolto, da 0 a 20 punti, con sufficienza a 12.
	\item Progetto individuale di valutazione, da 0 a 5 punti.
	\item Readings, da 0 a 5 punti.
	\item Eventuale integrazione orale.
\end{itemize}

La pagina di riferimento del corso è \url{http://www.dei.unipd.it/~agosti/ir-2016-2017/}. 
C'è anche il corso sul Moodle del DEI: la password del corso è \texttt{corso-IR201617}.

Testi di riferimento:
\begin{itemize}
	\item Information Retrieval (2ed). Cornelis Joos. 1979. \url{http://www.dcs.gla.ac.uk/Keith/Preface.html}.
	\item Information Retrieval in Practice. W. Bruce Croft. 2010. \url{http://ciir.cs.umass.edu/irbook/}
\end{itemize}


\section{Programma e quadro d'insieme}

L'obiettivo del corso è quello di fornire le competenze necessario per la ideazione, progettazione e implementazione di un sistema di reperimento dell'informazione.

Una parte importante del corso sarà dedicata all'indicizzazione: la parte del sistema che si occupa di gestire la rappresentazione dei documenti e della query dell'utente, specialmente per quanto riguarda l'ambito testuale.

Una volta effettuata l'indicizzazione viene definito un modello di valutazione che è la componente del sistema che si occupa di valutare la similarità tra la query e i documenti.

Il modello non è solo un algoritmo, ma c'è anche un insieme di costrutti che sono formalizzati allo scopo di rendere possibile la rappresentazione del contenuto dei documenti e delle interrogazioni. 

La valutazione del sistema si basa sull'efficacia e riguarda la coerenza che c'è tra i documenti reperiti e la richiesta dell'utente, così come ci sono metodi di valutazione per la parte legata all'indicizzazione e al modello.

\subsection{Programma}

\begin{itemize}
	\item Elementi introduttivi per la rappresentazione, gestione e reperimento automatico dell'informazione testuale.
	\item Indicizzazione: strutture dati idonee al reperimento dell'informazione.
	\item Modelli e sistemi per il reperimento dell'informazione.
	\item Valutazione: collezioni sperimentali, misure di efficacia e efficienza.
\end{itemize}

\chapter{Laboratorio}

\section{Un po' di cose su R}\label{un-po-di-cose-su-r}

\begin{itemize}
\item
  Tutti gli oggetti sono vettori
\item
  \texttt{ls()} per vedere le variabili disponibili
\item
  \texttt{x\ \textless{}-\ c(2,3,4,5)} crea un vettore con 1,2,3,4.
\item
  notazione \texttt{{[}1:20{]}} per un vettore con la successione da 1 a
  20
\item
  \texttt{xx\ \textless{}-\ seq(from=100,\ to=1)} crea sempre una
  sequenza di numeri, con parametro opzionale \texttt{by} per
  specificare lo step
\item
  \texttt{rep(2,5)} crea un vettore con 5 elementi uguali a 2
\item
  \texttt{a\ \textless{}-\ c(rep(2,3),4,5,rep(1,5))},
  \texttt{a\ =\ 2\ 2\ 2\ 4\ 5\ 1\ 1\ 1\ 1\ 1}
\item
  \texttt{2*x} esegue il prodotto scalare
\item
  \texttt{length(x)} per la lunghezza del vettore
\item
  \texttt{max(x)} e \texttt{min(x)}
\item
  \texttt{sum(x)} che ritorna un vettore di un solo elemento con la
  somma
\item
  \texttt{mean(x)}, \texttt{var(x)}, \texttt{range(x)}
\item
  \texttt{x{[}7{]}} per estrarre il settimo elemento di \texttt{x},
  l'indice credo parta da 1
\item
  \texttt{x{[}-4{]}} ritorna un vettore senza il quarto elemento
\item
  \texttt{x\ \textless{}-\ matrix(c(2,3,5,7,11,13),nrow\ =\ 3)} crea una
  matrice con gli elementi specificati e 3 righe. Alternativamente è
  possibile specificare anche il numero di colonne.
\item
  \texttt{x2\ \textless{}-\ scan("nome\ file",\ sep="")} con
  \texttt{sep} opzionale, per caricare il contenuto di un file in un
  vettore, per caricare una matrice
  \texttt{x2\ \textless{}-\ matrix(scan(...),\ ncol\ =\ 3,\ byrow=TRUE}.
\item
  \texttt{str(x)} specifica la struttura dell'oggetto
\item
  \texttt{dim(x)} ritorna la dimensione di una matrice, se invocato con
  un vettore ritorna \texttt{NULL}.
\item
  \texttt{x{[}18,{]}} per ottenere la 18-esima riga di una matrice
\item
  \textbf{Dataframe}: matrice le cui colonne possono avere formati
  diversi
\item
  \texttt{ciliegi\ \textless{}-\ read.table("nome\ file")}.
\item
  \texttt{names(ciliegi)} è il vettore con i nomi delle colonne del
  dataframe
\item
  \texttt{names(ciliegi)\ \textless{}-\ c("diametro",\ "altezza",\ "volume")}
  permette di impostare il nome delle colonne, può anche essere
  specificato come parametro opzionale \texttt{col.names} della
  funzione \texttt{read.table}.
\item
  \texttt{summary(ciliegi)} fornisce degli indicatori per ciascuna
  colonna
\item
  \textbf{Mediana}: elemento centrale di una distribuzione ordinata in
  senso crescente, \textbf{primo e terzo quartile}: generalizzazione
  della mediana, rispettivamente l'elemento che sta al 25 e 75 per cento
  della distribuzione. La differenza tra i due quartili da l'idea di
  quanto è variabile la distribuzione.
\item
  I dataframe possono essere acceduti anche con il nome della colonna
  \texttt{ciliegi\$volume}.
\item
  \textbf{attach di un file}: aggiungere al workspace un oggetto, ovvero
  \texttt{attach(ciliegi)} permette di accedere al nome della colonna
  direttamente utilizzando \texttt{volume}. Come complementare c'è il
  comando \texttt{detach}.
\item
  \texttt{hist(diametro)} crea l'istogramma per il diametro
\item
  \texttt{help(hist)} per avere l'help di una funzione
\item
  l'istrogramma che viene generato di default può contenere dei buchi,
  conviene quindi adattare il numero di colonne utilizzando il parametro
  \texttt{breaks}
\item
  \texttt{boxplot(diametro)} fornisce il box plot di un valore, è un
  grafico che rappresenta la mediana, i quartili e il 5 e 95\%. Risulta
  più espressivo dell'istogramma. L'ampiezza della scatola rappresenta
  la variabilità dei dati.
\item
  \texttt{ciliegi{[}altezza\textgreater{}80,{]}} prende tutti i ciliegi
  con altezza maggiore di 80.
\item
  \texttt{library(MASS)} permette di caricare la libreria MASS
\item
  \texttt{search()} permette di visualizzare la lista degli ottetti in
  cui R va a cercare quando deve eseguire un comando
\item
  Gli attributi qualitativi vengono trattati come tipo Factor
\item
  \texttt{table(painters\$School)} crea la tabella con le frequenze
  delle varie qualità
\item
  \texttt{barplot(..)} fa il plot delle barre per una variabile discreta
\item
  \texttt{pie(...)} fa il grafico a torta, anche se è sconsigliabile
  utilizzare un grafico a torta perché per l'occhio umano fa fatica a
  vedere la differenza tra gli angoli.
\item
  come scale colori si possono utilizzare \texttt{heat.colors(k)},
  \texttt{rainbow(k)}, \ldots{}
\item
  \texttt{plot(x,y)} disegna un diagramma di dispersione, il parametro
  \texttt{pch} specifica il tipo di carattere, \texttt{pch=16}
  rappresenta i pallini pieni, \texttt{col} specifica il colore da
  utilizzare, possono indicare \texttt{col=painter\$School} per far
  variare il colore in base al valore dell'attributo quantitativo
\end{itemize}


\part{Algoritmi}
%\chapter{Lezione 2}
% !TEX encoding = UTF-8
% !TEX TS-program = pdflatex
% !TEX root = computabilità e algoritmi.tex
% !TEX spellcheck = it-IT
\chapter{Lezione 2}
\section{Introduzione e Algoritmi sui grafi}\label{lezione-2---introduzione-e-algoritmi-sui-grafi}

C'è la possibilità di fare un pre-orale nella settimana dei compitini.

Libro: Cormen, Introduzione agli algoritmi e strutture dati
\href{http://catalogo.unipd.it/F/FCKK1DACESL2TDH5CF15FLDL2BUM936U1XG9U15MFDCKI764BV-10675?func=full-set-set\&set_number=011139\&set_entry=000001\&format=999}{BIB}:

\begin{itemize}
\tightlist
\item
  Introduzione agli algoritmi sui grafi

  \begin{itemize}
  \tightlist
  \item
    Strutture dati per i grafi
  \item
    Operazioni elementari sui grafi
  \end{itemize}
\item
  Algoritmi su stringhe (capitolo 22)
\item
  Algoritmi paralleli (capitolo 27)
\item
  Algoritmi di geometria computazionale
\end{itemize}

\subsection{Terminologia dei grafi}\label{terminologia-dei-grafi}

Un grafo \emph{G} è costituito da un insieme di vertici \emph{V} e di
archi \emph{E}. Ad ogni arco vengono associati due vertici in \emph{V}.

Se c'è un ordine tra i due estremi degli archi, il grafo prende il nome
di \textbf{orientato} o \textbf{diretto}. In questo caso, il primo
vertice prende il nome di \textbf{coda} e l'ultimo \textbf{testa}.

Un \textbf{cappio} è un grafo i cui due estremi coincidono.

Un grafo non orientato si dice \textbf{semplice} se non ha cappi e non
ci sono due archi con gli stessi estremi. Mentre se il grafo è
orientato, perché sia semplice non devono esserci archi con gli stessi
estremi, iniziali e finali. Un grafo non semplice prende il nome di
\textbf{multi-grafo}.

\begin{figure}[htbp]
\centering
\includegraphics[width=0.75\textwidth]{./notes/immagini/l2-grafi.png}
\caption{Varie tipologie di grafi}
\end{figure}

Se un grafo è semplice, un arco può essere espresso con:

$$
e = uv \in E \text{, con} u,v \in V
$$

e si dice che l'arco \emph{e} è incidente in \emph{u} e \emph{v}. Da
notare che se il grafo è orientato
$e = uv \neq vu$ e la terminologia diventa
``l'arco \emph{e} esce da \emph{u} entra in \emph{v}''.

Il \textbf{grado} di un vertice \emph{v} viene indicato con $\delta(v)$ e rappresenta il numero di archi incidenti in quel
vertice. Se il grafo è ordinato, il suo \textbf{grado uscente} $\delta^+(v)$ è il numero di archi uscenti e il suo \textbf{grado entrante} è $\delta^-(v)$.

Se due vertici sono collegati da un arco, questi vengono detti
\textbf{adiacenti}.

Un \textbf{cammino} di lunghezza \emph{k} da un vertice \emph{u} ad un
vertice \emph{v} in un grafo \emph{G=(V,E)}, è una sequenza di
\emph{k+1} vertici $x_0 \ldots x_k$, tali che $x_0 = u$, $x_k = v$ e $x_{i-1}x_i \in E \forall i = 1\ldots k$.

Se il cammino ha lunghezza 0, questo viene detto \textbf{nullo}, mentre
se il vertice di partenza coincide con quello di arrivo, il cammino
prende il nome di \textbf{chiuso}.

Un cammino viene detto \textbf{semplice} quanto tutti i vertici che lo
compongono sono distinti, ad eccezione del primo, che può coincidere
con l'ultimo. Un cammino semplice e il primo vertice coincide con
l'ultimo, questo prende il nome di \textbf{ciclo}. L'esempio più
semplice di ciclo è dato da un cappio.

Un grafo \textbf{aciclico} è un grafo che non contiene cicli.

Quando esiste almeno un cammino dal vertice \emph{u} al vertice
\emph{v}, si dice che \emph{v} è \textbf{accessibile} (o
\textbf{raggiungibile}) da \emph{u}. Questa definizione è simmetrica
solamente nel caso di un grafo non orientato.

Un grafo non orientato si dice \textbf{connesso} se esiste almeno un
cammino tra ogni coppia di vertici.

Le \textbf{componenti connesse} di un grafo sono le classi di
equivalenza dei suoi vertici rispetto alla relazione di accessibilità,
ovvero un sottoinsieme di vertici che sono tutti tra loro accessibili.

Nel caso di un grafo orientato, si dice che è \textbf{fortemente
connesso} se esiste almeno un cammino tra ogni vertice del grafo. In
modo analogo è possibile definire le \textbf{componenti fortemente
connesse}

Sia la \textbf{connessione} che la \textbf{connessione forte} hanno le
proprietà:

\begin{itemize}
\item
  \textbf{riflessiva}: se c'è una connessione tra \emph{u} e \emph{v},
  c'è anche tra \emph{v} e \emph{u}
\item
  \textbf{transitiva}: se c'è una connessione tra \emph{u} e \emph{v} e
  tra \emph{v} e \emph{z}, allora c'è anche tra \emph{u} e \emph{z}.
\end{itemize}

Un sotto-grafo del grafo \emph{G=(V,E)} è un grafo \emph{G' = (V', E')}
tale che:

$$
V' \subseteq V \: \text{e} \: E' \subseteq \{ uv : uv \in E \text{ e } u,v \in V' \}
$$

ovvero un grafo che ha alcuni vertici e alcuni archi del grafo iniziale.
Da notare che se tolgo un vertice, devo togliere anche tutti gli archi
incidenti in quel vertice.

Se il sotto-grafo viene ottenuto rimuovendo solo dei vertici, questo
prende il nome di \textbf{indotto}, perché la rimozione degli archi
viene forzata dalla rimozione dei vertici.

\subsection{Rappresentazione dei
grafi}\label{rappresentazione-dei-grafi}

Per rappresentare i grafi in un calcolatore è possibile utilizzare la
matrice delle adiacenze o la lista delle adiacenze.

\begin{figure}[htbp]
\centering
\includegraphics[width=0.75\textwidth]{./notes/immagini/l2-rappr.png}
\caption{Rappresentazione dei grafi}
\end{figure}

\subsubsection{~Lista delle adiacenze}\label{lista-delle-adiacenze}

Per ogni vertice del grafo viene tenuta in memoria una lista \textit{Adj} dei vertici
adiacenti al vertice:

$$
Adj[u] = \{v | uv \in E\} \: \forall u \in V 
$$

Questa rappresentazione richiede memoria per:

\begin{itemize}
\tightlist
\item
  \textit{n\ =\ \textbar{}V\textbar{}} puntatori alla cima delle liste
\item
  \textit{m\ =\ \textbar{}E\textbar{}} elementi per le liste (in totale)
  se il grafo orientato, se è non orientato è \textit{2m}.
\end{itemize}

\subsubsection{Matrice delle adiacenze}\label{matrice-delle-adiacenze}

Viene utilizzata una matrice booleana quadrata che tante righe e tante
colonne, quanti sono i vertici del grafo.

Ogni elemento della matrice vale 1 se i due vertici sono adiacenti, 0
altrimenti:

$$
a_{u,v} = 1 \text{ se } uv \in E
$$

Se il grafo è non orientato, la matrice delle adiacenze è simmetrica.

Il consumo di memoria è $n^2$.

Se il grafo è \textbf{sparso}, ovvero il grado dei vertici è minore del
logaritmo del numero dei vertici, la matrice delle adiacenze risulta
peggiore della rappresentazione con liste in termini di memoria
occupata.

Più formalmente, assumendo che il grafo abbia \textit{n} vertici e \textit{m} archi e che, sia i puntatori, sia gli interi, occupino 32 bit.

Si ha che la lista delle adiacenze occupa $32(n+2m)$, mentre la matrice richiede $n^2$.

La matrice risulta quindi vantaggiosa quando:

\begin{align*}
	32(n+2m) &< n^2 \\
	m &< \frac{n(n-32)}{64}
\end{align*}


\subsection{Calcolo del grafo trasposto}\label{calcolo-del-grafo-trasposto}

Dato un grafo orientato \emph{G=(V,E)} si vuole ottenere $ G^T = (V, E^T)$ in modo che gli archi siano rovesciati, ovvero $E^T = \{uv | vu \in E\}$.

Utilizzando la rappresentazione con la matrice delle adiacenze, è
necessario attraversare metà della matrice e mettere a 1 la cella
\emph{i,j} se \emph{j,i} è a 1. La complessità risulta quindi essere
$O(n^2)$.

Con la lista delle adiacenze l'algoritmo risulta essere


\begin{algorithm}
	\begin{algorithmic}
		\Function{Trasponi}{$Adj,\: Adj^T,\: n$}
			\For{$v = 1 \: to \: n$}
				\State $Adj^T[v] \gets nil$
			\EndFor
			\For{$u = 1 \: to \: n$}
				\State {$x \gets Adj[u]$}
				\While{$x \neq nil$}
					\State{$v \gets x.v$}
					\State{$y \gets nodo(u, Adj^T[v])$}
					\State{$Adj^T[v] \gets y$}
				\EndWhile
			\EndFor
		\EndFunction
	\end{algorithmic}
	\caption{Calcolo del grafo trasposto utilizzando la rappresentazione con la lista delle adiacenze}
\end{algorithm}

Ovvero viene attraversata la lista delle adiacenze del grafo originale,
e per ogni elemento delle liste, lo aggiunge ``\emph{al contrario}''
nella nuova lista delle adiacenze.

La complessità risulta quindi essere \emph{O(m+n)}, questo perché il
secondo \texttt{for} esamina tutti i possibili archi, quindi anziché
avere complessità \emph{n} (numero di vertici) ha complessità \emph{m}
(numero di archi).

\subsection{(Esercizio) Ricerca del pozzo
universale}\label{esericizio-ricerca-del-pozzo-universale}

Un vertice è un \textbf{pozzo universale} se può essere raggiunto da
tutti gli altri vertici del grafo, dal quale però non è possibile
raggiungere altri vertici.

Trovare un algoritmo che riesce a risolvere il problema in \emph{O(n)}.

% !TEX encoding = UTF-8
% !TEX TS-program = pdflatex
% !TEX root = data mining.tex
% !TEX spellcheck = it-IT
\chapter{Modello lineare semplice}\label{lezione-3---modello-lineare-semplice}

\begin{itemize}
\item
  \textbf{OLTP}: strumenti di interrogazione su specifiche informazioni
  da rihicedere ai vari database, detti operativi
\item
  \textbf{OLAP}: \ldots{}
\item
  \textbf{KDD}: Knowledge discovery database, si parte da uno o più
  database operativi per costruirne uno strategico, il data whare house;
  questa costruzione comporta anche un'operazione di omogeneizzazione di
  definizione di variabili e operazioni di pulizia dei dati (\textbf{data
  mining analitico}).
\end{itemize}



\textbf{Modello}: (o algoritmo) rappresentazione semplificata del
fenomeno di interesse, funzionale ad un obiettivo specifico.

Non esiste un modello vero in quanto si tratta di approssimazioni molto
dettagliate, ci sono dei modelli che in determinati contesti risultano
migliori di altri. Specialmente in ambiti non scientifici, il criterio
per la bontà di un modello è il \textbf{basta che funzioni}, questo
perché tipicamente i dati che vengono utilizzati non sono stati raccolti
con un criterio sperimentale.

Tipicamente il modello viene visto come una scatola nera, funziona ma
non si sa quale sia il vero meccanismo che regola il fenomeno. Tuttavia,
questa black box deve essere comunque manutenuta, non basta avere
solamente l'hardware e il software.

\section{Il modello lineare semplice}\label{il-modello-lineare-semplice}

Si parte da due variabili e si costruisce un modello che li mette in
relazione tra loro.

Il data set di riferimento riguarda i dati di 200 mercati su cui opera
un'azienda, per i quali si conosce la quantità di merce venduta in
migliaia e il budget speso per la pubblicità radiofonica in quella
zona.

Si vuole ottenere un'equazione che permetta di prevedere le vendite in
funzione del budget.

Il primo passo è quello del costruire un grafico di dispersione, dove
nelle \emph{x} ci sono le spese e nelle \emph{y} ci sono le unità
vendute.

\begin{figure}[htbp]
\centering
\includegraphics{./notes/immagini/l3-figura1.png}
\caption{Grafico di dispersione per il data set delle vendite}
\end{figure}

Dal grafico si può osservare un andamento lineare che può essere
approssimata con:

$$
vendite = \beta_0 + \beta_1 (radio) + (errore)
$$

dove la componente \emph{errore} esprime la parte delle vendite non
legate alle pubblicità via radio.

Un modello di questo tipo prende il nome di \textbf{modello di
regressione lineare semplice}.

La variabile \emph{y} (\emph{vendite}) prende il nome di
\textbf{variabile risposta/dipendente/output} mentre \emph{x}
(\emph{radio}) prende il nome di \textbf{variabile
esplicativa/indipendente/input} e i vari coefficiente $\beta$
prendono il nome di \textbf{parametri}.

Il gioco adesso diventa quello di andare a trovare dei valori
$\hat{\beta}_0$ e $\hat{\beta}_1$ che approssimano la retta nel miglior modo
possibile.

 La ricerca avviene utilizzando i dati presenti nel data set:


\begin{align*}
	y_1 &\approx \hat{\beta}_0 + \hat{\beta}_1 x_1 \\
	y_2 &\approx \hat{\beta}_0 + \hat{\beta}_1 x_2 \\
	&\ldots \\
	y_n &\approx \hat{\beta}_0 + \hat{\beta}_1 x_n
\end{align*}

Raffinando l'idea si ottiene il metodo dei \textbf{minimi quadrati}

$$
s^2(\beta_0, \beta_1) = \sum\limits_{i=1}^n(y_i - \beta_0 - \beta_1 x_i)^2
$$

e si vanno a cercare i parametri che minimizzano l'errore di stima
ai minimi quadrati.

$$ s^2(\hat{\beta}_0,\hat{\beta}_1) \leq s^2(\beta_0, \beta_1) $$

Si utilizza il quadrato della distanza, sia per rendere l'errore
indipendente dal segno, sia per dare maggior peso ad errori maggiori.

Ci sono un po' di barbatrucchi matematici per trovare il minimo quello
che interessa è:

\begin{align*}
	\beta_1 &= \frac{\sum\limits_{i=1}^{n} (x_i - \bar{x})(y_i - \bar{y})}{ \sum\limits_{i=1}^{n} (x_i - \bar{x})^2} \\
	 &= \frac{cov(X,Y)}{var(X)} \\
	 \: \\
	\beta_0 &= \bar{y} - \beta_1 \bar{x}
\end{align*}

Da notare che nel lato pratico non ci sarà mai un dataset con varianza
nulla, perché in quel caso il problema di regressione non ha senso. Per
il dataset delle vendite si ottiene come retta ai minimi quadrati

\begin{figure}[htbp]
\centering
\includegraphics{./notes/immagini/l3-figura5.png}
\caption{Retta ai minimi quadrati per il dataset delle vendite}
\end{figure}

\subsection{Residui}\label{residui}

Non è detto che la retta ai minimi quadrati approssimi bene i dati.

Un indicatore dell'andamento è dato dai \textbf{residui}, che
rappresentano la differenza tra i valori osservati e quelli ottenuti
utilizzando la retta.

$$ r_i = y_i - \hat{\beta}_0 - \hat{\beta}_1x_i \: \: \forall i = 1, \ldots, n$$

Per costruzione della retta, la somma di tutti i residui risulta essere
0:

\begin{align*}
	\sum\limits_{i=1}^{n} r_i &= \sum\limits_{i=1}^{n} (y_i - \hat{\beta}_0 - \hat{\beta}_1x_i)\\
												&= \sum\limits_{i=1}^{n} y_i -n\hat{\beta}_0 - \hat{\beta}_1\sum\limits_{i=1}^{n}x_i\\
												&= n\bar{y} -n(\bar{y}-\hat{\beta}_1\bar{x}) - n \hat{\beta}_1 \bar{x}\\
												&= 0
\end{align*}

Quindi per valutare la retta è necessario utilizzare la
\textbf{varianza} dei residui. Minore è la varianza, migliore è la
retta.

\begin{align*}
	var(r_1, \ldots, r_n) &= \frac{1}{n} \sum\limits_{i=1}^{n} (r_{i} - 0)^2 \\
									  &= var(Y) - \frac{cov^2(X,Y)}{var(X)}
\end{align*}

Nel caso peggiore, la varianza dei residui ha come bound superiore la
varianza della risposta, ovvero \emph{var(Y)}.

Più piccola è la varianza dei residui, più la retta di regressione ``spiega'' le variazioni della risposta.

È quindi possibile definire il \textbf{coefficiente di determinazione}:

$$
R^2 = 1 - \frac{var(r_i, \ldots, r_n)}{var(Y)}
$$

$R^2$ varia tra 0 e 1, dove 1 è il valore ottimo e 0 è il valore
peggiore.

Da notare che ciò vale solo per la retta.

%\appendix


%----------------------------------------------------------------------------------------
% BIBLIOGRAPHY
%----------------------------------------------------------------------------------------
%\bibliographystyle{unsrt}
%\bibliography{sample}
%----------------------------------------------------------------------------------------

\end{document}