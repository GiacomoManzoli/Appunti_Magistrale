% !TEX encoding = UTF-8
% !TEX program = pdflatex
% !TEX root = MEMOC.tex
% !TEX spellcheck = it-IT

% 7 Ottobre 2016

%chapter introduzione
%section informazioni pratiche

\chapter{Programmazione lineare}

\section{Modellazione di un problema}

Un \textbf{modello di programmazione matematico} descrive le caratteristiche della soluzione ottima di un problema di ottimizzazione sotto forma di relazione matematiche:

\begin{itemize}
	\item \textbf{Sets}: raggruppano gli elementi del sistema
	\item \textbf{Parametri}: i dati del problema, che rappresentano le quantità note che influiscono sul sistema.
	\item \textbf{Variabili decisionali}: le variabili per le quali si vuole trovare il valore per determinare la soluzione del problema.
	\item \textbf{Vincoli}: le relazioni matematiche che determinano se la soluzione è feasible o meno.
	\item \textbf{Funzione obiettivo}: la funzione che deve essere ottimizzata (massimizzata o minimizzata).
\end{itemize}

I modelli matematici nei quali la funzione obiettivo è lineare e anche i vincoli (constraints) sono un sistema di equazioni o disequazioni lineari, vengono detti lineari.
I modelli vengono poi classificati come:

\begin{itemize}
	\item \textbf{Linear programming model}: tutte le variabili possono assumere valori reali.
	\item \textbf{Integer Linear Programming model}: tutte le variabili possono assumere solo valori interi.
	\item \textbf{Mixed Integer Linear Programming}: le variabili possono assumere sia valori interi che reali.
\end{itemize}

La linearità limita l'espressività ma permette di utilizzare delle tecniche di soluzioni più veloci.

\subsubsection{Esempio di modellazione: I profumi}

Una fabbrica che produce profumi può produrre due nuove tipologie di profumi mescolando tre essente: rose, lilly e violet. Per ogni profumo 1 è necessario utilizzare 1.5 litri di lilly e 0.3 litri di violet. Per ogni profumo 2 è necessario utilizzare 1 litro di rose, 1 litro di lilly e 0.5 litri di viole.
Si hanno in magazzino 27 litri di rose, 21 litri di lilly e 9 litri di rose. La fabbrica ottiene un profitto di 130 euro per i profumi di tipo 1 e 100 per il tipo 2.
Si vuole determinare la quantità ottima di profumi da produrre per massimizzare i guadagni.

\begin{align*}
&\max 130 x_{one} + 100 x_{two} & & \quad\text{funzione obiettivo} \\
\text{s.t. } &1.5 x_{one} + x_{two} &\leq 27 &\\
					&x_{one} + x_{two} &\leq 21 &\\
					&0.3 x_{one} + 0.5 x_{two} &\leq 9 &\\
					&x_{one}, x_{two} &\geq 0 & \quad\text{dominio}
\end{align*}

\subsubsection{Optimal Production Mix}

Il problema precedentemente trattato è simile a quello del contadino e del telefono.
Si può quindi identificare un classe di modelli che risolve problemi simili, in questo caso legati alla massimizzazione del guadagno sotto dei vincoli di consumo di risorse.

Si possono quindi individuare:

\begin{itemize}
	\item Set $I$: contente le risorse, precedentemente $I = \{\text{rose, lilly, violet}\}$
	\item Set $J$: contente le tipologie di prodotti che possono essere prodotti.
	\item Parametri $D_i$: disponibilità della risorsa $i \in I$.
	\item Parametri $P_j$: profitto generato dalla vendita del prodotto $j \in J$.
	\item Parametri $Q_{i,j}$: risorse di tipo $i \in I$ necessarie per ogni unità di prodotto $j \in J$.
	\item Variabili $x_j$: quantità di prodotto $j \in J$ da produrre.
\end{itemize}

I vincoli risultano quindi essere:

\begin{align*}
	\max &\sum\limits_{j \in J} P_j x_j &\\
	\text{s.t. } &\sum\limits_{j \in J} Q_{i,j} x_j \leq D_i \quad \forall i \in I \\
	&x_j \in \mathbb{R}_{+} [\: \mathbb{Z}_{+} \:|\: \{0,1\} \:] \forall j \in J
\end{align*}

\subsubsection{Minimum cost covering}

Un'altra categoria di problemi sono quelli nei quali c'è una funzione obiettivo da minimizzare, con dei vincoli che impongono di stare sopra determinate soglie.
Un esempio di questa categoria è il problema della dieta: si vuole preparare una dieta economica, garantendo comunque di assumere un quantitativo minimo di nutrienti.

Gli elementi tipici di questo modello sono:

\begin{itemize}
	\item Set $I$: risorse disponibili.
	\item Set $J$: materiale richiesto.
	\item Parametri $C_i$: costo unitario della risorsa $i \in I$.
	\item Parametri $R_j$: quantitativo richiesto del materiale $j \in J$.
	\item Parametri $A_{i,j}$: quantità richiesta del materiale $j \in J$ soddisfatta dalla risorsa $i \in I$.
	\item Variabili $x_i$: quantità della risorsa $i \in I$.
\end{itemize}

\begin{align*}
	\min &\sum\limits_{i \in I} C_i x_i \\
	\text{s.t. }& \sum\limits_{i \in I} A_{i,j} x_i \geq D_j \quad\forall j \in J \\
	&x_i \in \mathbb{R}_+ [\:\mathbb{Z}_+ \:|\: \{0,1\}\:] \quad \forall i \in I
\end{align*}

\paragraph{Esempio: La dieta}

Bisogna preparare una dieta che fornisce almeno 20mg di proteine, 30mg di ferro e 10mg di calcio. Gli alimenti che possono essere acquistati sono verdure (5mg/Kg di proteine, 6mg/Kg di ferro e 5mg/Kg di calcio al costo di 4E/Kg), carne (15mg/Kg di proteine, 10mg/Kg di ferro e 3mg/Kg di calcio al costo di 10E/Kg) e frutta (4mg/Kg di proteine, 5mg/Kg di ferro e 12mg/Kg di calcio al costo di 7E/Kg). Si vuole trovare la dieta di costo minimo.

\begin{align*}
\min \: & 4x_V + 10 x_M + 7 x_F \\
\text{s.t. } & 5x_V + 15x_M + 4x_F \geq 20 \\
&6x_V + 10 x_M + 5 x_F \geq 30 \\
&5x_V + 3x_M +12x_F \geq 10 \\
&x_V, x_M, x_F \geq 0
\end{align*}

\subsubsection{The Transportation problem}

In questo problema ci sono dei produttori di oggetti che devono spostarli verso diverse destinazioni che hanno una determinata richiesta. C'è un costo di trasporto variabile in base a quale sorgente rifornisce quale destinatario e si vuole trovare il piano di trasporto che minimizza i costi.

\begin{itemize}
	\item Set $I$: sorgenti.
	\item Set $J$: destinazioni.
	\item Parametri $O_i$: capacità produttiva della sorgente $i \in I$.
	\item Parametri $D_j$: richiesta della destinazione $j \in J$.
	\item Parametri $C_{i,j}$: costo di trasporto da $i$ a $j$.
	\item Variabili $x_{i,j}$: quantità di materiale da spostare da $i$ a $j$.
\end{itemize}

I vincoli sono:

\begin{align*}
	\min \: & \sum\limits_{i \in I} \sum\limits_{j \in J} C_{i,j} x_{i,j} \\
	\text{s.t. }& \sum\limits_{i \in J} x_{i,j} \geq D_j \quad \forall j \in J \quad \text{vincoli sulla domanda} \\
	& \sum\limits_{j \in J} x_{i,j} \leq O_i \quad \forall i \in I \quad \text{vincoli sulla produzione} \\
	x_{i,j} \in \mathbb{R}_+ [\:\mathbb{Z} \:|\: \{0,1\} \:]
\end{align*}

\paragraph{Esempio: La fabbrica}

Una compagnia produce frigoriferi in tre fabbriche diverse ($A, B, C$) e deve spostarti in 4 magazzini (1,2,3,4). La produzione delle fabbriche è rispettivamente 50, 70 e 20. I magazzini possono contenere 10, 60, 30 e 40 unità. Il costo per spostare i frighi è riportato nella seguente tabella:

\begin{table}[htbp]
	\centering
	\begin{tabular}{|l|l|l|l|l|}
		\hline
		Costo & 1 & 2 & 3 & 4 \\ \hline
		A    & 6 & 8 & 3 & 4\\ \hline
		B    &2 & 3 & 1 &3\\ \hline
		C    & 2 & 4 & 6 &5\\ \hline
	\end{tabular}
\end{table}

Si vuole minimizzare il costo di trasporto.

\subsubsection{Fixed Cost}

Sono problemi del tipo in cui si vogliono prendere delle decisioni riguardo a delle azioni da intraprendere.
Ogni azione ha un costo fisso, ma produce un certo guadagno.
Si vuole determinare quali azioni conviene intraprendere sotto qualche vincolo riguardo le azioni.

\begin{itemize}
	\item Set $I$: possibili azioni.
	\item Parametro $W$: budget a disposizione per le azioni.
	\item Parametri $F_i$: costo fisso da pagare per l'esecuzione dell'azione $i \in I$.
	\item Parametri $C_i$: costo variabile per l'esecuzione dell'azione $i \in I$.
	\item Parametri $R_i$: guadagno dell'azione $i \in i$.
	\item Variabili $x_i$: \textit{``quantità''} dell'azione $i \in I$. Ad esempio: apro 200 $\text{m}^2$ di supermercato nella locazione $i$.
	\item Variabili $y_i$: variabili binarie che prendo il valore 1 se viene intrapresa l'azione $i \in I$.
\end{itemize}

Sono presenti due variabili di costo perché le azioni possono essere del tipo ``\textit{quanti metri quadri di supermercato aprire}''. Azioni di questo tipo hanno tipicamente un costo fisso, legato in questo caso alla costruzione, e un costo variabile, legato alle spese di gestione del supermercato, le quali dipendono dalla sua dimensione.

I vincoli risultano essere:

\begin{align*}
	\max &\sum\limits_{i \in I} R_i x_i \\
	\text{s.t. } &\sum\limits_{i \in I} C_i x_i + F_i y_i \leq W &\quad\text{budget} \\
	&x_i \leq M y_i  \quad \forall i \in I &\quad\text{vincolo \textbf{BigM}}  \\
	&\sum\limits_{i \in I} y_i \leq K &\quad\text{vincolo sulle azioni da intraprendere} \\
	&x_i \in \mathbb{R}_+, y_i \in \{0,1\} \quad \forall i \in I&
\end{align*}

I vincoli \textbf{BigM} servono per collegare tra loro le due variabili.
Così facendo quando il risolutore prova a risolvere il problema impostando una variabile $x_i > 0$ è costretto anche ad impostare la variabile $y_i = 1$, perché altrimenti non raggiungerebbe una soluzione feasible. Allo stesso modo se $y_i$ è 0, anche $x_i$ deve essere 0.

Il valore della costante $M$ non è importante durante la modellazione e viene considerato come un valore grande a piacere.
Nel lato pratico $M$ deve essere sufficientemente grande in modo da non porre vincoli sulle $x_i$, ma allo stesso tempo deve essere il più piccolo possibile, in modo da rendere efficiente la risoluzione del modello.

Una variante di questo modello è quella che prevede un limite superiore $U_i$ alla quantità di azione $i \in I$ che può essere intrapresa. In questo caso si possono sostituire i vincoli BigM con 

$$
x_i \leq U_i y_i \quad \forall i \in I
$$

\paragraph{Esempio: il supermercato}

Una catena di supermercati ha un budget $W$ per aprire dei nuovi negozi. Un analisi preliminare ha identificato un set $I$ di possibili locazioni.
Aprire un negozio in $i \in I$ ha un costo fisso $F_i$ e un costo variabile di $C_i$ per 100 $m^2$ di negozio.
Una volta aperto, il negozio in $i$ garantisce un guadagno di $R_i$ per $100 m^2$. Determinare il sotto insieme di località dove aprire un negozio e la relativa dimensione in modo da massimizzare il guadagno totale, tenendo conto che al massimo $K$ negozi possono essere aperti.

\subsection{Un problema più complesso}

Una compagnia di costruzioni deve spostare delle impalcature da 3 siti di costruzioni in chiusura ($A,B,C$) che devono essere spostate in 3 nuovi siti in apertura ($1,2,3$).
Le impalcature consistono di travi di ferro, nei siti \textit{A, B, C} ci sono rispettivamente 7000, 6000 e 4000 travi di ferro, mentre i nuovi siti richiedono rispettivamente 8000, 5000 e 4000 travi. 

Spostare le travi ha un certo costo riportato in tabella.

\begin{table}[htbp]
	\centering
	\begin{tabular}{|l|l|l|l|}
		\hline
		Costo & 1 & 2 & 3 \\ \hline
		A    & 9 & 6 & 5 \\ \hline
		B    & 7 & 4 & 9 \\ \hline
		C    & 4 & 6 & 3 \\ \hline
	\end{tabular}
\end{table}

Per spostare le travi vengono utilizzati dei camion. Ogni camion può trasportare 10000 travi.
Trovare un modello che permette di minimizzare i costi, tenendo conto di:

\begin{itemize}
	\item Usare un camion aggiunge un costo di 50 euro.
	\item Possono essere usati solo 4 camion, e ognuno può essere usato solo una volta.
	\item Le travi che arrivano al sito 2 non possono venire da $A$ e $B$.
	\item Si può noleggiare un quinto camion al costo di 65 euro.
\end{itemize} 

\subsubsection{Modellazione}

Alla base di tutto c'è un problema di trasporto, al quale vengono aggiunti dei vincoli extra che riguardano i camion.

\begin{itemize}
	\item $I = \{A, B, C\}$.
	\item $J = \{1,2,3\}$.
	\item $C_{i,j}$ costo per il trasporto di una singola trave da $i$ a $j$.
	\item $D_i$: numero di travi disponibili nel sito $i$.
	\item $R_j$: numero di travi richieste dal sito $j$.
	\item $F$: costo fisso per l'utilizzo di un camion.
	\item $L$: costo di noleggio del camion extra.
	\item $N$: numero di camion.
	\item $K$: capacità di un camion.
	\item $x_{i,j}$: variabile che indica la quantità di travi portata dal sito $i$ al sito $j$.
	\item $y_{i,j}$: variabile binaria che indica l'utilizzo di un camion per portare delle travi dal sito $i$ al sito $j$.
	\item $z$: variabile binaria che indica l'utilizzo del camion extra.
\end{itemize}

\begin{align*}
\min &\underbrace{\sum\limits_{i \in I, j \in J} C_{i,j} x_{i,j}}_{\text{costo fisso di trasporto}}+ F \overbrace{
\sum\limits_{i \in I, j \in J} y_{i,j}}_{\text{costo variabile per l'utilizzo di un camion}} + \underbrace{(L-F)z}_{\text{costo extra del noleggio}} \\
\text{s.t. }&\sum\limits_{i \in I} x_{i,j} \geq R_j \quad\forall j \in J\\
&\sum\limits_{j \in J} x_i,j \leq D_i \quad\forall i \in I \\
&x_{i,j} \leq K y_{i,j} \quad\forall i \in I, j \in J \quad\text{vincoli di collegamento delle variabili}\\
&\sum\limits_{i \in I, j \in J} y_{i,j} \leq N + z \quad\text{vincolo sul numero di camion}\\ 
&\text{Domini:} \\
&x_{i,j} \in \mathbb{Z}_+  \quad\forall i \in I, j \in J  \\
&y_{i,j} \in \{0,1\}  \quad\forall i \in I, j \in J  \\
&z \in \{0,1\}
\end{align*}