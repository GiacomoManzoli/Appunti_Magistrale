% !TEX encoding = UTF-8
% !TEX program = pdflatex
% !TEX root = MEMOC.tex
% !TEX spellcheck = it-IT

% 20 Ottobre 2016
% Section Modellazione di un problema
% Subsection Lettura dei giornali
% Subsubsection Modellazione

% esercizio delle costruzione della barca.
\subsection{Costruzione di una barca (Esercizio 3)}

La costruzione di una barca da diporto comporta il completamento delle operazioni indicate nella tabella che segue, che ne riporta anche la durata in giorni.

\begin{table}[htbp]
	\centering
	\begin{tabular}{ccc}
		\textbf{Operazione} & \textbf{Durata} & \textbf{Precedenze} \\ \hline
		A                   & 2               & nessuna             \\ 
		B                   & 4               & A                   \\ 
		C                   & 2               & A                   \\ 
		D                   & 5               & A                   \\ 
		E                   & 3               & B,C                 \\ 
		F                   & 3               & E                   \\ 
		G                   & 2               & E                   \\ 
		H                   & 7               & D,E,G               \\ 
		I                   & 4               & F,G                 \\ 
	\end{tabular}
\end{table}

Si consideri che alcune operazioni sono in alternativa. In particolare, bisogna eseguire solo una tra le operazioni B e C, e solo una tra le operazioni F e G. Inoltre, se si eseguono sia C che G, la durata dell'operazione I si allunga di 2 giorni.
La tabella indica anche, per ogni operazione, l'insieme delle precedenze (operazioni che devono essere completate prima di poter eseguire l'operazione stessa).
Scrivere un modello di programmazione lineare per decidere quali operazioni in alternativa eseguire, con l'obiettivo di minimizzare la durata complessiva delle operazioni di costruzione.

\subsubsection{Modellazione}

Le scelte in questo caso riguardano quali attività svolgere, tra quelle che possono essere eseguite in alternativa al fine di minimizzare il makespan.

Dal momento che si vuole minimizzare la durata, conviene scegliere dopo quanti giorni dall'inizio dei lavori deve terminare una determinata attività:

$$
t_i \quad \text{dopo quanti giorni termina l'attività }i \in A = \{A, \ldots, I \}
$$

\noindent Così risulta facile definire la funzione obiettivo

$$
\min z
$$

\noindent dove \textit{z} è il makespan, ovvero il giorno in cui termina l'ultima attività da eseguire. Per specificare ciò nel modello serve il vincolo

$$
z \geq t_I
$$

\noindent \`E necessario poi modellare le varie precedenze tra le attività e il fatto che non due attività non possono essere eseguite in parallelo.
Questo viene fatto con una serie di vincoli del tipo:

$$
t_i \geq t_{j} + d_{i} \forall \: i \in A, j \in prec(i)
$$

\noindent dove $d_i$ è la durata dell'attività $i$ e $prec(i)$ è l'insieme delle attività che devono essere svolte prima di $i$. Ad esempio: $prec(H) = \{D,E,G\}$.

Resta poi da modellare il fatto che alcune attività possono essere svolte in alternativa. In questo servono delle variabili binarie $y_i$, una per ogni attività che può essere eseguita e che indicano l'attività viene svolta o meno.

Per vincolare la scelta tra due attività è necessario aggiungere i vincoli del tipo

$$
y_i + y_j = 1
$$

\noindent dove $i$ e $j$ sono due attività che possono essere eseguite in alternativa.

\`E inoltre necessario modificare alcuni dei vincoli riguardo le precedenze, perché se un'attività non viene svolta, questa non deve essere presa in considerazione nella pianificazione:

$$
t_i \geq t_{j} + d_i - M(1 - y_i)
$$ 

\noindent Così facendo, se la soluzione prevede che l'attività $i$ non venga svolta ($y_i = 0$), il vincolo diventa ridondante e non va ad influenzare il makespan.
Con i dati del problema alcuni di questi vincoli sono:

\begin{align*}
	t_B &\geq t_{A} + d_B - M(1 - y_B) \\
	t_C &\geq t_{A} + d_C - M(1 - y_C)
\end{align*}

C'è inoltre da modellare il fatto che se vengono eseguite determinate attività la durata di altre attività aumenta.

Serve quindi una variabile booleana $c$ che specifica se questa condizione si verifica. Questa variabile viene poi utilizzata per aggiornare i vincoli relativi alle attività interessate. Ad esempio per i dati del problema si ha

\begin{align*}
	&y_C + y_G \leq 1 + c \quad \text{attivazione di \textit{c}} \\
	&t_I \geq t_F + d_I + 2c \quad \text{aumento della durata per l'attività \textit{I} se vale \textit{c}} \\ 
	&t_I \geq t_G + d_I + 2c \quad \text{''} \\
\end{align*}

\noindent Rimane infine da specificare i domini delle variabili:

\begin{align*}
	t_i &\in \mathbb{R} \: \forall \: i \in A \\
	y_i, c, &\in \{0,1\} \\
	z &\in \mathbb{R}
\end{align*}

\noindent Servono poi i parametri $d_i \in \mathbb{R}$ che rappresentano le durate e la costante $M$ che rappresenta un numero sufficientemente grande in grado di rendere ridondanti i vincoli in cui compare.

\subsection{Turni delle farmacie (Esercizio 5)}

La federazione dei farmacisti vuole organizzare i turni festivi delle farmacie sul territorio regionale. 
\`E stabilito a priori il numero dei turni, che devono essere bilanciati in termini di numero di farmacie, considerando che ciascuna farmacia deve appartenere, per equità, a un solo turno. 
Ad esempio, se il numero complessivo di farmacie è 12 e si vogliono organizzare tre turni, ciascun turno sarà formato da quattro farmacie. 
Sia le farmacie che gli utenti si considerano distribuiti sul territorio e concentrati in centroidi (corrispondenti in genere con comuni o quartieri). 
Per ogni centroide sono noti il numero di utenti e il numero di farmacie. \`E inoltre nota la distanza tra ogni coppia ordinata di centroidi. 
In prima istanza, si trascurano problemi relativi alla congestione e si assume che gli utenti, in ciascun turno, si servano dalla farmacia aperta più vicina. 
Si vuole determinare la distribuzione dei turni festivi che minimizza la distanza complessiva percorsa dagli utenti per il servizio festivo.

\subsubsection{Modellazione}

In questo caso vogliamo decidere quale farmacia fa quale turno, in modo che ci sia una buona copertura del territorio, assumendo che le persone vadano nella farmacia più vicina.

L'obiettivo è quindi quello di minimizzare la strada che devono fare le persone per raggiungere le farmacie di turno.

Di sicuro serve una variabile che specifica quale farmacia è aperta in quale turno.

$$
y_{i,k} = \begin{cases}
1 \quad &\text{la farmacia \textit{i} è aperta nel turno \textit{k}} \\
0 \quad &\text{altrimenti}
\end{cases}
$$

\noindent con $i \in P$ e $k \in 1 \ldots K$. Dove $P$ è l'insieme delle farmacie e $K$ è il numero di turni che si voglio fare.

Per esprimere la nostra funzione obiettivo servono altre variabili, perché dobbiamo anche prendere in considerazione la distanza delle farmacie, in modo da poterla minimizzare.

Ci sarà quindi il set $C$ di clienti che devono essere serviti e dei parametri che specificano la distanza $D_{j,i}$ che c'è tra un cliente $j \in C$ e la farmacia $i \in P$.
Tuttavia la distanza che l'utente deve fare \textbf{dipende dalle farmacie aperte} in un determinato turno e quindi non conviene utilizzare direttamente il parametro, in quanto in base al turno la distanza è variabile.

Conviene quindi aggiungere una variabile che specifica quanta strada il cliente $j$ deve fare durante il turno $k$ per raggiungere la farmacia più vicina aperta.

$$
d_{j,k} \: \text{distanza tra il cliente \textit{j} e la farmacia più vicina durante il turno \textit{k}}
$$

\noindent Bisogna però in qualche modo collegare le variabili $d_{j,k}$ con l'apertura/chiusura delle farmacie.

Serve quindi un modo per discriminare in quale farmacia va l'utente in un determinato turno:

$$
x_{j,i,k} = \begin{cases}
1 \quad & \text{se \textit{j} va nella farmacia \textit{i} durante il turno \textit{k}} \\
0 \quad &\text{altrimenti}
\end{cases}
$$

\noindent Così facendo risulta semplice trovare un valore per i $d_{j,k}$, perché basta il vincolo:

$$
d_{j,k} = \sum\limits_{i \in P} D_{j,i} x_{j,i,k} \quad \forall \:j \in C, k \in K
$$

\noindent Con questo vincolo viene presa in considerazione solo una distanza per ogni turno, perché durante un turno il cliente va sempre nella farmacia più vicina e quindi, fissati un $j$ e un $k$, ci sarà solo un $x_{j,i,k}$ che vale 1.
Quest'ultima cosa il risolutore non lo sa e quindi bisogna aggiungere gli opportuni vincoli:

$$
\sum\limits_{i \in P} x_{j,i,k} = 1 \quad \forall \: j \in C, k \in K
$$

\noindent Manca ancora il vincolo che ogni farmacia faccia esattamente un turno, il quale può essere semplicemente aggiunto con una sommatoria sulle $y_{i,k}$:

$$
\sum\limits_{k = 1}^{K} y_{i,k} = 1 \quad \forall \: i \in P
$$

\noindent Per completare il modello rimane da collegare le $x$ con le $y$, perché ovviamente un cliente non può andare in una farmacia chiusa.

$$
x_{j,i,k} \leq y_{j,k} \quad \forall \: i,j,k
$$

\noindent La funzione obiettivo risulta quindi essere:

$$
\min \sum\limits_{j \in C}\sum\limits_{k = 1}^{K} d_{j,k}
$$

\noindent Rimane inoltre da imporre che ogni turno sia bilanciato, ovvero che ci sia sempre un numero simile di farmacie aperte:

$$
\bigg\lfloor \frac{|P|}{K} \bigg\rfloor \leq \sum\limits_{i \in P} y_{i,k} \leq  \bigg\lceil \frac{|P|}{K} \bigg\rceil \quad \forall \: k
$$

\noindent Rimane da specificare i domini delle variabili:

\begin{align*}
	y_{i,k} &\in \{0,1\} \\
	x_{j,i,k} &\in \{0,1\} \\
	d_{j,k} &\in \mathbb{R}
\end{align*}

\noindent Peccato che ci sia un problema. Con i vincoli attuali abbiamo espresso che per ogni turno un cliente va sempre nella stessa farmacia e che quella farmacia deve essere aperta, ma non viene specificato che il cliente va alla farmacia più vicina.

In realtà questo non è un problema, perché è durante il processo di ottimizzazione che viene impostata che le varie distanze vengono minimizzate.

Questo perché \textbf{l'obiettivo di un modello è quello di descrivere le caratteristiche di una soluzione}, mentre è il risolutore che cercando la soluzione ottima effettua la minimizzazione. Infatti, una soluzione che manda un cliente in una farmacia diversa da quella aperta che gli è più vicina, è comunque una soluzione accettabile, ma di sicuro non è ottima e quindi viene scartata.

\paragraph{Osservazione - Simmetrie}

Una volta trovata una soluzione ottima per questo problema si può osservare che permutando l'ordine dei turni ottenuto si ottiene un'altra soluzione ottima con un ordine diverso.

Questo è causato dal fatto che una volta scelte le farmacie che sono aperte nei vari turni, l'ordine in cui sono effettuati i turni è indifferente, ottenendo così una soluzione simmetrica. 
La presenza di queste simmetrie è tipicamente un problema perché può portare ad un'esplosione combinatoria delle soluzioni.

L'origine di queste simmetrie è tipicamente causata dal modello, in questo caso il problema deriva dal fatto che viene dato ``\textit{un nome}'' ai turni e non sempre è possibile ri-modellare il problema in modo che non ci siano simmetrie.

\subsubsection{Modellazione alternativa}

Dato che abbiamo un'insieme di farmacie $P$ e che ogni farmacia fa solo un turno, possiamo vedere un turno come un sottoinsieme di $P$.

La scelta dei turni diventa quindi una scelta di quali sottoinsiemi selezionare dall'insieme delle parti $2^P$.

Questa scelta può essere modellata con una variabile binaria

$$
x_J = \begin{cases}
1 \quad& \text{se il sottoinsieme \textit{J} è un turno} \\
0 \quad&\text{altrimenti}
\end{cases} \quad \forall \: J \subset P, J \in 2^P
$$

\noindent Con questa variabile non ci sono simmetrie in quanto la variabile è direttamente collegata al turno che rappresenta.

La minimizzazione da fare diventa quindi ($j$ rappresenta i clienti, $J$ il turno)

$$
\min \sum\limits_{J \in 2^P} \sum\limits_{j \in C} D_{j,J}x_J
$$

\noindent Nella funzione obiettivo non compare più la variabile $d_{j,k}$, ma compare un parametro $D_{j,J}$, questo perché nella formulazione precedente la composizione dei vari turni era variabile e di conseguenza anche la distanza cambiava in base alla composizione del turno, mentre con questo nuovo modello so a priori quali sono le farmacie che appartengono ad un determinato turno e quindi per ogni turno e per ogni cliente posso pre-calcolare la distanza minima.

Ci sono poi altri vincoli che devono essere ri-formulati.

Per specificare che si siano esattamente $K$ turni, basta effettuare la sommatoria sulle $x_J$.

$$
\sum\limits_{J \in 2^P} x_J = K
$$

\noindent Bisogna inoltre imporre il vincolo che ogni farmacia faccia esattamente un turno, perché al momento la stessa farmacia può comparire in più turni (sottoinsiemi).

In questo caso serve un'ulteriore \textbf{parametro} che specifichi se una farmacia è in un determinato turno.

$$
A_{i,J} = \begin{cases}
1 \quad &\text{se } i \in J \\
0 \quad &\text{altriment}
\end{cases} \quad \forall \: J \in 2^P
$$

\noindent Da notare che è un parametro e non una variabile perché è un valore che può essere pre-calcolato quando viene costruito l'insieme delle parti.

Con questi parametri risulta semplice porre il vincolo che una farmacia faccia al massimo un turno.

$$
\sum\limits_{J \in 2^P} A_{i,J} x_J = 1 \quad \forall \: i \in P
$$

\noindent Rimane da modellare il fatto che i turni devono essere bilanciati, ma per fare questo non servono nuovi vincoli. Infatti basta considerare, al posto di tutto l'insieme delle parti $2^P$, un suo sottoinsieme $G$ composto solamente dai sottoinsiemi di $P$ che hanno cardinalità simile.

$$
G = \bigg\{x \:  | \: x \in 2^P, \bigg\lfloor \frac{|P|}{K} \bigg\rfloor \leq |\:x\:|\leq  \bigg\lceil \frac{|P|}{K} \bigg\rceil \bigg\}
$$

\noindent Questo modello non ha simmetrie ed è molto semplice, tuttavia soffre di un grande problema: se ci sono $100$ farmacie, il calcolo dell'insieme delle parti di $P$ e dei parametri può richiedere troppo tempo a causa della crescita esponenziale della cardinalità dell'insieme delle parti.









