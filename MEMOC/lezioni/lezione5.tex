% !TEX encoding = UTF-8
% !TEX program = pdflatex
% !TEX root = MEMOC.tex
% !TEX spellcheck = it-IT

% 27  Ottobre 2016
% Section Modellazione di un problema
% Subsection Turni delle farmacie
% Subsubsection Modellazione alternativa

\subsection{Problema sui grafi - Minimum cost network flow}

Una compagni di distribuzione elettrica ha varie stazioni di distribuzione connesse tramite la rete cablata. Ogni stazione \textit{i} può:

\begin{itemize}
	\item produrre $p_i$ kW di energia.
	\item distribuire energia in una sotto rete che ha domanda $d_i$ kW.
	\item trasferire energia da una stazione ad un'altra.
\end{itemize}

\noindent i cavi che collegano la stazione $i$ alla stazione $j$ hanno una capacità massima di $u_{i,j}$ e un costo $c_{i,j}$ euro per ogni kW di energia trasportata dai cavi.

La compagnia vuole determinare il piano di distribuzione di costo minimo, sotto l'assunzione che la totalità dell'energia prodotta sia pari a quella richiesta da tutte le sotto reti.

\subsubsection{Modellazione}

Per pianificare la distribuzione dobbiamo decidere la quantità di energia che viene trasferita da una stazione all'altra.

$$
x_{i,j} = \text{ quantità di energia trasferita da \textit{i} a \textit{j}}
$$

\noindent Una caratteristica interessante di questo problema è che può essere modellato come un grafo $G = (N,A)$ i cui nodi corrispondo alle stazioni energetiche e gli archi rappresentano le connessioni tra le varie stazioni.

Per semplificare la modellazione del problema è possibile aggiungere un parametro $b_v$ per ogni nodo $v \in N$ della rete che rappresenta la differenza tra la domanda che al stazione deve soddisfare e la quantità di energia che può produrre:

\begin{itemize}
	\item se $b_v$ è un valore positivo, la domanda è superiore alla capacità della stazione e quindi è necessario trasferire energia da altre stazioni.
	\item se $b_v$ è un valore negativo, la stazione produce più energia di quella necessaria e quindi l'energia in eccesso deve essere inviata alle altre stazioni.
	\item se $b_v = 0$, la stazione è autosufficiente oppure è un nodo di trasmissione perché $p_v = d_v = 0$.
\end{itemize}

\noindent Risulta semplice definire la funzione obiettivo:

$$
\min \sum\limits_{(i,j) \in A} c_{i,j} x_{i,j}
$$

\noindent Bisogna ora porre il vincolo che vengano ricevuti da ogni nodo esattamente $b_v$ unità di flusso (negative se devono essere tolte) (\textbf{node balance constraint}).

$$
\underbrace{\sum\limits_{(i,v) \in A} x_{i,v}}_{\text{flusso in ingresso}} - \underbrace{\sum\limits_{(v,j) \in A} x_{v,j}}_{\text{flusso in uscita}} = b_v \quad \forall \: v \in N
$$ 

\noindent Infine, è necessario imporre il limite sulla capacità dei cavi (\textbf{arc capacity constraint}):

$$
x_{i,j} \leq u_{i,j} \quad (i,j) \in A
$$

\paragraph{Shortest path problem} Un altro problema tipico dei grafi è quello di cercare il percorso di lunghezza minima tra due nodi con dei costi variabili assegnati agli archi.
Questo problema può essere visto come una variante del problema precedente. Basta porre 1 unità di flusso nel nodo di partenza, -1 nel nodo destinazione e 0 in tutte le altre. Quest'unità rappresenta la persona che si deve spostare.
Possono però essere presenti più percorsi ottimi e quindi è necessario rendere le $x_{i,j}$ delle variabili binarie.

\paragraph{SP + Flow} Una combinazione dei due problemi può essere quella di dover pianificare il flusso di energia nella rete, soddisfacendo il vincolo che la distanza massima percorsa dal flusso sia $H$ archi.
In questo caso le $x_{i,j}$ non possono essere binarie, perché devono modellare la quantità di flusso. \`E quindi necessario introdurre una nuova variabile binaria $y_{i,j}$ per definire i vincoli:

$$
\underbrace{\sum\limits_{(i,j) \in A} y_{i,j} \leq H}_{\text{vincolo sulla lunghezza massima}} \qquad x_{i,j}\leq M y_{i,j} \qquad x_{i,j} \leq u_{i,j} y_{i,j}
$$

\paragraph{Diversi tipi di energia} Una variante del problema è quella in cui ci sono ogni stazione gestisce vari tipi di energia e il costo di trasporto dipende dal tipo. La capacità degli archi non è influenzata dal tipo di energia che passa.
La risoluzione di questo problema è analoga a quella della versione classica, con la differenza che viene utilizzato un altro indice per i parametri e per le variabili che discrimina il tipo di energia.

\begin{align*}
	\min & \sum\limits_{k \in K}\sum\limits_{(i,j) \in A} c_{i,j}^k x_{i,j}^k \\
	\st & \sum\limits_{(i,v) \in A}x_{i,v}^k - \sum\limits_{(v,j) \in A}x_{v,j}^k = b_{v}^k \quad\forall \: v \in N, k \in K\\
	& \sum\limits_{k \in K} x_{i,j}^k \leq u_{i,j}^k \quad\forall \: (i,j) \in A
\end{align*}

\noindent Questa variante del problema prende il nome di \textbf{multi-commodity flow problem} e risulta più complessa da risolvere perché servono molte più variabili e vincoli dato che i flussi dipendono l'uno dall'altro. Se questi fossero indipendenti sarebbe possibile scomporre questo problema in $|K|$ problemi di flusso minimo e poi combinare tra loro le varie soluzioni.



