% !TEX encoding = UTF-8
% !TEX TS-program = pdflatex
% !TEX root = data mining.tex
% !TEX spellcheck = it-IT
\chapter{Introduzione al corso}\label{lezione-1---data-mining}

Con l'evoluzione tecnologica abbiamo raggiunto un punto in cui costa
poco rilevare dati in modo automatico e allo stesso tempo costa poco
memorizzarli in data base, da qui il termine \textbf{big data}.

\textbf{Datawarehouse}: database che contiene tutti i dati su cui fare
mining, i dati contenuti possono essere estratti da vari database che
non sono necessariamente uniformi. Tipicamente non sono necessarie tutte
le informazioni presenti nel DWH, pertanto è necessario tenere in
considerazione gli obiettivi delle analisi in modo da utilizzare
solamente i dati utili (\textbf{datamart}). Avendo a disposizione un
elevato numero di campioni, può tornare utile estrarre un sottoinsieme
casuale.

\section{Struttura del corso}\label{struttura-del-corso}

Parte 1:

\begin{itemize}
\item
  Modello lineare
\item
  Inferenza sui parametri del modello
\item
  Estensione del modello
\item
  Modello lineare
\end{itemize}

Parte 2:

\begin{itemize}
\item
  Tecniche statistiche per l'approccio analitico
\item
  Tecniche generali per la selezione di un modello
\item
  Metodi di previsioni di variabili quantitativi
\item
  Modelli di classificazioni (lineari e generalizzazione) (NN, Alberi,
  ecc.)
\item
  Cenni ai metodi di raggruppamento (apprendimento non supervisionato)
\end{itemize}

Circa metà delle lezioni saranno in laboratorio, utilizzeremo vari
software e la scelta è libera, a condizione che venga raggiunto
l'obiettivo.

Come linguaggio di programmazione utilizzeremo \textbf{R}.

L'esame si concentra su:

\begin{itemize}
\item
  conoscenza delle metodologie studiate
\item
  capacità di applicare
\end{itemize}

Prova parziale alla fine della prima parte con esercizi e domande di
teoria. Al termine della seconda parte un progetto applicativo
utilizzando gli strumenti appresi durante il corso (con relazione e
discussione orale).

\url{homes.stat.unipd.it/bruno/dm-inf }


\url{scarpa@stat.unipd.it}
