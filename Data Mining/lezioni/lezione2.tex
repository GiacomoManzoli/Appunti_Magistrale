% !TEX encoding = UTF-8
% !TEX TS-program = pdflatex
% !TEX root = data mining.tex
% !TEX spellcheck = it-IT
\chapter{Lezione 2 - Introduzione al data
mining}\label{lezione-2---introduzione-al-data-mining}

Il tutto è nato dalla presenza di data set enormi, che hanno portato dei
problemi sia a livello di gestione dei dati, che nell'analisi efficace
dei dati.

L'elevata dimensionalità dei dati porta un problema sia computazionale,
che concettuale, dal momento che risulta più difficile trovare le
correlazioni tra i dati. Sono pertanto necessarie strutture dati più
complesse come reti o alberi.

\section{Data mining}\label{data-mining}

\textbf{Data mining}: attività di elaborazione in forma grafica o
numerica di grandi raccolte o di flussi continui di dati con lo scopo di
estrarre informazione utile a chi detiene i dati stessi.

L'informazione utile può essere caratterizzata come:

\begin{itemize}
\item
  il comportamento complessivo del fenomeno in esame, con l'obiettivo di
  costruire un modello globale a partire dai dati disponibili
\item
  l'individuazione di particolarità, ovvero delle caratteristiche
  specifiche di un piccolo insieme di dati.
\end{itemize}

Il data mining riguarda vari settori:

\begin{itemize}
\item
  l'informatica, per la gestione e il trattamento dei dati
\item
  l'intelligenza artificiale, per la definizione di tecniche automatiche
  per il trattamento dell'informazione (machine learning)
\item
  alla base di tutto c'è pero la statistica che viene applicata per
  trattare i dati.
\end{itemize}

\subsection{Complicanze}\label{complicanze}

\begin{itemize}
  \item Enorme quantità e dimensioni dei dati
\item
  I dati vengono raccolti per scopi diversi da quelli per cui vengono
  analizzati. Generalmente vengono  per esigenze gestionali o
  simili e non per fini statistici (\textbf{ambito osservazionale}).

  \begin{itemize}
  \item
    non esiste un \textbf{piano di campionamento} ma si ha a
    disposizione solo quello che si riesce ad osservare
  \item
    i dati sono sporchi
  \item
    molto spesso ci sono dei dati mancanti
  \end{itemize}
\item
  spesso si va alla ricerca di qualcosa, senza sapere che cosa e si
  trova comunque qualcosa.
\end{itemize}
