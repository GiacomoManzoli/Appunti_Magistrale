% !TEX encoding = UTF-8
% !TEX TS-program = pdflatex
% !TEX root = data mining.tex
% !TEX spellcheck = it-IT
\chapter{Soluzioni Parziali}

\section{Aprile 2015}

\subsection{Esercizio 5}

\subsubsection{Domanda a}


\subsubsection{Domanda b}

Bisogna completare i dati dell'output.

I gradi di libertà per la $ F_{oss} $ sono 2 e 19, rispettivamente $ p-1 $ e $ n-p $ con \textit{p} numero di parametri. I gradi di libertà degli errori sono sempre $ n-p = 19 $.

Per calcolare $ RSE $ serve $ s^2 $ che è la stima della varianza degli errori (e dei residui).

$$
RSE = \sqrt{\frac{SSE}{n-p}} = \sqrt{\frac{\sum\limits_{i=1}^n (y_i - \hat{y}_i)^2}{n-p}} = \sqrt{s^2} 
$$

Sappiamo che:

$$
\widehat{\var(\vec{\hat{\beta}})} = (\textbf{X}^T\textbf{X})^{-1}s^2
$$

e che $ \sqrt{\var (\hat{\beta}_0)} = 126,758$ e $(\textbf{X}^T\textbf{X})_{11}^{-1} = 0,6363  $ da cui segue $ s^2 = \frac{126,758^2}{0,6363} =  25.251,596$, pertanto $ RSE = \sqrt{s^2} = 158,..$.